% Licensed to the Apache Software Foundation (ASF) under one
% or more contributor license agreements.  See the NOTICE file
% distributed with this work for additional information
% regarding copyright ownership.  The ASF licenses this file
% to you under the Apache License, Version 2.0 (the
% "License"); you may not use this file except in compliance
% with the License.  You may obtain a copy of the License at
%
%   http://www.apache.org/licenses/LICENSE-2.0
%
% Unless required by applicable law or agreed to in writing,
% software distributed under the License is distributed on an
% "AS IS" BASIS, WITHOUT WARRANTIES OR CONDITIONS OF ANY
% KIND, either express or implied.  See the License for the
% specific language governing permissions and limitations
% under the License.

\subsection{Cache Groups}
A Cache Group is exactly what it sounds like it is: a group of Cache Servers.
Typically a Cache Group is representative of the available Cache Servers within a
specific geographical location. Despite that Cache Servers have their own
Physical Locations, when Cache Servers are chosen to serve content to a client
based on geographic location the geographic location actually used for
comparisons is that for the Cache Group that contains it, not the geographic
location of the Cache Server itself.

\subsubsection{Cache Servers}
All of the Cache Servers within a Cache Group are represented by a set of their
numeric IDs.

\subsubsection{Latitude}
A Cache Group's Latitude is a floating-point number that simply represents a
Cache Group's geographic latitude on the range [-90, 90] - with the positive
range representing North and the negative range representing South.

\subsubsection{Longitude}
A Cache Group's Longitude is a floating-point number that simply represents a
Cache Group's geographic longitude on the range [-180, 180] - with the positive
range representing East and the negative range representing West.

\subsubsection{Name}
A Cache Group is uniquely identified by its Name, which is a string that MUST
NOT be empty.

\subsubsection{Tags}
The Tags associated with a Cache Group is represented by a set of strings that
are Tag Names.

\subsubsection{Type}
Cache Groups can only contain one Type of Cache Server, which is herein
reflected by the string constant value of said Type.

\begin{itemize}
	\item \code{EDGE} - This Cache Group may only contain Cache Servers that
		are of the \code{EDGE} Type.
	\item \code{MID} - This Cache Group may only contain Cache Servers that are
		of the \code{MID} Type.
\end{itemize}
