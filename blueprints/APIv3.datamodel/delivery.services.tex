% Licensed to the Apache Software Foundation (ASF) under one
% or more contributor license agreements.  See the NOTICE file
% distributed with this work for additional information
% regarding copyright ownership.  The ASF licenses this file
% to you under the Apache License, Version 2.0 (the
% "License"); you may not use this file except in compliance
% with the License.  You may obtain a copy of the License at
%
%   http://www.apache.org/licenses/LICENSE-2.0
%
% Unless required by applicable law or agreed to in writing,
% software distributed under the License is distributed on an
% "AS IS" BASIS, WITHOUT WARRANTIES OR CONDITIONS OF ANY
% KIND, either express or implied.  See the License for the
% specific language governing permissions and limitations
% under the License.

\subsection{Delivery Services}
Delivery Services are, at their most basic, an association between a source of
content and a set of Cache Servers and configuration options used to distribute
that content.

\subsubsection{Common Properties}
Herein described are the properties common to all Delivery Service objects. The
Routing Type of a Delivery Service encapsulates the methods by which clients
may request content routing, and depending on its value the Delivery Service
takes on a set of additional properties. Put simply, these are all different
types of objects that are closely related.\\
This section details all of the properties that are common to \emph{all} types
of Delivery Services.

\paragraph{Anonymous Blocking}
A Delivery Service that has "Anonymous Blocking" tells Traffic Router to block
requests from anonymized IP addresses. Whether or not and how well that can
actually be done is dependent on the configuration of each Traffic Router
itself, and if this Delivery Service is DNS-routed the only IP address Traffic
Routers will be capable of checking for anonymization (e.g. known proxy/VPN/TOR
exit node) will be the downstream router requesting the name resolution and thus
is likely much less effective.

\paragraph{Bypass Destination}
This is a string that describes the network location to which clients will be
directed if the traffic served by this Delivery Service exceeds its allowed
maximums. This MUST always be represented as a string - even if the
representation format supports IP Addresses as a native type - and its
interpretation is dependent on the Delivery Service's Routing Type.\\
If the Delivery Service's Routing Type is \code{HTTP}, then this is interpreted
- and validated by the API - as a Fully Qualified Domain Name (FQDN) optionally
followed by a colon and port number that defines an HTTP server to which client
requests will be directed.\\
If the Delivery Service's Routing Type is \code{DNS} or \code{STATIC}, then if
this is a valid IPv4 address it is assumed to be one and will be presented as an
AA record. If not, and it's a valid IPv6 address, then it is assumed to one and
will be presented as an AAAA record. Finally, it may be an FQDN in which case it
will be presented as a CNAME record. If none of these formats can be validated,
then it MUST be rejected by the API.\\
If the Delivery Service's Routing Type is \code{STEERING}, then this MUST be the
Name of an existing Delivery Service - though it need not name one of this
Delivery Service's Targets.

\paragraph{Deep Caching}
A boolean value that describes whether or not "Deep Caching" may be used for
this Delivery Service.

\paragraph{Caching}
Caching describes how Delivery Service content is cached - if at all. It is a
string content restricted to one of the values:

\begin{itemize}
	\item \code{CACHE} The Delivery Service's content will be cached normally.
	\item \code{RAM\_ONLY} The Delivery Service's content will only be cached
	in RAM block devices.
	\item \code{NO\_CACHE} The Delivery Service's content is proxied through
	Cache Servers without ever being actually cached.
\end{itemize}

\paragraph{CDN}
The CDN to which a Delivery Service belongs is expressed as a string that is the
Name used to uniquely identify it.

%TODO: use-case for check path?

\paragraph{Denied Access Redirect}
This is a string that describes the network location to which clients will be
directed if they are denied access on the basis of Anonymous Blocking and/or
Geographic Limiting settings. This MUST always be represented as a string - even
if the representation format supports IP Addresses as a native type - and its
interpretation is dependent on the Delivery Service's Routing Type.

%TODO: steering and static won't have this?
\paragraph{DSCP}
Sets the
\href{https://tools.ietf.org/html/rfc2474}{Differentiated Services Code Point}
which will be marked on the Delivery Service's traffic. This is an unsigned
integer with a maximum value of 64.

%TODO: Necessary?
%\paragraph{ECS}

\paragraph{Geographic Limiting}
This property describes limitations to the availability of this Delivery
Service's content on the basis of the requesting client's geographic location.
It is a set of strings, each of which is an
\href{https://www.iso.org/obp/ui/#search/code/}{ISO 3166-1} alpha-2 country
code, optionally with ISO 3166-2 subdivisional alphabetic code. This is a "white
list" of countries/subdivisions wherein content is to be made
available\footnote{This property is meant to inform Traffic Router; Cache
Servers cannot be relied upon to approve or deny access on a geographic basis.
Thus, if routing is bypassed, restricted content is totally accessible to
requesting clients.}.\\
Content is \emph{always} available to clients whose IP addresses are found
within the Traffic Routers' Coverage Zone File(s). With that in mind, when this
property is an empty set it means that no geographic regions are "whitelisted"
and thus \emph{only} clients whose IP addresses are found within a Coverage Zone
File will be granted access to content. When this property has a "Null" type,
there is no geographic restriction placed on the Delivery Service's content
access.

%TODO: steering and static won't have this?
\paragraph{Maximum Origin Connections}
This is an unsigned integer which determines the maximum number of connections
that any \emph{one} Cache Server may open to the Delivery Service's Origin.\\
If this is null-typed, it has the special meaning "no limit".

%TODO: use-case for header-rewrite text?

\paragraph{Miss Location}

\paragraph{Name}
A Delivery Service's "Name" is a string that uniquely identifies it among all
Delivery Services. It MUST only contain alphanumerics, hyphens, underscores and
spaces, and MUST NOT begin with a non-alphanumeric character nor end with a
non-alphanumeric character. This is used to generate part of the default request
hostnames by replacing all non-alphanumeric characters with a hyphen.

\paragraph{Notes}
This section is an arbitrary string containing miscellaneous, human-friendly
information about the Delivery Service. Other ATC components SHOULD NOT parse
this for specific information fields, or expect it to be in a particular format.

%TODO: MSO use-case?
\paragraph{Origin}

\paragraph{Query String Handling}

\begin{itemize}
	\item \code{DROP}
	\item \code{IGNORE}
	\item \code{USE}
\end{itemize}

\paragraph{Range Request Handling}
\begin{itemize}
	\item \code{NO\_CACHE}
	\item \code{WHOLE\_OBJECT}
	\item \code{CACHE}
\end{itemize}

%TODO Raw-Remap use-case?
%TODO Regex-Remap use-case?

\paragraph{Required Capabilities}

\paragraph{Routing Name}

%TODO: ANY_MAP use-case?
\paragraph{Routing Type}
After creation, Routing Type is a read-only property.
\begin{itemize}
	\item \code{HTTP}
	\item \code{DNS}
	\item \code{STEERING}
	\item \code{STATIC}
\end{itemize}

\paragraph{Status}
The "Status" of a Delivery Service is a string constant that expresses its
ability to serve content at the present moment in time. It may have one of three
values:

\begin{itemize}
	\item \code{ACTIVE} A Delivery Service that is "active" is one that is
	functionally in service, and fully capable of delivering content. This means
	that its configuration is deployed to Cache Servers and it is available for
	routing traffic.
	\item \code{PRIMED} A Delivery Service that is "primed" has had its
	configuration distributed to the various servers required to serve its
	content. However, the content itself is still inaccessible\footnote{The
	content is not available through normal routing. This does not, though,
	guarantee that Cache Servers do not already have the content stored and/or
	are incapable of serving it if routing is bypassed.}.
	\item \code{INACTIVE} A Delivery Service that is "inactive" is not available
	for routing and has not had its configuration distributed to its assigned
	Cache Servers.
\end{itemize}

\paragraph{Supported Protocols}
This is a set of strings that name protocols served by the Delivery Service.
Note that this is the method used to retrieve content from the caching system,
not the method used for routing. The only protocols officially supported by ATC
are "HTTP" and "HTTPS".\\
This set is case-insensitive, such that if a Delivery Service is created with a
Supported Protocols set containing "HTTP" the resulting set is equivalent to
what would result from creating it with a Supported Protocols set containing
"http". Representations produced by the Traffic Ops API MUST always use
only uppercase characters.

\paragraph{Tags}
The Tags associated with a Delivery Service are represented by a set of strings
that are Tag Names.

\paragraph{Tenant}
The Tenant to which a Delivery Service belongs is represented by a string that
is that Tenant's unique Name.

\paragraph{Topology}
The Topology used by a Delivery Service is represented by a string that is the
Topology's unique Name.

\paragraph{Vanity Hostnames}
"Vanity Hostnames" is a set of strings that are Fully Qualified Domain Names
(FQDN) which may be used as alternates to the standard
\code{\emph{Routing Name}.\emph{Name}.\emph{CDN Domain}} FQDN when requesting
content from the Delivery Service.\\
No two Delivery Services may be allowed to share any single Vanity Hostname.\\
Note that Traffic Ops - and in fact Traffic Control in general - cannot and
does not guarantee that these vanity names will work, only that Traffic Router
will respond to them as equivalents to a normal Delivery Service FQDN when
content is requested through it. In general, because Vanity Hostnames are
typically outside of the CDN's Domain (which is the only domain for which
Traffic Router must be authoritative), this requires the DNS servers that are
authoritative for each Vanity Hostname's Domain to contain records that will
point to Traffic Router for resolution of these names.

\subsubsection{DNS-Routed Properties}

\paragraph{Bypass TTL}
An unsigned integer that defines the Time-To-Live (TTL) of DNS responses from
the Traffic Router for this Delivery Service's Bypass Destination, in seconds.

\paragraph{DNS TTL}
An unsigned integer that defines the Time-To-Live (TTL) of DNS responses from
the Traffic Router for this Delivery Service's routing, in seconds.

\paragraph{Max Records}

\subsubsection{HTTP-Routed Properties}

\paragraph{Additional Response Headers}

\paragraph{Consistent Hashing Regular Expression}

\paragraph{Logged Request Headers}

\paragraph{Significant Query Parameters}

\subsubsection{Steering-Routed Properties}

\paragraph{Targets}
