% Licensed to the Apache Software Foundation (ASF) under one
% or more contributor license agreements.  See the NOTICE file
% distributed with this work for additional information
% regarding copyright ownership.  The ASF licenses this file
% to you under the Apache License, Version 2.0 (the
% "License"); you may not use this file except in compliance
% with the License.  You may obtain a copy of the License at
%
%   http://www.apache.org/licenses/LICENSE-2.0
%
% Unless required by applicable law or agreed to in writing,
% software distributed under the License is distributed on an
% "AS IS" BASIS, WITHOUT WARRANTIES OR CONDITIONS OF ANY
% KIND, either express or implied.  See the License for the
% specific language governing permissions and limitations
% under the License.

\section{Authentication and Authorization\label{sec:auth}}
Authentication with the Traffic Ops API is done via the \code{/login} endpoint
(detailed in Section \ref{sec:endpoint:login}) to obtain credentials, and
thereafter by presenting those credentials in requests made to the server. The
default authorization scheme is using a cookie named "mojolicious" which
contains the authenticated user's username using SCRYPT. For more information,
consult the
\href{https://godoc.org/golang.org/x/crypto/scrypt}{documentation for the \code{golang.org/x/crypto/scrypt}
package used to implement it}. Traffic Ops API endpoints MUST not only recognize
this method of authentication, but also MUST check it before any other
authentication scheme.\\

\subsection{JWT Authentication}
Alternatively, authentication can be done using OAuth2 Bearer tokens in JSON
Web-Token (JWT) format\footnote{The intention of this section is to declare that
JWT-based authentication \emph{will} be supported by Traffic Ops API version 3
at some point. This does not necessarily mean it will be included in 3.0, but
the standards defined here should be enough for clients to start building out
support for it.} as defined by
\href{https://tools.ietf.org/html/rfc7519}{RFC7519}, signed with JSON
Web-Signatures (JWS). If a client requests a resource and is denied due to
failed authentication, the endpoint MUST return a \code{401 Unauthorized}
response code that contains an appropriate Alert as described in Section
\ref{sec:401}, and also MUST set the "WWW-Authenticate" header to \emph{exactly}
\code{Bearer realm="Apache Traffic Control"} by default. The realm may be
configurable.\\
Clients may present authentication in the "Authorization" HTTP header in the
format \code{Bearer \emph{JWT}} where "JWT" is the base-64-encoded JWT (and
accompanying JWS). When authentication using this method is unsuccessful, the
endpoint MUST set the "WWW-Authenticate" HTTP header to include the "error"
field set to "invalid\_token" and the "error\_description" field, which will
likely have the same or similar content as the accompanying Alert. An example of
such a failed response when the token presented by a client has expired is shown
in Listing \ref{code:expired-token}.

\begin{codelisting}
\captionof{listing}{Expired Token Response}
\label{code:expired-token}
\begin{minted}[tabsize=2]{http}
HTTP/1.1 401 Unauthorized
Content-Type: application/json
WWW-Authenticate: Bearer realm="Apache Traffic Control",
                         error="invalid_token",
                         error_description="token has expired"
Transfer-Encoding: chunked

{
	"alerts": [
		{
			"level": "error",
			"text": "Authentication failed; token has expired"
		}
	]
}
\end{minted}
\end{codelisting}

Valid JWTs consist of a JSON Object Signing and Encryption (JOSE) header
followed by a JWS Payload and finally a JWS signature, as described in
\href{https://tools.ietf.org/html/rfc7515}{RFC7515}. The JOSE header MUST define
its "typ" field as "JWT" and its "alg" field as "HS256". This means that the JWS
signature was generated using HMAC SHA-256 with the JWS Payload (un-encoded) as
input. The JOSE header's "jwk" MUST be set to the value of the public key used
to sign the SHA-256 hash in JSON Web Key (JWK) format as described in
\href{https://tools.ietf.org/html/rfc7517}{RFC7517}.\\
The claims of the JWS Payload - the actual JWT - MUST include the following
members:

\begin{itemize}
	\item \code{iss} The issuer of the JWT. This MUST have the value of the
	network location of the Traffic Ops server that generated the JWT - e.g.
	"trafficops.infra.ciab.test".
	\item \code{sub} The "subject" of the JWT - in this case, the Traffic Ops
	user. This MUST be a JSON-encoded representation of the authenticated user's
	identifying Username and permissions Role. An example is shown in Listing
	\ref{code:user-sub}.
	\item \code{exp} The timestamp at which this JWT will expire. Unlike other
	dates and times dealt with through the Traffic Ops API, this MUST be in the
	format of a Unix epoch timestamp in \emph{seconds}.
	\item \code{iat} The timestamp at which the JWT was issued. Unlike other
	dates and times dealt with through the Traffic Ops API, this MUST be in the
	format of a Unix epoch timestamp in \emph{seconds}.
\end{itemize}

\begin{codelisting}
\captionof{listing}{User Subject Example}
\label{code:user-sub}
\begin{minted}[tabsize=2]{json}
{
	"username": "testUser",
	"role": "admin"
}
\end{minted}
\end{codelisting}

\subsection{User Permissions}
User permissions are defined by the Permissions assigned to their Role as
described in Section \ref{sec:roles-and-perms}. Endpoint definitions will define
the Permissions required to interact with them via each HTTP method. If a user
does not have the required Permission to interact with a resource in the way
they requested, the server MUST respond with a \code{403 Forbidden} response
code containing an Alert that describes what Permission(s) the user lacks,
according to Section \ref{sec:403}. In general, these Permissions are defined on
an endpoint-by-endpoint basis, but some of them are shared or are applied
conditionally. For example, when creating a Cache Server, the
"cache-servers-write" Permission is required. However, if the created Cache
Server has one or more Delivery Service assignments then the
"delivery-services-write" Permission is also required (and the assignments are
subject to the rules of Tenancy as described in Section \ref{sec:tenancy}
because Delivery Services are Tenantable objects as discussed in Section
\ref{sec:tenantable}).
