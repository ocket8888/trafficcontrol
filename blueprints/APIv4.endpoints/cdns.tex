% Licensed to the Apache Software Foundation (ASF) under one
% or more contributor license agreements.  See the NOTICE file
% distributed with this work for additional information
% regarding copyright ownership.  The ASF licenses this file
% to you under the Apache License, Version 2.0 (the
% "License"); you may not use this file except in compliance
% with the License.  You may obtain a copy of the License at
%
%   http://www.apache.org/licenses/LICENSE-2.0
%
% Unless required by applicable law or agreed to in writing,
% software distributed under the License is distributed on an
% "AS IS" BASIS, WITHOUT WARRANTIES OR CONDITIONS OF ANY
% KIND, either express or implied.  See the License for the
% specific language governing permissions and limitations
% under the License.

\subsection{\code{/cdns}}
This endpoint deals with manipulations and representations of the group of CDN
objects configured in Traffic Ops.

\subsubsection{GET}
Retrieves CDN representations.
\begin{description}
	\item[Required Permissions] \code{cdns-read}
	\item[Response Type] Array
\end{description}

\paragraph{Request Structure}
This method of this endpoint implements the Age Filtering and Sorting and
Pagination query parameters as outlined in Sections \ref{sec:age-filtering} and
\ref{sec:pagination}, respectively. It further provides the query parameters in
Table \ref{tbl:cdns:get:qparams}.

\begin{table}[h]
	\centering
	\caption{GET \code{/cdns} Query Parameters\label{tbl:cdns:get:qparams}}
	\begin{tabularx}{\linewidth}{|l|X|}
		\hline
		\textbf{Parameter} & \textbf{Description}\\
		\hline
		dnssecEnabled & Filters results to only contain CDNs with this DNSSEC
			Enabled setting - must represent a boolean (\code{true} or
			\code{false}). If this cannot parse to a boolean, it will cause the
			endpoint to ignore the filter and emit a warning about the bad
			value for this parameter.\\
		\hline
		domain & Filters results to only contain CDNs that have this Domain\\
		\hline
		name & Filters results to only contain CDNs that have this Name\\
		\hline
		omit & This can be either a single property name (e.g.
			\code{omit=cacheServers}) or a comma-delimited array of property
			names (e.g. \code{omit=trafficMonitors,trafficRouters}). The named
			propert(y/ies) will be stripped from the output objects; that is,
			the returned objects will be missing these properties from their
			representations. The properties that can be omitted in this way
			are:
			\begin{itemize}
				\item Cache Servers
				\item Delivery Services
				\item Infrastructure Servers
				\item Origins
				\item Traffic Monitors
				\item Traffic Routers
				\item Traffic Stats Servers
			\end{itemize}
			If this parameter is provided and cannot be parsed as either a
			single property name or a comma-delimited list thereof, or if an
			attempt is made to omit a property other than those allowable, a
			400 Bad Request error response will be returned.\\
		\hline
	\end{tabularx}
\end{table}

\begin{codelisting}
\captionof{listing}{Request Example}
\begin{minted}{http}
GET /api/4.0/cdns?name=CDN-in-a-Box HTTP/1.1
Host: trafficops.infra.ciab.test
Accept: application/json, */*;q=0.9
Cookie: mojolicious=...

\end{minted}
\end{codelisting}

\paragraph{Response Structure}
The response is an array of representations of CDN objects, each representation
extended with the \code{lastUpdated} property containing the Date/Time at which
the CDN object was last modified.\\
This method of this endpoint also implements the \code{count} property of the
top-level \code{summary} object as described in Section
\ref{sec:summary-object}.

\begin{codelisting}
\captionof{listing}{Response Example}
\begin{minted}[tabsize=2]{http}
HTTP/1.1 200 OK
Content-Type: application/json
Server: Traffic Ops/5.0
Date: Wed, 14 Nov 2018 20:46:57 GMT
Content-Length: 237

{ "response": [
	{
		"cacheServers": [ 1, 2 ],
		"deliveryServices": [ "Demo1" ],
		"dnssecEnabled": false,
		"domain": "mycdn.ciab.test",
		"infrastructureServers": [ 1, 2, 3 ],
		"lastUpdated": "2018-11-14T18:21:14Z",
		"name": "CDN-in-a-Box",
		"origins": [ 1 ],
		"trafficMonitors": [ 1 ],
		"trafficRouters": [ 1 ],
		"trafficStatsServers": [ 1 ]
	}
],
"summary": {
	"count": 2
}}
\end{minted}
\end{codelisting}

\subsubsection{POST}
Creates a new CDN.
\begin{description}
	\item[Required Permissions] \code{cdns-write}
	\item[Response Type] Array
\end{description}

\subsection{\code{/cdns/\{\{CDN Name\}\}}}
This endpoint deals with manipulation and representation of a single CDN object
identified by its Name.\\
For all methods of this endpoint, if the part of the path composed of
\code{CDN Name} is not the name of a CDN that exists in Traffic Ops, a 404 Not
Found error response will be returned in accordance with Section \ref{sec:404}.

\subsection{\code{/cdns/\{\{CDN Name\}\}/snapshot}}
