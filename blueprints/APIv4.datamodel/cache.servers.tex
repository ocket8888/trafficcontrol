% Licensed to the Apache Software Foundation (ASF) under one
% or more contributor license agreements.  See the NOTICE file
% distributed with this work for additional information
% regarding copyright ownership.  The ASF licenses this file
% to you under the Apache License, Version 2.0 (the
% "License"); you may not use this file except in compliance
% with the License.  You may obtain a copy of the License at
%
%   http://www.apache.org/licenses/LICENSE-2.0
%
% Unless required by applicable law or agreed to in writing,
% software distributed under the License is distributed on an
% "AS IS" BASIS, WITHOUT WARRANTIES OR CONDITIONS OF ANY
% KIND, either express or implied.  See the License for the
% specific language governing permissions and limitations
% under the License.

\subsection{Cache Servers}
Cache servers are the Mid-and-Edge-tier HTTP caching proxies ultimately
responsible for caching and serving content for a Delivery Service.

\subsubsection{Cache Group}
The Cache Group to which the Cache Server belongs is represented as a string
that uniquely names it.

\subsubsection{Capabilities}
A Cache Server's "Capabilities" express a Cache Server's ability to serve
certain kinds of traffic. They are represented on Cache Server objects as a set
of Capability Names.

\subsubsection{CDN}
The CDN to which a Cache Server belongs is represented by its Name, which
uniquely identifies it.

\subsubsection{Domain\label{sec:server:domain}}
This is the "domain" part of the Cache Server's Fully Qualified Domain Name
(FQDN) as a string. For example, a Cache Server with an FQDN of
\code{github.com} has a Domain of \code{com} and a Cache Server with an FQDN of
\code{trafficcontrol.apache.org} has a Domain of \code{apache.org}.\\
In other words, this is every part of the Cache Server's Fully Qualified Domain
Name that isn't part of its Host Name as defined in Section
\ref{sec:server:hostname}.

\subsubsection{Host Name\label{sec:server:hostname}}
This is the "host" part of the Cache Server's Fully Qualified Domain Name (FQDN)
as a string. For example, a Cache Server with an FQDN of \code{github.com} has a
Host Name of \code{github}, and a Cache Server with an FQDN of
\code{trafficcontrol.apache.org} has a Host Name of \code{trafficcontrol}.\\
In other words, a Cache Server's Host Name is the only part of its FQDN that
is not part of its Domain as defined in Section \ref{sec:server:domain}.\\
This field does NOT need to be unique, though operators are encouraged to make
this - or at least the concatenation \code{\emph{Host Name}.\emph{Domain}}
- unique for ease of operation.

\subsubsection{HTTP Port}
An unsigned number that designates the port on which this Cache Server listens
for incoming HTTP requests. Note that this may be zero, which will cause the
Cache Server to be incapable of serving HTTP traffic.

\subsubsection{HTTPS Port}
An unsigned number that designates the port on which this Cache Server listens
for incoming HTTPS requests. Note that this may be zero, which will cause the
Cache Server to be incapable of serving HTTPS traffic.

\subsubsection{ID}
A Cache Server's "ID" is an unsigned integer that uniquely identifies it among
all Cache Servers. It serves no purpose beyond unique identification of the
Cache Server.

\subsubsection{Interfaces}
Interfaces is a set of objects that represent the network interfaces used by a
Cache Server. This set MUST NOT ever be empty. An Interface's properties are
herein listed.

\paragraph{IP Addresses}
A network interface's IP Addresses are represented as a set of objects containing
the various properties of an IP Address. A network interface MUST always have at
least one IP Address object in this set. Their properties are herein listed.\\
Note that there is no structural distinction made between different IP
versions.

\indent{}\subparagraph{Address}
Address is the actual IP Address being described. This may be represented
internally - by either clients or the server - as an actual network address
type, but MUST be represented in API payloads as a string type - even if the
representation encoding supports native network address types.

\indent{}\subparagraph{Gateway}
Gateway is the IP Address of the network gateway used to access this IP
Address. This may be represented internally - by either clients or the server -
as an actual network address type, but MUST be represented in API payloads as a
string type - even if the representation encoding supports native network
address types.\\
This MUST NOT ever be allowed to be represented as an empty string. However, if
it is a "null"-type, then the implication is that no network gateway is used to
access the IP Address.

\indent{}\subparagraph{Service Address}
Service Address is a boolean that describes whether or not an IP Address is one
on which a Cache Server provides its service. Only one such IP Address is
allowed to exist for a given Cache Server per IP version. That is, a Cache
Server may have one IPv4 Service Address and one IPv6 Service Address, but is
not permitted to have two or more IPv4 Service Addresses.

\paragraph{Max Bandwidth}
This unsigned integer describes the maximum bandwidth - in kb/s - that is
allowed for this interface to be considered "healthy" by Traffic Monitors.\\
This has no effect if this Interface's Monitor value is not True.\\
The value \code{0} has the meaning "no limit".

\paragraph{Monitor}
A boolean that describes whether or not this interface should be monitored by
Traffic Monitor. Note that multiple Interfaces that have True Monitor values
is incompatible with a Health Polling Format value of "astats" or
"astats-dsnames".

\paragraph{MTU}
MTU is an unsigned integer that gives the Interface's Maximum Transmission
Unit. It is recommended that UIs built on the API provide the values 9000
and 1500 for selection only, as other values are very rarely correct. However,
the API itself places no such restrictions on the value, which can be any
unsigned integer or a "null"-type.\\
If the value is a "null"-type, then it is assumed that the interface's MTU is
not known or is not relevant for the Cache Server's operation (common on
non-Service-Address-containing Interfaces).

\paragraph{Name}
The string containing a network interface's Name MUST NOT be allowed to be
empty. It should name the actual network interface device on the Cache Server.
For example, two common Names are "eth0" and "bond0".

\subsubsection{Notes}
This section is an arbitrary string containing miscellaneous, human-friendly
information about the Cache Server. Other ATC components SHOULD NOT parse this
for specific information fields, or expect it to be in a particular format.

\subsubsection{Physical Location}
The Physical Location at which a Cache Server resides is represented by a string
containing its Name.

\subsubsection{Profile}
The Profile used by a Cache Server is represented by a string containing its
Name.

\subsubsection{Revalidation Pending}
This boolean represents whether or not the Cache Server has content revalidation
requests yet to satisfy. When a new Content Revalidation Request is created on
one or more Delivery Services to which this Cache Server is assigned, this will
be set to "true" and updated to "false" when the operation has been completed.
Being "false" does not mean that the content invalidation request(s) performed
by the Cache Server have expired and are no longer in effect.

\subsubsection{Status}
The Cache Server's "Status" is a string constant, which MUST always be one of

\begin{itemize}
	\item \code{ADMIN\_DOWN} - The Cache Server is considered unhealthy and its
	thresholds and connectivity state are not monitored. Its existence is not
	disclosed to Traffic Router(s).
	\item \code{OFFLINE} - The Cache Server is considered unhealthy regardless
	of any thresholds or connectivity state.
	\item \code{ONLINE} - The Cache Server will always be considered healthy
	regardless of any thresholds or connectivity state.
	\item \code{REPORTED} - The Cache Server's health is presented to the
	Traffic Router(s) as it is reported by its various thresholds, as determined
	by the Traffic Monitor(s).
\end{itemize}

\subsubsection{Tags}
A Cache Server's "Tags" are represented as a set of Tag Names.

\subsubsection{Type}
A Cache Server's "Type" is a string constant, which MUST always be one of

\begin{itemize}
	\item \code{EDGE} - This is an Edge-tier Cache Server which acts as a
	reverse proxy.
	\item \code{MID} - This is a Mid-tier Cache Server which acts as a forward
	proxy.
\end{itemize}

\subsubsection{Updates Pending}
This boolean represents whether or not the Cache Server has configuration
updates pending. When such updates are applied, this will be set to "false" by
the Cache Server's ORT/ATSTCCFG instance.
