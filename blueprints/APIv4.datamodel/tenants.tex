% Licensed to the Apache Software Foundation (ASF) under one
% or more contributor license agreements.  See the NOTICE file
% distributed with this work for additional information
% regarding copyright ownership.  The ASF licenses this file
% to you under the Apache License, Version 2.0 (the
% "License"); you may not use this file except in compliance
% with the License.  You may obtain a copy of the License at
%
%   http://www.apache.org/licenses/LICENSE-2.0
%
% Unless required by applicable law or agreed to in writing,
% software distributed under the License is distributed on an
% "AS IS" BASIS, WITHOUT WARRANTIES OR CONDITIONS OF ANY
% KIND, either express or implied.  See the License for the
% specific language governing permissions and limitations
% under the License.

\subsection{Tenants}
A Tenant is ultimately a set of users with permissions to access a set of
resources. This concept differs from a Role or Permission in that it is not
a specific action being controlled, but rather any set of actions performed on
specific objects - \emph{not} specific types of objects.\\
Object types that are scoped to Tenants are called "tenantable" and the types
of objects that are "tenantable" are cataloged in Section
\ref{sec:tenantable}. Note that a Tenant falls within its own Tenancy, as well
as the Tenancy of all of its ancestors.\\
Traffic Ops comes with one Tenant that is defined to be the "highest-level" or,
more commonly, the "root" Tenant. It is the special Tenant which is the only
one allowed to not have a Parent, and all tenantable objects fall within its
Tenancy. This Tenant is always Active, and has the name \code{root}.

\subsubsection{Active}
Active is a boolean concept that describes whether or not a Tenant's group of
users are allowed to actively manipulate resources. Resources within an
inactive Tenant continue to function in the ways in which they are configured,
but members of the inactive Tenant are incapable of manipulating that
configuration until they are once again "active".\\
When a Tenant is inactive, \emph{all} users within that Tenancy are inactive.
For example, if a Tenant A exists and user 1 is within that Tenant then if
Tenant A becomes inactive user 1 may no longer manipulate the objects that
fall within the A Tenancy (including A itself). If Tenant A has a child Tenant
B and some user 2 is within that Tenancy, then when A becomes inactive all of
the users in B are also affected in that they cannot manipulate any resources
that fall within the A Tenancy (including all resources that fall within the
B Tenancy). Thus the Active property of a Tenant is only considered when
determining if a Tenant is active or inactive if all of that Tenant's
ancestor Tenants are themselves active; otherwise the inactive state of an
ancestor overrides the active/inactive state of the descendant.\\
The "root" Tenant is never allowed to become inactive.

\subsubsection{Name}
The Name of a Tenant is a string that uniquely identifies it. A Tenant Name
MUST contain only alphanumeric characters, underscores, and hyphens.

\subsubsection{Parent}
A Tenant's Parent is a string which is the Name of the Tenant from which this
Tenant directly descends.\\
The Parent of a Tenant is not allowed to be either the Tenant itself, or any
descendant of that Tenant - a "descendant" being any Tenant which either has
this Tenant as its direct Parent, or has as its direct Parent a descendant of
the Tenant.\\
A tenantable object is said to fall within the Tenancy of a Tenant if it is
either assigned to the Tenant or some descendant thereof. For example, if a
Delivery Service X is within the B Tenant, and the B Tenant's Parent is the A
Tenant, and the A Tenant's Parent is the "root" Tenant, then X falls within the
Tenancy of "root", A, and B. However, if the B Tenancy is the Parent of the C
Tenant, then X is \emph{not} within the C Tenancy.\\
Only the "root" Tenant is permitted to have a "null"-typed Parent. In
particular, the "root" Tenant is defined to be the only Tenant which has a
"null"-typed Parent.
