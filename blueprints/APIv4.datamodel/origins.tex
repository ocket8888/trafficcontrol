% Licensed to the Apache Software Foundation (ASF) under one
% or more contributor license agreements.  See the NOTICE file
% distributed with this work for additional information
% regarding copyright ownership.  The ASF licenses this file
% to you under the Apache License, Version 2.0 (the
% "License"); you may not use this file except in compliance
% with the License.  You may obtain a copy of the License at
%
%   http://www.apache.org/licenses/LICENSE-2.0
%
% Unless required by applicable law or agreed to in writing,
% software distributed under the License is distributed on an
% "AS IS" BASIS, WITHOUT WARRANTIES OR CONDITIONS OF ANY
% KIND, either express or implied.  See the License for the
% specific language governing permissions and limitations
% under the License.

\subsection{Origins}
Origins are sources of content for the CDN to serve. They are typically external
to the CDN's infrastructure and in many cases may in fact not be administered by
the same organization as the CDN.

\begin{codelisting}
\captionof{listing}{Origin Object as a TypeScript Interface}
\begin{minted}[tabsize=2]{typescript}
interface Origin {
	ipv4Address: string | null;
	ipv6Address: string | null;
	latitude: number | null;
	longitude: number | null;
	notes: string;
	tags: Set<string>;
	tenant: string;
	url: URL;
}
\end{minted}
\end{codelisting}

\subsubsection{IPv4 Address}
This string is a valid IP (v4) address at which the Origin may be found, or
"null"-typed to indicate there is no static IPv4 Address where the Origin may be
relied on to be.

\subsubsection{IPv6 Address}
This string is a valid IP (v6) address at which the Origin may be found, or
"null"-typed to indicate there is no static IPv6 Address where the Origin may be
relied on to be.

\subsubsection{Latitude}
A floating-point number on the interval $[-90, 90]$ that indicates the
geographic latitude at which the Origin resides (used for routing purposes). If
this is "null"-typed, then it is assumed that the Origin's geophysical location
is unknown or irrelevant and its Longitude MUST also be "null"-typed.

\subsubsection{Longitude}
A floating-point number on the interval $[-180, 180]$ that indicates the
geographic longitude at which the Origin resides (used for routing purposes). If
this is "null"-typed, then it is assumed that the Origin's geophysical location
is unknown or irrelevant and its Latitude MUST also be "null"-typed.

\subsubsection{Notes}
This is a free text field used to record miscellaneous notes about the Origin.

\subsubsection{Tags}
The Tags associated with an Origin - or, with which it is "tagged" - are a set
of strings that are Tag Names.

\subsubsection{Tenant}
The Tenant to which an Origin belongs is represented by a string which is its
Name.

\subsubsection{URL}
An Origin's URL is a string that is a valid URL which is used by the CDN (and
ostensibly any client) to request content. As such, the only supported protocol
schemes are those for HTTP and HTTPS - though this is not enforced as a
restriction on the values of the URL.\\
The port part of the network location within the URL may be included, but will
be inferred from the scheme if it is not, e.g. 80 for HTTP protocol schemes.\\
An Origin's URL is also used to uniquely identify it. URLs are considered unique
if they are unique \emph{in totality} - but \emph{not} including inferred
quantities. That is, \code{https://github.com} is uniquely distinguishable from
\code{http://github.com}, but \emph{not} from \code{https://github.com:443}.\\
URLs MUST NOT have paths - other than optionally specifying the root path
(\code{/}) - as content is always assumed to be retrieved relative to the root
of the Origin service.
