% Licensed to the Apache Software Foundation (ASF) under one
% or more contributor license agreements.  See the NOTICE file
% distributed with this work for additional information
% regarding copyright ownership.  The ASF licenses this file
% to you under the Apache License, Version 2.0 (the
% "License"); you may not use this file except in compliance
% with the License.  You may obtain a copy of the License at
%
%   http://www.apache.org/licenses/LICENSE-2.0
%
% Unless required by applicable law or agreed to in writing,
% software distributed under the License is distributed on an
% "AS IS" BASIS, WITHOUT WARRANTIES OR CONDITIONS OF ANY
% KIND, either express or implied.  See the License for the
% specific language governing permissions and limitations
% under the License.

\subsection{Capabilities}
A Capability expresses the capacity of a Cache Server to serve a specific kind
of traffic. They are similar to Tags in that they are applied to group objects
into logical groupings and no special meaning may be inferred from their Name
or from patterns of their Names\footnote{Including by Traffic Ops and the
Traffic Ops API.}. However, a Capability differs from a Tag in that the
presence or absence of a Capability on an object \emph{does} have semantic
meaning.\\
In particular, a Delivery Service has Required Capabilities that define the
kinds of traffic a Cache Server must be able to handle in order to serve
content for that Delivery Service. This requirement is enforced by ensuring
that only Cache Servers that have \emph{all} of a Delivery Service's Required
Capabilities within their own Capabilities set are selected via parentage at
the Cache Server configuration generation step, as well as by Traffic Router
when selecting Edge-Tier Cache Servers to which to direct clients requesting
the Delivery Service's content.\\
Note that while Capabilities are enforced by routing and Cache Server parentage
it is not possible to enforce that a Cache Server actually \emph{has} the
Capabilities that are within its Capabilities set, and in fact a Capability's
Name has no semantic meaning that could be checked in any case. In this way it
is possible to use them to express arbitrary concepts, but operators must be
careful to only assign Capabilities to Cache Servers when it is known that they
actually posses those Capabilities.

\begin{codelisting}
\captionof{listing}{Capability Object as a Typescript Interface}
\begin{minted}[tabsize=2]{typescript}
interface Capability {
	Description: string;
	Name: string;
}
\end{minted}
\end{codelisting}

\subsubsection{Description}
This string contains arbitrary text which ideally describes the capacity
encapsulated and expressed by the Capability, as well as the reason for its
existence.

\subsubsection{Name}
A Capability's Name is a string that uniquely identifies it, and ideally
conveys the capacity it expresses, e.g. a Capability Named \code{400GB\_RAM} may
express that a Cache Server has/must have a minimum of 400GB of RAM device
block storage to be used for caching.\\
A Capability Name MUST consist only of alphanumeric characters, hyphens, and
underscores.


