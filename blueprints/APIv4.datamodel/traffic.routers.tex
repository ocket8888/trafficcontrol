% Licensed to the Apache Software Foundation (ASF) under one
% or more contributor license agreements.  See the NOTICE file
% distributed with this work for additional information
% regarding copyright ownership.  The ASF licenses this file
% to you under the Apache License, Version 2.0 (the
% "License"); you may not use this file except in compliance
% with the License.  You may obtain a copy of the License at
%
%   http://www.apache.org/licenses/LICENSE-2.0
%
% Unless required by applicable law or agreed to in writing,
% software distributed under the License is distributed on an
% "AS IS" BASIS, WITHOUT WARRANTIES OR CONDITIONS OF ANY
% KIND, either express or implied.  See the License for the
% specific language governing permissions and limitations
% under the License.

\subsection{Traffic Routers}
Traffic Routers are Traffic Router instances. They contain all of the server
information as well as configuration for the Traffic Router service.

\subsubsection{API Port}
This unsigned integer defines the port on which Traffic Router serves its HTTP
API. This MUST be less than $2^{16}$ ($65,536$).

\subsubsection{Cache Control Max Age}
This unsigned integer is the value - in seconds - used for the "max-age"
parameter of the Cache-Control HTTP header in HTTP responses to the client. If
it is "null"-typed then Traffic Router will use its default setting - which is
defined by Traffic Router and not by the Traffic Ops API.

\subsubsection{CDN}
The CDN to which a Traffic Router belongs is represented by its Name, which
uniquely identifies it. The Traffic Router MUST be an/the authoritative DNS
server for the Domain of the CDN to which it belongs.

\subsubsection{Coverage Zone Polling Interval}
This unsigned integer is the interval - in milliseconds - on which Traffic
Router will poll for an updated Coverage Zone File. A "null"-typed value causes
Traffic Router to use its default interval duration, which is not defined by the
Traffic Ops API.

\subsubsection{Deep Coverage Zone Polling Interval}
This unsigned integer is the interval - in milliseconds - on which Traffic
Router will poll for an updated Deep Coverage Zone File. A "null"-typed value
causes Traffic Router to use its default interval duration, which is not defined
by the Traffic Ops API.

\subsubsection{DNSSEC}
This property is an object, the properties of which are configuration options
for DNSSEC-related behavior.

\paragraph{Allow Expired Keys}
A boolean which, when true, will allow Traffic Router to use expired DNSSEC keys
to sign zones.

\paragraph{Effective Multiplier}
An integer that is used when creating an effective date for a new DNSSEC key
set. New keys are generated with an effective date that is the effective
multiplier multiplied by the Time to Live less than the old key's expiration
date. A "null"-typed value instructs Traffic Router to use its pre-configured
default value, which is not defined by the Traffic Ops API\footnote{At the time
of this writing, the default value is 2.}.

\paragraph{Fetch Interval}
An unsigned integer that defines the interval - in seconds - on which Traffic
Router will check the Traffic Ops API for new DNSSEC keys. A "null"-typed value
instructs Traffic Router to use its pre-configured default value - which is not
defined by the Traffic Ops API.

\paragraph{Fetch Retries}
An unsigned integer that defines the number of times Traffic Router will attempt
to load DNSSEC keys before giving up. A "null"-typed value instructs Traffic
Router to use its pre-configured default number of retries - which is not
defined by the Traffic Ops API\footnote{The default value at the time of this
writing is 5.}.

\paragraph{Fetch Timeout}
An unsigned integer that defines the timeout - in milliseconds - for requests to
the DNSSEC key management endpoint(s) of the Traffic Ops API. A "null"-typed
value indicates that Traffic Router should not time-out such requests.

\paragraph{Fetch Wait}
An unsigned integer that defines the number of milliseconds Traffic Router will
wait between attempts to load DNSSEC keys ("retries"). A "null"-typed value
instructs Traffic Router to use its pre-configured default value - which is not
defined by the Traffic Ops API.

\paragraph{Generation Multiplier}
An unsigned integer used to determine when new DNSSEC keys need to be generated.
Keys are re-generated if expiration is less than the Generation Multiplier
multiplied by the Time to Live. A "null"-typed value instructs Traffic Router to
use its pre-configured default multiplier - which is not defined by the Traffic
Ops API\footnote{The default value at the time of this writing is 10.}.

\paragraph{Zone Comparisons}
A boolean which, when true - and DNSSEC is enabled on the CDN to which this
Traffic Router belongs - allows Traffic Router to compare existing zones with
newly generated zones. If the newly generated zone is the same as the existing
zone, Traffic Router will simply re-use the existing signed zone instead of
signing the same, new zone.

\subsubsection{Domain}
This is the "domain" part of the Cache Server's Fully Qualified Domain Name
(FQDN) as a string. For example, a Cache Server with an FQDN of
\code{github.com} has a Domain of \code{com} and a Cache Server with an FQDN of
\code{trafficcontrol.apache.org} has a Domain of \code{apache.org}.

\subsubsection{Dynamic Zone Cache Priming}
This property is an object whose properties are configuration options for
"Dynamic Zone Cache Priming".

\paragraph{Prime}
A boolean which, if true, will allow Traffic Router to prime the dynamic zone
cache.

\paragraph{Priming Limit}
An unsigned integer used to limit the number of permutations to prime when
"Dynamic Zone Cache Priming". A "null"-typed value instructs Traffic Router to
use its pre-configured default, which is not defined by the Traffic Ops
API\footnote{The default value is 500 at the time of this writing.}.

\subsubsection{EDNS0 Client Subnet Enabled}
A boolean that sets whether or not the EDNS0 DNS extension mechanism described
in \href{https://tools.ietf.org/html/rfc2671}{RFC2671} should be made available
to clients.

\subsubsection{Forced Diversity}
A boolean which, when true, causes Traffic Router to diversify the list of Cache
Servers returned in responses to requests for STEERING-type Delivery Services'
content by including more unique Edge-Tier Cache Servers in the response to the
client's request.

\subsubsection{Geolocation Polling Interval}
An unsigned integer number of milliseconds that Traffic Router will use as an
interval on which to poll for an geographic IP mapping database. A "null"-typed
value will instruct Traffic Router to use its pre-configured default value -
which is not defined by the Traffic Ops API.

\subsubsection{Host Name}
This is the "host" part of the Cache Server's Fully Qualified Domain Name (FQDN)
as a string. For example, a Cache Server with an FQDN of \code{github.com} has a
Host Name of \code{github}, and a Cache Server with an FQDN of
\code{trafficcontrol.apache.org} has a Host Name of \code{trafficcontrol}.\\
This field does NOT need to be unique, though operators are encouraged to make
this - or at least the concatenation \code{\emph{Host Name}.{Domain}} - unique
for ease of operation.

\subsubsection{ID}
A Cache Server's "ID" is an unsigned integer that uniquely identifies it among
all Cache Servers. It serves no purpose beyond unique identification of the
Cache Server.

\subsubsection{Notes}
This section is an arbitrary string containing miscellaneous, human-friendly
information about the Cache Server. Other ATC components SHOULD NOT parse this
for specific information fields, or expect it to be in a particular format.

\subsubsection{Physical Location}
The Physical Location at which a Cache Server resides is represented by a string
containing its Name.

\subsubsection{SOA}
This property is an object that contains configuration information for the
"Start of Authority" (SOA) records served by the Traffic Router.

\paragraph{Admin}
An email address for the administrator of the DNS zones for which Traffic Router
is authoritative. This string may be "null"-typed to indicate such an address is
unavailable.

\paragraph{Expire}
An unsigned integer number of seconds which will be used in the "expire" field
the Traffic Router DNS Server will respond with on SOA records.

\paragraph{Minimum}
An unsigned integer number of seconds which will be used as the value for the
"minimum" field the Traffic Router DNS Server will respond with on SOA records.

\paragraph{Refresh}
An unsigned integer number of seconds which will be used as the value for the
"refresh" field the Traffic Router DNS Server will respond with on SOA records.

\paragraph{Retry}
An unsigned integer number of seconds which will be used as the value for the
"retry" field the Traffic Router DNS Server will respond with on SOA records.

\subsubsection{Steering Polling Interval}
An unsigned integer number of seconds on which Traffic Router will check for new
steering mappings. A "null"-typed value instructs Traffic Router to use its
pre-configured default interval - which is not defined by the Traffic Ops API.

\subsubsection{TTLs}
This property is a map of record types (in lower-case) to unsigned integers that
are the Time-to-Live values - in seconds - that Traffic Router will use for the named
DNS record type.\\
The meaningful keys in this map are (but are not limited to for the forseeable
future):

\begin{itemize}
	\item a
	\item aaaa
	\item dnskey
	\item ds
	\item ns
	\item soa
\end{itemize}

\subsubsection{Zones}
The Zones of a Traffic Router is a property that is itself an object having
properties which express configuration options for DNS zone management.

\paragraph{Cache Maintenance Interval}
This unsigned integer defines an interval - in seconds - on which Traffic Router
will check for zones that need to be re-signed or if dynamic zones need to be
expired from its cache.

\paragraph{Dynamic Initial Capacity}
This unsigned integer defines an initial size to be used by the Traffic Router's
zone manager's cache. If this is "null"-typed, Traffic Router is instructed to
use its pre-configured default value, which is not defined by the Traffic Ops
API\footnote{At the time of this writing the default size is $10,000$.}.

\paragraph{Dynamic Response Expiration}
This unsigned integer defines how long - in seconds - Traffic Router will allow
a dynamic zone to remain in cache after its last access. If it is "null"-typed,
Traffic Router is instructed to use its pre-configured default duration, which
is not defined by the Traffic Ops API\footnote{At the time of this writing the
default duration is $3,600$.}.

\paragrahp{Initialization Timeout}
This unsigned integer sets a cap - in minutes - on the amount of time Traffic
Router will spend initializing its dynamic zone manager. Larger numbers impact
startup time, but allow for more zones to be pre-computed. If it is "null"-typed
Traffic Router is instructed to use its pre-configured default timeout, which is
not defined by the Traffic Ops API\footnote{At the time of this writing the
default timeout is $10$.}.

\paragraph{Thread Count}
This unsigned integer sets a number of concurrent threads for the Traffic
Router's zone manager to operate on. The value 0 is not allowed. If this is
"null"-typed, the number of threads is calculated based on the number of
allowed/available CPU cores as defined by Thread Multiplier.

\paragraph{Thread Multiplier}
This floating point number is multiplied by the number of available CPU cores on
the Traffic Router to calculate the number of threads the zone manager will use,
rounded down to the nearest integer. For example, a value of $1.0$ will instruct
Traffic Router to use a thread for every available CPU core.\\
This has no effect if Thread Count is not "null"-typed. The value MUST be
greater than zero. If this is "null"-typed (and Thread Count is not), Traffic
Router will use its pre-configured default multiplier, which is not defined by
the Traffic Ops API\footnote{At the time of this writing, the default multiplier
is $0.75$.}.
