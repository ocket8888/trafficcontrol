% Licensed to the Apache Software Foundation (ASF) under one
% or more contributor license agreements.  See the NOTICE file
% distributed with this work for additional information
% regarding copyright ownership.  The ASF licenses this file
% to you under the Apache License, Version 2.0 (the
% "License"); you may not use this file except in compliance
% with the License.  You may obtain a copy of the License at
%
%   http://www.apache.org/licenses/LICENSE-2.0
%
% Unless required by applicable law or agreed to in writing,
% software distributed under the License is distributed on an
% "AS IS" BASIS, WITHOUT WARRANTIES OR CONDITIONS OF ANY
% KIND, either express or implied.  See the License for the
% specific language governing permissions and limitations
% under the License.

\subsection{Traffic Portals}
A Traffic Portal represents a running instance of a Traffic Portal UI.

\subsubsection{IPv4 Address}
This string contains a valid IPv4 Address at which the UI is served. If it is
"null"-typed, then it is assumed that there is no static IPv4 Address at which
the Traffic Portal instance may be reached - or it does not communicate over
IPv4 at all.

\subsubsection{IPv6 Address}
This string contains a valid IPv6 Address at which the UI is served. If it is
"null"-typed, then it is assumed that there is no static IPv6 Address at which
the Traffic Portal instance may be reached - or it does not communicate over
IPv6 at all.

\subsubsection{Notes}
This string contains arbitrary text for containing miscellaneous information.

\subsubsection{Tags}
The Tags associated with a Traffic Portal are represented by a set of strings
that are Tag Names.

\subsubsection{URL}
This string is a full URL at which the UI is served, including scheme (e.g.
\code{http://}) and optionally port (e.g. \code{80}), e.g.
\code{https://trafficportal.infra.ciab.test:443}.\\
If the port is omitted, it will be guessed based on the protocol indicated by
the schema, e.g. \code{80} for \code{http://}, which indicates the HTTP
protocol. The only valid schemes are \code{https://} and \code{http://}.\\
This property uniquely identifies a Traffic Portal, and must therefore
obviously be unique\footnote{This allows multiple Traffic Portals - or other
services - to be running at the same static IP address(es) as long as they
are not literally the same network location (port, host, or protocol could
differ).}.
