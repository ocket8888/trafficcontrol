% Licensed to the Apache Software Foundation (ASF) under one
% or more contributor license agreements.  See the NOTICE file
% distributed with this work for additional information
% regarding copyright ownership.  The ASF licenses this file
% to you under the Apache License, Version 2.0 (the
% "License"); you may not use this file except in compliance
% with the License.  You may obtain a copy of the License at
%
%   http://www.apache.org/licenses/LICENSE-2.0
%
% Unless required by applicable law or agreed to in writing,
% software distributed under the License is distributed on an
% "AS IS" BASIS, WITHOUT WARRANTIES OR CONDITIONS OF ANY
% KIND, either express or implied.  See the License for the
% specific language governing permissions and limitations
% under the License.

\subsection{Traffic Monitors}
A Traffic Monitor is a service that communicates with Traffic Ops over HTTP and
is responsible for collecting health and statistics information from Cache
Servers to determine if they are fit to service content.

\subsubsection{CDN}
The CDN to which a Traffic Monitor belongs is represented by its Name, which
uniquely identifies it. The Monitor will monitor all Cache Servers that are in
its CDN.

\subsubsection{Config Polling Interval}
Config Polling Interval is an unsigned integer that defines an interval - in
milliseconds - on which the Monitor should poll Traffic Ops for its
configuration changes. This MUST NOT be \code{0}.

\subsubsection{Domain}
This is the "domain" part of the Traffic Monitor's Fully Qualified Domain Name
(FQDN) as a string. For example, a Traffic Monitor with an FQDN of
\code{github.com} has a Domain of \code{com} and a Traffic Monitor with an FQDN
of \code{trafficcontrol.apache.org} has a Domain of \code{apache.org}.

\subsubsection{Event Count}
This unsigned integer defines the number of "events" of which the Monitor will
keep track internally.\\
The value \code{0} means "no limit".

\subsubsection{Health Polling Interval}
This unsigned integer defines an interval - in milliseconds - on which to poll
Cache Server health. This MUST NOT be \code{0}.

\subsubsection{Heartbeat Polling Interval}
This unsigned integer defines an interval - in milliseconds - on which to poll
Cache Server "heartbeats". This poll is not meant to determine health, but
rather whether or not the Cache Server(s) may be reached over the network at
all. This MUST NOT be \code{0}.

\subsubsection{Host Name}
This is the "host" part of the Traffic Monitor's Fully Qualified Domain Name
(FQDN) as a string. For example, a Traffic Monitor with an FQDN of
\code{github.com} has a Host Name of \code{github}, and a Traffic Monitor with an
FQDN of \code{trafficcontrol.apache.org} has a Host Name of
\code{trafficcontrol}.\\
This field does NOT need to be unique, though operators are encouraged to make
this - or at least the concatenation \code{\emph{Host Name}.{Domain}} - unique
for ease of operation.

\subsubsection{ID}
A Traffic Monitor's "ID" is an unsigned integer that uniquely identifies it
among all Traffic Monitors. It serves no purpose beyond unique identification of
the Traffic Monitor.

\subsubsection{Notes}
This section is an arbitrary string containing miscellaneous, human-friendly
information about the Traffic Monitor. Other ATC components SHOULD NOT parse
this for specific information fields, or expect it to be in a particular
format.

\subsubsection{Physical Location}
The Physical Location at which a Traffic Monitor resides is represented by a
string containing its Name.

\subsubsection{Tags}
A Traffic Monitor's "Tags" are represented as a set of Tag Names.

\subsubsection{Thread Count}
Thread Count is an unsigned integer that determines the number of threads used
to concurrently poll Cache Servers. It is recommended that UIs built on the
Traffic Ops API set this to \code{0} by default and caution users against
changing it. The value \code{0} will cause Traffic Monitor to choose a value
for itself\footnote{A recommendation for Traffic Monitor engineers: a
reasonable default value would be the number of available CPU cores.}.

\subsubsection{Time Pad}
This is an unsigned integer that defines a set number of milliseconds to add
to request timers to help spread requests out for Traffic Control systems that
use a large number of Traffic Monitors.
