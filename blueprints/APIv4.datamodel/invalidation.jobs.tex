% Licensed to the Apache Software Foundation (ASF) under one
% or more contributor license agreements.  See the NOTICE file
% distributed with this work for additional information
% regarding copyright ownership.  The ASF licenses this file
% to you under the Apache License, Version 2.0 (the
% "License"); you may not use this file except in compliance
% with the License.  You may obtain a copy of the License at
%
%   http://www.apache.org/licenses/LICENSE-2.0
%
% Unless required by applicable law or agreed to in writing,
% software distributed under the License is distributed on an
% "AS IS" BASIS, WITHOUT WARRANTIES OR CONDITIONS OF ANY
% KIND, either express or implied.  See the License for the
% specific language governing permissions and limitations
% under the License.

\subsection{Invalidation Jobs}
Invalidation Jobs - also called "content invalidation jobs", "revalidations",
"content revalidation jobs", "revalidation jobs", or even just "jobs" - are, at
their most basic, periods of time during which specific content should not be
served from cache but instead considered cache-invalid and fetched from
upstream. These objects are limited in scope to within a particular Delivery
Service, but are not strictly "Tenantable", as such.

\begin{codelisting}
\captionof{listing}{Invalidation Job Object as a Typescript Interface}
\label{code:datamodel:invalidation-job}
\begin{minted}[tabsize=2]{typescript}
interface InvalidationJob {
	assetPattern: RegExp;
	createdBy: string;
	deliveryService: string;
	id: bigint;
	startTime: Date;
	tenant: string;
	timeToLive: bigint;
}
\end{minted}
\end{codelisting}

\subsubsection{Asset Pattern}
This is a regular expression that defines the content paths which will be
"invalidated". It MUST begin with \code{/} since all valid request paths begin
with \code{/}. Example: \code{/.*\textbackslash{}.jpg}.

\subsubsection{Created By}
The User that created the Invalidation Job may be referenced by this string
which is their Username\footnote{This Username will be visible to Users with the
appropriate Permission(s) to see Invalidation Jobs, regardless of Tenancy;
Invalidation Jobs are not "Tenantantable".}.

\subsubsection{Delivery Service}
The Delivery Service within which the Invalidation Job operates is given by this
string which is its Name.

\subsubsection{ID}
ID is an unsigned integer that uniquely identifies an Invalidation Job. There is
no semantic meaning to this property other than unique identification.

\subsubsection{Start Time}
This is the date and time at which the Invalidation Job is set to begin. Upon
creation, this MUST be at most one hour before the current time\footnote{This is
done to account for potential differences between what time the server thinks it
is and what time the client thinks it is - it is neither the intention nor,
indeed, possible that a content invalidation job begin before it is submitted.}.

\subsubsection{Tenant}
The Tenant of an Invalidation Job is a string which is the Name of the Tenant to which it belongs. The Delivery Service to which the Invalidation Job applies
MUST fall within the Tenancy of this Tenant\footnote{This is enforced upon
creation by the API.}.

\subsubsection{Time to Live}
This unsigned integer number of hours for which the Invalidation Job should
remain "active" - that is, this defines the window within which the specified
content will not be served from cache.\\
This may not be zero ($0$). This has an upper bound at creation time defined by
the Max Revalidation Days property of the Global Configuration object.
