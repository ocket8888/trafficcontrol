% Licensed to the Apache Software Foundation (ASF) under one
% or more contributor license agreements.  See the NOTICE file
% distributed with this work for additional information
% regarding copyright ownership.  The ASF licenses this file
% to you under the Apache License, Version 2.0 (the
% "License"); you may not use this file except in compliance
% with the License.  You may obtain a copy of the License at
%
%   http://www.apache.org/licenses/LICENSE-2.0
%
% Unless required by applicable law or agreed to in writing,
% software distributed under the License is distributed on an
% "AS IS" BASIS, WITHOUT WARRANTIES OR CONDITIONS OF ANY
% KIND, either express or implied.  See the License for the
% specific language governing permissions and limitations
% under the License.

\subsection{Delivery Services}
Delivery Services are, at their most basic, an association between a source of
content and a set of Cache Servers and configuration options used to distribute
that content.

\begin{codelisting}
\captionof{listing}{Delivery Service as a Typescript Type}
\begin{minted}[tabsize=2]{typescript}
type DeliveryService = DNSDeliveryService |
	DNSMSODeliveryService |
	HTTPDeliveryService |
	HTTPMSODeliveryService |
	SteeringDeliveryService |
	SteeringMSODeliveryService |
	StaticDeliveryService;
\end{minted}
\end{codelisting}

\subsubsection{Common Properties}
Herein described are the properties common to all Delivery Service objects. The
Routing Type of a Delivery Service encapsulates the methods by which clients
may request content routing, and depending on its value the Delivery Service
takes on a set of additional properties. Put simply, these are all different
types of objects that are closely related.\\
This section details all of the properties that are common to \emph{all} types
of Delivery Services.

\begin{codelisting}
\captionof{listing}{Delivery Service Object as a Typescript Interface}
\begin{minted}[tabsize=2]{typescript}
interface BaseDeliveryService {
	anonymousBlocking: boolean;
	bypassDestination: string | null;
	cdn: string;
	deniedAccessRedirect: string | null;
	dnsTTL: bigint | null;
	edns0ClientSubnetEnabled: boolean;
	geographicLimiting: Set<string> | null;
	maximumRecords: bigint | null;
	missLocation: {
		latitude: number;
		longitude: number;
	} | null;
	name: string;
	notes: string;
	routingName: string;
	routingType: 'HTTP' | 'DNS' | 'STEERING' | 'STATIC';
	status: 'ACTIVE' | 'PRIMED' | 'INACTIVE';
	tags: Set<string>;
	tenant: string;
	vanityHostnames: string;
}
\end{minted}
\end{codelisting}

\paragraph{Anonymous Blocking}
A Delivery Service that has "Anonymous Blocking" tells Traffic Router to block
requests from anonymized IP addresses. Whether or not and how well that can
actually be done is dependent on the configuration of each Traffic Router
itself, and if this Delivery Service is DNS-routed the only IP address Traffic
Routers will be capable of checking for anonymization (e.g. known proxy/VPN/TOR
exit node) will be the downstream router requesting the name resolution and thus
is likely much less effective.

\paragraph{Bypass Destination}
This is a string that describes the network location to which clients will be
directed if the traffic served by this Delivery Service exceeds its allowed
maximums. This MUST always be represented as a string - even if the
representation format supports IP Addresses as a native type - and its
interpretation is dependent on the Delivery Service's Routing Type.\\
If the Delivery Service's Routing Type is \code{HTTP}, then this is interpreted
- and validated by the API - as a Fully Qualified Domain Name (FQDN) optionally
followed by a colon and port number that defines an HTTP server to which client
requests will be directed.\\
If the Delivery Service's Routing Type is \code{DNS} or \code{STATIC}, then if
this is a valid IPv4 address it is assumed to be one and will be presented as an
AA record. If not, and it's a valid IPv6 address, then it is assumed to one and
will be presented as an AAAA record. Finally, it may be an FQDN in which case it
will be presented as a CNAME record. If none of these formats can be validated,
then it MUST be rejected by the API.\\
If the Delivery Service's Routing Type is \code{STEERING}, then this MUST be the
Name of an existing Delivery Service - though it need not name one of this
Delivery Service's Targets.

\paragraph{CDN}
The CDN to which a Delivery Service belongs is expressed as a string that is the
Name used to uniquely identify it.

%TODO: use-case for check path?

\paragraph{Deep Caching}
A boolean value that describes whether or not "Deep Caching" may be used for
this Delivery Service.

\paragraph{Denied Access Redirect}
This is a string that describes the network location to which clients will be
directed if they are denied access on the basis of Anonymous Blocking and/or
Geographic Limiting settings. This MUST always be represented as a string - even
if the representation format supports IP Addresses as a native type - and its
interpretation is dependent on the Delivery Service's Routing Type.

\paragraph{DNS TTL}
An unsigned integer that defines the Time-To-Live (TTL) of DNS responses from
the Traffic Router for this Delivery Service's routing, in seconds.
If this is null-typed, the Traffic Router serving this Delivery Service's
traffic will use its pre-configured default TTL for whatever type of record was
requested.

\paragraph{EDNS0 Client Subnet Enabled}
A boolean that describes whether or not the EDNS0 DNS extension mechanism
described in \href{RFC2671}{https://tools.ietf.org/html/rfc2671} should be made
available to clients.\\
Note that the ability of a Traffic Router to actually implement this setting
depends on its own EDNS0 Client Subnet enabled value.

%TODO: Necessary?
%\paragraph{ECS}

\paragraph{Geographic Limiting}
This property describes limitations to the availability of this Delivery
Service's content on the basis of the requesting client's geographic location.
It is a set of strings, each of which is an
\href{https://www.iso.org/obp/ui/#search/code/}{ISO 3166-1} alpha-2 country
code, optionally with ISO 3166-2 subdivisional alphabetic code. This is a "white
list" of countries/subdivisions wherein content is to be made
available\footnote{This property is meant to inform Traffic Router; Cache
Servers cannot be relied upon to approve or deny access on a geographic basis.
Thus, if routing is bypassed, restricted content is totally accessible to
requesting clients.}.\\
Content is \emph{always} available to clients whose IP addresses are found
within the Traffic Routers' Coverage Zone File(s). With that in mind, when this
property is an empty set it means that no geographic regions are "whitelisted"
and thus \emph{only} clients whose IP addresses are found within a Coverage Zone
File will be granted access to content. When this property has a "Null" type,
there is no geographic restriction placed on the Delivery Service's content
access.

%TODO: use-case for header-rewrite text?

\paragraph{Maximum Records}
This unsigned integer sets the maximum number of records returned in DNS
responses. It MUST be greater than zero, or null-typed, which indicates that
there should be no limit placed on the maximum returned records.

\paragraph{Miss Location}
This is an object with two properties, latitude and longitude, which are
floating-point numbers that define a pair of geographic coordinates to be used as
a fallback in the event that attempts to find a geographic location for a client
based on Coverage Zone Files and IP address look-up have failed. If this is
null-typed, then there is no configured miss location and Traffic Router will
use its pre-configured default fallback location. \footnote{The Miss Location of
a Delivery Service is uneditable by users that do not have the
\code{delivery-service-miss-location} Permission (as well as any other
Permission(s) required by the endpoint used for editing).}

\paragraph{Name}
A Delivery Service's "Name" is a string that uniquely identifies it among all
Delivery Services. It MUST only contain alphanumerics, hyphens, underscores and
spaces, and MUST NOT begin with a non-alphanumeric character nor end with a
non-alphanumeric character. This is used to generate part of the default request
hostnames by replacing all non-alphanumeric characters with a hyphen.

\paragraph{Notes}
This section is an arbitrary string containing miscellaneous, human-friendly
information about the Delivery Service. Other ATC components SHOULD NOT parse
this for specific information fields, or expect it to be in a particular format.

%TODO Raw-Remap use-case?
%TODO Regex-Remap use-case?

\paragraph{Routing Name}
The lowest-level DNS label (e.g. in \code{traffic-control-cdn.readthedocs.io}
the lowest-level DNS label is \code{traffic-control-cdn}) used to route to a
Delivery Service is set by this string. UIs currently default this value to
"cdn", but it may be changed for vanity purposes. It MAY NOT be an empty string
and MUST be a valid DNS label.

%TODO: ANY_MAP use-case?
\paragraph{Routing Type}
After creation, Routing Type is a read-only property. It's a string that
describes the type of routing used to serve content for the Delivery Service.
Each of these routing types has its own section below.
\begin{itemize}
	\item \code{HTTP}
	\item \code{DNS}
	\item \code{STEERING}
	\item \code{STATIC}
\end{itemize}

\paragraph{Status}
The "Status" of a Delivery Service is a string constant that expresses its
ability to serve content at the present moment in time. It may have one of three
values:

\begin{itemize}
	\item \code{ACTIVE} A Delivery Service that is "active" is one that is
	functionally in service, and fully capable of delivering content. This means
	that its configuration is deployed to Cache Servers and it is available for
	routing traffic.
	\item \code{PRIMED} A Delivery Service that is "primed" has had its
	configuration distributed to the various servers required to serve its
	content. However, the content itself is still inaccessible\footnote{The
	content is not available through normal routing. This does not, though,
	guarantee that Cache Servers do not already have the content stored and/or
	are incapable of serving it if routing is bypassed.}.
	\item \code{INACTIVE} A Delivery Service that is "inactive" is not available
	for routing and has not had its configuration distributed to its assigned
	Cache Servers.
\end{itemize}

\paragraph{Supported Protocols}
This is a set of strings that name protocols served by the Delivery Service.
Note that this is the method used to retrieve content from the caching system,
not the method used for routing. The only protocols officially supported by ATC
are "HTTP" and "HTTPS".\\
This set is case-insensitive, such that if a Delivery Service is created with a
Supported Protocols set containing "HTTP" the resulting set is equivalent to
what would result from creating it with a Supported Protocols set containing
"http". Representations produced by the Traffic Ops API MUST always use
only uppercase characters.

\paragraph{Tags}
The Tags associated with a Delivery Service are represented by a set of strings
that are Tag Names.

\paragraph{Tenant}
The Tenant to which a Delivery Service belongs is represented by a string that
is that Tenant's unique Name.

\paragraph{Vanity Hostnames}
"Vanity Hostnames" is a set of strings that are Fully Qualified Domain Names
(FQDN) which may be used as alternates to the standard
\code{\emph{Routing Name}.\emph{Name}.\emph{CDN Domain}} FQDN when requesting
content from the Delivery Service.\\
No two Delivery Services may be allowed to share any single Vanity Hostname.\\
Note that Traffic Ops - and in fact Traffic Control in general - cannot and
does not guarantee that these vanity names will work, only that Traffic Router
will respond to them as equivalents to a normal Delivery Service FQDN when
content is requested through it. In general, because Vanity Hostnames are
typically outside of the CDN's Domain (which is the only domain for which
Traffic Router must be authoritative), this requires the DNS servers that are
authoritative for each Vanity Hostname's Domain to contain records that will
point to Traffic Router for resolution of these names.

\subsubsection{DNS-Routed Properties}
A DNS-Routed Delivery Service is defined primarily by having a Routing Type of
"DNS". These Delivery Services direct clients to content by responding to DNS
queries for Delivery Service hostnames with records indicating the network
addresses of cache servers.

\begin{codelisting}
\captionof{listing}{DNS-Routed Delivery Service as a Typescript Interface}
\begin{minted}[tabsize=2]{typescript}
interface DNSDeliveryService extends BaseDeliveryService {
	bypassTTL: bigint | null;
	caching: 'CACHE' | 'RAM_ONLY' | 'NO_CACHE';
	dscp: bigint;
	maximumOriginConnections: bigint | null;
	origin: bigint | (Array<bigint> & {0: bigint});
	queryStringHandling: 'DROP' | 'IGNORE' | 'USE';
	rangeRequestHandling: 'NO_CACHE' | 'WHOLE_OBJECT' | 'CACHE';
	requiredCapabilities: Set<string>;
	routingType: 'DNS';
	topology: string;
}
\end{minted}
\end{codelisting}

\paragraph{Bypass TTL}
An unsigned integer that defines the Time-To-Live (TTL) of DNS responses from
the Traffic Router for this Delivery Service's Bypass Destination, in seconds.
If this is null-typed, the Traffic Router serving this Delivery Service's
traffic will use its pre-configured default TTL for whatever type of record was
requested.

\paragraph{Caching}
Caching describes how Delivery Service content is cached - if at all. It is a
string content restricted to one of the values:

\begin{itemize}
	\item \code{CACHE} The Delivery Service's content will be cached normally.
	\item \code{RAM\_ONLY} The Delivery Service's content will only be cached
	in RAM block devices.
	\item \code{NO\_CACHE} The Delivery Service's content is proxied through
	Cache Servers without ever being actually cached.
\end{itemize}

\paragraph{DSCP}
Sets the
\href{https://tools.ietf.org/html/rfc2474}{Differentiated Services Code Point}
which will be marked on the Delivery Service's traffic. This is an unsigned
integer with a maximum value of 64.

\paragraph{Maximum Origin Connections}
This is an unsigned integer which determines the maximum number of connections
that any \emph{one} Cache Server may open to the Delivery Service's Origin.\\
If this is null-typed, it has the special meaning "no limit".

\paragraph{Origin}
In normal cases, the Origin whose content is served by this Delivery Service is
identified by an unsigned integer that is its ID. However, in the event that
the Delivery Service is a Multi-Site Origin Delivery Service, it will have an
array of such identifiers indicating all Origins used by the Delivery Service.
More detail about Multi-Site Origin Delivery Services can be found in
\ref{sec:mso-props}.

\paragraph{Query String Handling}
Query String Handling is a string with one of three possible values that
describe how the cache layers should handle query strings in requested URLs.

\begin{description}
	\item[\code{DROP}] The first cache layer encountered (EDGE-tier Cache
	Servers) will totally ignore all query strings, they will not be used in the
	cache key, and will not be passed upstream in requests.
	\item[\code{IGNORE}] The first cache layer encountered (EDGE-tier Cache
	Servers) will not use query strings as part of the cache key, but will pass
	them upstream in requests.
	\item[\code{USE}] The first cache layer encountered (EDGE-tier Cache
	Servers) will use query strings in the cache key and will pass them upstream
	in requests.
\end{description}

\paragraph{Range Request Handling}
Range Request Handling is a string with one of three possible values that
describe how cache servers should handle HTTP Range requests.

\begin{description}
	\item[\code{NO\_CACHE}] Range requests will not be cached, and requests for
	ranges will always be proxied upstream all the way to the origin.
	\item[\code{WHOLE\_OBJECT}] When a request for a range of an object is
	received, the entire object is fetched from upstream and cached, and
	subsequent range requests will be served by taking ranges of the cached
	object.
	\item[\code{CACHE}] Range requests will be cached as normal requests, each
	unique range representing a unique, cached object - this includes ranges
	that may overlap.
\end{description}

\paragraph{Required Capabilities}
The Capabilities required by a Delivery Service for cache servers to serve its
content are given by a set of strings that are the names of those Capabilities.

\paragraph{Topology}
The Topology used by a Delivery Service is represented by a string that is the
Topology's unique Name.

\subsubsection{HTTP-Routed Properties}
An HTTP-Routed Delivery Service is defined primarily by having a Routing Type of
"HTTP". These Delivery Services direct clients to content by responding to DNS
queries for Delivery Service hostnames with records indicating its own network
address, then replying to subsequent HTTP requests for Delivery Service content
with HTTP 302 Found responses directing clients to cache servers.

\begin{codelisting}
\captionof{listing}{HTTP-Routed Delivery Service as a Typescript Interface}
\begin{minted}[tabsize=2]{typescript}
interface HTTPDeliveryService extends BaseDeliveryService {
	additionalResponseHeaders: Map<string, string>;
	caching: 'CACHE' | 'RAM_ONLY' | 'NO_CACHE';
	consistentHashingRegularExpression: RegExp | null;
	dscp: bigint;
	loggedRequestHeaders: Set<string>;
	maximumOriginConnections: bigint | null;
	origin: bigint | (Array<bigint> & {0: bigint});
	queryStringHandling: 'DROP' | 'IGNORE' | 'USE';
	rangeRequestHandling: 'NO_CACHE' | 'WHOLE_OBJECT' | 'CACHE';
	requiredCapabilities: Set<string>;
	routingType: 'HTTP';
	significantQueryParameters: Set<bigint> | null;
	topology: string;
}
\end{minted}
\end{codelisting}

\paragraph{Additional Response Headers}
Any additional HTTP headers desired to appear in HTTP responses from Traffic
Router are indicated here as header names mapped to their desired values.

\paragraph{Caching}
Caching describes how Delivery Service content is cached - if at all. It is a
string content restricted to one of the values:

\begin{itemize}
	\item \code{CACHE} The Delivery Service's content will be cached normally.
	\item \code{RAM\_ONLY} The Delivery Service's content will only be cached
	in RAM block devices.
	\item \code{NO\_CACHE} The Delivery Service's content is proxied through
	Cache Servers without ever being actually cached.
\end{itemize}

\paragraph{Consistent Hashing Regular Expression}
When Traffic Router performs consistent hashing on client HTTP requests to find
a cache server to which to redirect, it will use this regular expression - if it
is not null-typed - to extract the parts of the request path to use as the
hashing key.

\paragraph{DSCP}
Sets the
\href{https://tools.ietf.org/html/rfc2474}{Differentiated Services Code Point}
which will be marked on the Delivery Service's traffic. This is an unsigned
integer with a maximum value of 64.

\paragraph{Logged Request Headers}
Any HTTP headers in client requests that should appear in Traffic Router logs
are identified in this set of string by header name.

\paragraph{Maximum Origin Connections}
This is an unsigned integer which determines the maximum number of connections
that any \emph{one} Cache Server may open to the Delivery Service's Origin.\\
If this is null-typed, it has the special meaning "no limit".

\paragraph{Origin}
In normal cases, the Origin whose content is served by this Delivery Service is
identified by an unsigned integer that is its ID. However, in the event that
the Delivery Service is a Multi-Site Origin Delivery Service, it will have an
array of such identifiers indicating all Origins used by the Delivery Service.
More detail about Multi-Site Origin Delivery Services can be found in
\ref{sec:mso-props}.

\paragraph{Query String Handling}
Query String Handling is a string with one of three possible values that
describe how the cache layers should handle query strings in requested URLs.

\begin{description}
	\item[\code{DROP}] The first cache layer encountered (EDGE-tier Cache
	Servers) will totally ignore all query strings, they will not be used in the
	cache key, and will not be passed upstream in requests.
	\item[\code{IGNORE}] The first cache layer encountered (EDGE-tier Cache
	Servers) will not use query strings as part of the cache key, but will pass
	them upstream in requests.
	\item[\code{USE}] The first cache layer encountered (EDGE-tier Cache
	Servers) will use query strings in the cache key and will pass them upstream
	in requests.
\end{description}

\paragraph{Range Request Handling}
Range Request Handling is a string with one of three possible values that
describe how cache servers should handle HTTP Range requests.

\begin{description}
	\item[\code{NO\_CACHE}] Range requests will not be cached, and requests for
	ranges will always be proxied upstream all the way to the origin.
	\item[\code{WHOLE\_OBJECT}] When a request for a range of an object is
	received, the entire object is fetched from upstream and cached, and
	subsequent range requests will be served by taking ranges of the cached
	object.
	\item[\code{CACHE}] Range requests will be cached as normal requests, each
	unique range representing a unique, cached object - this includes ranges
	that may overlap.
\end{description}

\paragraph{Required Capabilities}
The Capabilities required by a Delivery Service for cache servers to serve its
content are given by a set of strings that are the names of those Capabilities.

\paragraph{Significant Query Parameters}
If this is not null-typed, then client HTTP request paths are modified prior to
consistent hashing by parsing any query string that's present in the request
path as an application/x-www-form-urlencoded set of parameter-value pairs, and
stripping all pairs that don't have keys present in this set of strings.

\paragraph{Topology}
The Topology used by a Delivery Service is represented by a string that is the
Topology's unique Name.

\subsubsection{Static-Routed Properties}
\begin{codelisting}
\captionof{listing}{Static-Routed Delivery Service as a Typescript Interface}
\begin{minted}[tabsize=2]{typescript}
interface StaticDeliveryService extends BaseDeliveryService {
	origin: string;
	routingType: 'STATIC';
}
\end{minted}
\end{codelisting}

\paragraph{Origin}

\subsubsection{Steering-Routed Properties}
\begin{codelisting}
\captionof{listing}{Steering-Routing Delivery Service as a Typescript Interface}
\begin{minted}[tabsize=2]{typescript}
interface SteeringTarget {
	target: string;
	type: 'WEIGHT' |
		'ORDER' |
		'GEO_WEIGHT' |
		'GEO_ORDER';
	value: bigint;
}

interface SteeringDeliveryService extends BaseDeliveryService {
	routingType: 'STEERING';
	targets: Set<SteeringTarget>;
}
\end{minted}
\end{codelisting}

\paragraph{Targets}
The "target" Delivery Services of a Steering-Routed Delivery Service are given
as a set of of Target objects which express which Delivery Service each target
is as well as the rules for balancing requests between them. The properties of
these Targets are described in this paragraph.

\subparagraph{Target}
This string is the Name of the Delivery Service to which the Target refers. No
two Targets may exist within the same Targets set that have the same Type and
Target values.

\subparagraph{Type}
This string defines how and/or under what conditions the Target is chosen as the
"destination" Delivery Service for client requests for content from the "parent"
Steering-Routed Delivery Service. It may have one of the following values with
the associated meanings:

\begin{description}
	\item[ORDER] The Value of an ORDER Target sets a strict order of preference.
	In cases where a response to a client contains multiple Delivery Services,
	those Targets with a lower Value appear earlier than those with a higher
	Value. In cases where two or more Targets share the same Value, they each
	have an equal chance of being presented to the client - effectively
	spreading traffic evenly across them.
	\item[WEIGHT] The Values of WEIGHT Targets are interpreted as “weights”,
	which define how likely it is that any given client will be routed to a
	specific Delivery Service - effectively this determines the spread of
	traffic across each Target of this Type.
	\item[GEO\_ORDER] These Targets behave exactly like ORDER Targets, but
	Delivery Services are grouped according to the “locations” of their Origins.
	Before choosing a Delivery Service to which to direct the client, Traffic
	Router will first create subsets of choices according to these groupings,
	and order them by physical distance from the client (closest to farthest).
	Within these subsets, The Values of the Targets establish a strict
	precedence ordering, just like ORDER Targets.
	\item[GEO\_WEIGHT] These targets behave exactly like WEIGHT targets, but
	Delivery Services are grouped according to the “locations” of their Origins.
	Before choosing a Delivery Service to which to direct the client, Traffic
	Router will first create subsets of choices according to these groupings,
	and order them by physical distance from the client (closest to farthest).
	Within these subsets, the Values of the Targets establish the likelihood
	that any given target within the subset will be chosen for the client -
	effectively determining the spread of traffic across Targets within that
	subset.
\end{description}

The precedence order when multiple Targets of disparate Types exist within the
same Targets set is undefined.

\subparagraph{Value}
This unsigned integer defines the conditions under which this Delivery Service
will be chosen as the destination Delivery Service for client requests to the
Steering-Routed "parent" Delivery Service. The exact process by which Traffic
Router makes these decisions is determined by the Target's Type.

\subsubsection{Multi-Site Origin Properties\label{sec:mso-props}}
Multi-Site Origin (MSO) Delivery Services are, simply, Delivery Services with
more than one Origin. They have special properties that define the interactions
between the caching layers and these multiple Origins. Thus, only Delivery
Services which include the ability to cache may be MSO Delivery Services -
specifically only DNS-routed and HTTP-routed Delivery Services.

\begin{codelisting}
\captionof{listing}{Multi-Site Origin Delivery Services as Typescript Interfaces}
\begin{minted}[tabsize=2]{typescript}
interface DNSMSODeliveryService extends DNSDeliveryService {
	maximumSimpleRetries: bigint | null;
	maximumUnavailableServerRetries: bigint | null;
	msoAlgorithm: 'ALWAYS_FIRST' |
		'ROUND_ROBIN' |
		'LATCHED' |
		'STRICT' |
		'CONSISTENT_HASH';
	origin: Array<bigint> & {0: bigint};
	parentRetry: 'BOTH' |'SIMPLE' | 'UNAVAILABLE_SERVER';
	serverRetryResponses: Set<bigint>;
}

interface HTTPMSODeliveryService extends HTTPDeliveryService {
	maximumSimpleRetries: bigint | null;
	maximumUnavailableServerRetries: bigint | null;
	msoAlgorithm: 'ALWAYS_FIRST' |
		'ROUND_ROBIN' |
		'LATCHED' |
		'STRICT' |
		'CONSISTENT_HASH';
	origin: Array<bigint> & {0: bigint};
	parentRetry: 'BOTH' |'SIMPLE' | 'UNAVAILABLE_SERVER';
	serverRetryResponses: Set<bigint>;
}
\end{minted}
\end{codelisting}

\paragraph{Maximum Simple Retries}
This unsigned integer sets the number of times that Cache Servers will select
new parents and retry requests upon receiving a 404 Not Found response from the
origin.\\
This may be null-typed if and only if Parent Retry is "UNAVAILABLE\_SERVER".

\paragraph{Maximum Unavailable Server Retries}
This unsigned integer sets the number of times that Cache Servers will select
new parents and retry requests upon receiving a response code found in the
Server Retry Responses set - the Cache Servers in question will also mark the
parent returning that response code as "down" according to the semantics of the
specific caching proxy software\footnote{In particular, different versions of
Apache Traffic Server might use different timings to mark servers as "up" after
having been "down". Consult caching proxy software documentation for details.}
used by the Cache Servers.\\
This may be null-typed if and only if Parent Retry is "SIMPLE".

\paragraph{MSO Algorithm}
This string defines the algorithm used for parent selection at the highest
caching tier. It may have one of the following values with the associated
meanings:

\begin{description}
	\item[ALWAYS\_FIRST] The first parent in the Origin array is chosen to
	service every request until or unless it becomes marked "down", at which
	point the second parent will be chosen for every request until the previous
	parent becomes "up" again, and so on until the list is exhausted (at which
	point requests will, of course, fail if they are not serviceable from the
	cache).\footnote{This algorithm is known in current API versions as the
	\code{mso.algorithm} Parameter Value "False", as that is the exact value in
	the \code{round\_robin} field of an Apache Traffic Server
	\code{parent.config} file entry that is used to accomplish this.}
	\item[CONSISTENT\_HASH] The parent Origin is chosen using a consistent hash
	of the request URL so that one parent is chosen for a given URL. If a parent
	is "down", the traffic that would go to the "down" parent is re-hashed
	amongst the remaining parents. The other traffic is unaffected. Once the
	"downed" parent becomes available, the traffic distribution returns to the
	pre-"down" state.
	\item[LATCHED] Like ALWAYS\_FIRST, the first parent in the Origin array is
	chosen to service every request until or unless it becomes marked "down", at
	which point the second parent will be chosen for every request. Unlike
	ALWAYS\_FIRST, however, the first parent will not be chosen again once it
	becomes "up". Instead, each successively chosen parent will be used until
	and unless it becomes marked "down", and the list of parents is always
	"traversed" by successive marking-"down"s in forward order.
	\item[STRICT] Requests to parent Origins will be divided strictly between
	them according to no metric other than the absolute number of requests. That
	is, the first parent will be selected to service the first request, the
	second parent will be selected to service the second request, wrapping back
	to the first parent once the list has been exhausted.
	\item[ROUND\_ROBIN] Requests to parent Origins will be divided between them
	based on the requesting client or lower-level Cache Server. That is, the
	first parent will be chosen to service the first request, and the second
	request will use the second parent if and only if it originates from a
	different IP address than the first.\footnote{This algorithm is known in
	current API versions as the \code{mso.algorithm} Parameter Value "True", as
	that is the exact value in the \code{round\_robin} field of an Apache
	Traffic Server \code{parent.config} file entry that is used to accomplish
	this.}
\end{description}

\paragraph{Origin}
In MSO Delivery Services, Origin is an Array of Origin IDs, each of which may be
selected as a parent by the highest/last caching layer to service requests for
content that cannot be serviced from cache. There must be at least one ID in the
array.\\
The meaning of the order of the parents is defined by the setting of MSO
Algorithm.

\paragraph{Parent Retry}
This string defines how Cache Servers react in various cases where content from
the parent Origin that was selected to service a request is not available. It
may have one of the following values with the associated meanings:

\begin{description}
	\item[BOTH] The behavior of both SIMPLE and UNAVAILABLE\_SERVER
	is exhibited by Cache Servers.
	\item[SIMPLE] Cache Servers will select a new parent - according to the MSO
	Algorithm - and re-try requests when they receive a 404 Not Found response
	from the selected parent Origin.
	\item[UNAVAILABLE\_SERVER] Cache Servers will select a new parent -
	according to the MSO Algorithm - and re-try requests when they receive a
	503 Service Unavailable response from the selected parent Origin \emph{or}
	when they receive any response code from the selected parent Origin that is
	in the Server Retry Responses set.
\end{description}

\paragraph{Server Retry Responses}
This set of unsigned integers defines the set of response codes that Cache
Servers will consider indicative of a parent Origin being "down". Note that the
response code 503 Service Unavailable \emph{always} indicates this, and need not
be present in this set (which of course means that the behavior of considering
the status code an indicator that the parent is "down" cannot be disabled.)
