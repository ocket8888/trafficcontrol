% Licensed to the Apache Software Foundation (ASF) under one
% or more contributor license agreements.  See the NOTICE file
% distributed with this work for additional information
% regarding copyright ownership.  The ASF licenses this file
% to you under the Apache License, Version 2.0 (the
% "License"); you may not use this file except in compliance
% with the License.  You may obtain a copy of the License at
%
%   http://www.apache.org/licenses/LICENSE-2.0
%
% Unless required by applicable law or agreed to in writing,
% software distributed under the License is distributed on an
% "AS IS" BASIS, WITHOUT WARRANTIES OR CONDITIONS OF ANY
% KIND, either express or implied.  See the License for the
% specific language governing permissions and limitations
% under the License.
\subsection{Infrastructure Servers}
Infrastructure servers are arbitrary servers that aren't ATC components. This
is intended primarily for record-keeping purposes for organizations with large
support infrastructure for their CDNs, as no out-of-the-box functionality is
provided by ATC for these servers.

\begin{codelisting}
\captionof{listing}{Infrastructure Server Object as a TypeScript Interface}
\begin{minted}[tabsize=2]{typescript}
interface InfrastructureServer {
	cdn: string | null;
	domain: string;
	hostName: string;
	id: bigint;
	notes: string;
	online: boolean;
	physicalLocation: string | null;
	tags: Set<string>;
	servicePort: int | null;
	serviceProtocol: string | null;
}
\end{minted}
\end{codelisting}

\subsubsection{CDN}
This is a string that is the Name of the CDN to which the Infrastructure Server
belongs, if any. If this is "null"-typed, then it is understood that the server
does not operate in a way that is limited in scope to a single CDN.

\subsubsection{Domain}
The "domain" part of the Infrastructure Server's Fully Qualified Domain Name
(FQDN) as a string. For example, an Infrastructure Server with an FQDN of
\code{github.com} has a Domain of \code{com} and an Infrastructure Server with
an FQDN of \code{trafficcontrol.apache.org} has a Domain of
\code{apache.org}.\\
In other words, this is every part of the Infrastructure Server's FQDN that
isn't part of its Host Name.

\subsubsection{Host Name}
This is the "host" part of the Infrastructure Server's Fully Qualified Domain
Name (FQDN) as a string. For example, an Infrastructure Server with an FQDN of
\code{github.com} has a Host Name of \code{github}, and an Infrastructure
Server with an FQDN of \code{trafficcontrol.apache.org} has a Host Name of
\code{trafficcontrol}.\\
In other words, an Infrastructure Server's Host Name is the only part of its
FQDN that is not part of its Domain.\\
This field does NOT need to be unique, though operators are encouraged to make
this - or at least the concatenation \code{\emph{Host Name}.\emph{Domain}} -
unique for ease of operation.

\subsubsection{ID}
An integral, unique identifier for the Infrastructure Server. It carries no
meaning or significance beyond being a unique identifier.

\subsubsection{Notes}
This is a string of arbitrary text for miscellaneous purposes.

\subsubsection{Online}
This boolean value indicates whether or not the Infrastructure Server is online
and/or currently serving its main service.

\subsubsection{Physical Location}
This string gives the Name of a Physical Location within which this
Infrastructure Server resides. If this is "null"-typed value, then it is
understood that the Infrastructure Server has no well-defined and/or
meaningful Physical Location.

\subsubsection{Tags}
The Tags on an Infrastructure Server are represented by a set of Tag Names as
strings.

\subsubsection{Service Port}
This unsigned integer is the network port on which the server's primary service
listens for connections, e.g. \code{80} for HTTP servers.\\
This value may not exceed 65535. If this value is "null"-typed, then it is
understood that the service provided by this server may not be reached on any
particular network port (reliably or at all).

\subsubsection{Service Protocol}
This string of alphanumeric characters defines the protocol used by the
Infrastructure Server's primary service, e.g. \code{"HTTP"} for HTTP servers.\\
If this is "null"-typed, then it is understood that the server's primary service
is not offered using any particular protocol (reliably or at all).
