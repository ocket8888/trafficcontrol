% Licensed to the Apache Software Foundation (ASF) under one
% or more contributor license agreements.  See the NOTICE file
% distributed with this work for additional information
% regarding copyright ownership.  The ASF licenses this file
% to you under the Apache License, Version 2.0 (the
% "License"); you may not use this file except in compliance
% with the License.  You may obtain a copy of the License at
%
%   http://www.apache.org/licenses/LICENSE-2.0
%
% Unless required by applicable law or agreed to in writing,
% software distributed under the License is distributed on an
% "AS IS" BASIS, WITHOUT WARRANTIES OR CONDITIONS OF ANY
% KIND, either express or implied.  See the License for the
% specific language governing permissions and limitations
% under the License.

\subsection{CDNs}
A CDN is a full collection of caching and infrastructural servers and ATC
components required to service distributed requests for content, as well as the
configuration options for the methods of content delivery and the content
delivered therein.

\begin{codelisting}
\captionof{listing}{CDN Object as a TypeScript Interface}
\label{code:datamodel:cdn}
\begin{minted}[tabsize=2]{typescript}
interface CDN {
	cacheServers: Set<bigint>;
	deliveryServices: Set<string>;
	dnssecEnabled: boolean;
	domain: string;
	infrastructureServers: Set<bigint>;
	name: string;
	origins: Set<bigint>;
	trafficMonitors: Set<bigint>;
	trafficRouters: Set<bigint>;
}
\end{minted}
\end{codelisting}

\subsubsection{Cache Servers}
A CDN's collection of Cache Servers is represented as set of Cache Server IDs.

\subsubsection{Delivery Services}
A CDN's collection of Delivery Services is represented as a set of Delivery
Service Names.

\subsubsection{DNSSEC Enabled}
A CDN that has "DNSSEC Enabled" is capable of supporting DNSSEC configurations
on its constituent Delivery Services. This does \emph{not} mean that all DNS
requests through ATC for this CDN's Domain will be secured with DNSSEC.\\
This is a boolean concept and as such is represented by boolean values.

\subsubsection{Domain}
The Top-Level Domain used by the CDN, represented as a string. This MUST be a
uniquely valued property across all CDNs, and MUST not be an invalid DNS label
as defined by
\href{https://tools.ietf.org/html/rfc1035#section-2.3.1}{Section 2.3.1 of RFC1035}.

\subsubsection{Infrastructure Servers}
A CDN's collection of Infrastructure Servers is represented as a set of
Infrastructure Server IDs.

\subsubsection{Name}
A CDN's "Name" is a string that uniquely identifies it among all CDNs. It MUST
NOT be allowed to contain "special characters" - meaning anything other than
alphanumeric characters, spaces, hyphens, colons and underscores. It further
MUST NOT be allowed to begin with a non-alphanumeric character nor end with a
space.

\subsubsection{Origins}
A CDN's collection of Origins is represented by a set of Origin IDs.

\subsubsection{Traffic Monitors}
A CDN's collection of Traffic Monitors is represented by a set of Traffic
Monitor IDs. Note that while a functioning CDN must contain at least one Traffic
Monitor, no such restriction is placed on CDN objects in general.

\subsubsection{Traffic Routers}
A CDN's collection of Traffic Routers is represented by a set of Traffic Router
IDs. Note that while a functioning CDN must contain at least one Traffic Router,
no such restriction is placed on CDN objects in general.
