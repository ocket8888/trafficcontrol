% Licensed to the Apache Software Foundation (ASF) under one
% or more contributor license agreements.  See the NOTICE file
% distributed with this work for additional information
% regarding copyright ownership.  The ASF licenses this file
% to you under the Apache License, Version 2.0 (the
% "License"); you may not use this file except in compliance
% with the License.  You may obtain a copy of the License at
%
%   http://www.apache.org/licenses/LICENSE-2.0
%
% Unless required by applicable law or agreed to in writing,
% software distributed under the License is distributed on an
% "AS IS" BASIS, WITHOUT WARRANTIES OR CONDITIONS OF ANY
% KIND, either express or implied.  See the License for the
% specific language governing permissions and limitations
% under the License.

\section{Guidelines}
This section lays out guidelines to be considered by API endpoint authors. What follows are not \emph{strictly} hard-and-fast rules, but
there should be a very convincing argument accompanying endpoints that do not follow them.

\subsection{Response Bodies}
All valid API responses will be in the form of some serialized object. The main
data that represents the result of the clients request MUST appear in the
\code{response} property of that object. If a warning, error message, success
message, or informational message is to be issued by the server, then they MUST
appear in the \code{alerts} property of the response. Some endpoints may return
ancillary statistics such as the total number of objects when pagination occurs
(see Section \ref{sec:pagination}), which should be placed in the \code{summary}
property of the response.\\

\subsubsection{Response}
The \code{response} property of a serialized response object MUST only contain
object representations as requested by the client. In particular, it MUST NOT
contain admonitions, success messages, informative messages, or statistic
summaries beyond the scope requested by the client.\\
Equally unacceptable API responses are shown in Listings
\ref{code:badresponseexample:success-as-response} and
\ref{code:badresponseexample:extra-field}.

\begin{codelisting}
\captionof{listing}{Success message as response object}
\label{code:badresponseexample:success-as-response}
\begin{minted}[tabsize=2]{json}
{
	"response": "Thing was successfully created."
}
\end{minted}
\end{codelisting}

\begin{codelisting}
\captionof{listing}{Extra top-level property}
\label{code:badresponseexample:extra-field}
\begin{minted}[tabsize=2]{json}
{
	"response": {"foo": "bar"},
	"someOtherField": {"someOtherObject"}
}
\end{minted}
\end{codelisting}

When requests are serviced by Traffic Ops that pass data asking that the returned object list be filtered, the \code{response} property
MUST be a filtered array of those objects (assuming the request may be successfully serviced). This is true even if filtering is being
done according to a uniquely identifying property - e.g. a numeric ID. The \code{response} field of an object returned in response to a
request to create, update, or delete one or more resources may be either a single object representation or an array thereof according to
the number of objects created, updated, or deleted. However, if a handler is \emph{capable} of creating, updating, or deleting more than
one object at a time, the \code{response} field SHOULD consistently be represented as an array - even if its length would only be 1.\\
The proper value of an empty collection is an empty collection. If a Foo can have zero or more Bars, then the representation of a Foo with
no Bars MUST be an empty array/list/set, and in particular MUST NOT be either missing from the representation or represented as the "Null"
value of the representation format. That is, if Foos have no other property than their Bars, then a Foo with no Bars may be represented
in JSON encoding as \code{\{"bars":[]\}}, but not as \code{\{"bars":null\}} or \code{\{\}}. Similarly, an empty string field is properly
represented as an empty string - e.g. \code{\{"bar":""\}} not \code{\{"bar":null\}} or \code{\{\}} - and the "zero-value" of numbers is
zero itself - e.g. \code{\{"bar":0\}} not \code{\{"bar":null\}} or \code{\{\}}. Note that "null" values \emph{are allowed} when
appropriate, but "null" values represent the \emph{absence} of a value rather than the "zero-value" of a property. If a property is
\emph{missing} from an object representation it indicates the absence of that property, and because of that there must be a \emph{very
convincing} argument if and when that is the case.\\
As a special case, endpoints that report statistics including minimums,
maximums and arithmetic means of data sets MUST use the property names
\code{min}, \code{max}, and \code{mean}, respectively, to express those
concepts. These SHOULD be properties of \code{response} directly whenever that
makes sense.

\subsubsection{Alerts}
Alerts should be presented as an array containing objects which each conform to the object definition laid out by
\href{https://godoc.org/github.com/apache/trafficcontrol/lib/go-tc#Alert}{the ATC library's Alert structure}. The allowable \code{level}s of an Alert are:

\begin{itemize}
	\item \code{error} - This level MUST be used to indicate conditions that caused a request to fail. Because of this, this level MUST NOT appear
	in the \code{alerts} array of responses with any HTTP response code less than 400. Details of server workings and/or failing
	components MUST NOT be exposed in this message, which should otherwise be as descriptive as possible.
	\item \code{info} - This level SHOULD be used to convey supplementary information to a user that is not directly the result of their request.
	This SHOULD NOT carry information indicating whether or not the request succeeded and why/why not, as that is best left to the \code{error}
	and \code{success} levels.
	\item \code{success} - This level MUST be used to convey success messages to the client. In general, it is expected that the message will be
	directly displayed to the user by the client, and therefore can be used to add human-friendly details about a request beyond the response
	payload. This level MUST NOT appear in the \code{alerts} array of responses with an HTTP response code that is not between 200 and 399
	(inclusive).
	\item \code{warning} - This level is used to warn clients of potentially dangerous conditions when said conditions have not caused a request
	to fail. The best example of this is deprecation warnings, which should appear on all API routes that have been deprecated.
\end{itemize}

\subsubsection{Summary}
The \code{summary} object is used to provide summary statistics about object
collections. In general the use of \code{summary} is left to be defined by API
endpoints (subject to some restrictions). However, its use is not appropriate in
cases where the user is specifically requesting summary statistics, but should
rather be used to provide supporting information - pre-calculated - about a set
of objects or data that the client \emph{has} requested.\\
Endpoints MUST use the following, reserved properties of \code{summary} for
their described purposes (when use of \code{summary} is appropriate) rather than
defining new \code{summary} or \code{response} properties to suit the same
purpose:

\begin{itemize}
	\item \code{count} Count contains an unsigned integer that defines the total
		number of results that could possibly be returned given the
		non-pagination query parameters supplied by the client (see Section
		\ref{sec:pagination}).
\end{itemize}

\subsection{HTTP Request Methods}
\href{https://tools.ietf.org/html/rfc2616#section-9}{RFC 2616 Section 9} defines the semantics of HTTP/1.1 request methods. Authors should conform
to that set of standards whenever possible, but for convenience the methods recognized by Traffic Ops and their meanings in that context are
herein defined.

\subsubsection{GET}
HTTP GET requests are issued by clients who want some data in response. In the context of Traffic Ops, this generally means a serialized
representation of some object. GET requests MUST NOT alter the state of the server. An example of an API endpoint created in API version 1
that violates this restriction is \code{cdns/name/\{\{name\}\}/dnsseckeys/delete}.\\
This is the standard method to be used by all read-only operations, and as such handlers for this method should generally be accessible to
users with the "read-only" Role.\\
All endpoints dealing with the manipulation or fetching representations of Data Model objects as defined in Section \ref{sec:data-model}
MUST support this method.

\subsubsection{POST}
POST requests ask the server to process some provided data. Most commonly, in Traffic Ops, this means creating an object based on the serialization
of said object contained in the request body, but it can also be used virtually whenever no other method is appropriate. When an object \emph{is}
created, the response body MUST contain a representation of the newly created object.\\
POST requests do not need to be \emph{idempotent}, unlike PUT requests.

\subsubsection{PUT}
PUT is used to replace existing data with new data that is provided in the request body.
\href{https://tools.ietf.org/html/rfc2616#section-9.1.2}{RFC 2616 Section 9.1.2} lists PUT as an "idempotent" request method, which means that
subsequent identical requests should ensure the same state is maintained on the server. What this means is that a client that PUTs an object
representation to Traffic Ops expects that if they then GET a representation of that object, do the same PUT again and GET another representation,
the two retrieved representations should be identical. Effectively, the \code{last\_updated} field that is common to objects in the Traffic Ops API
violates this, but the other properties of objects - which can actually be defined - generally obey this restriction. In general, fulfilling this
restriction means that handlers will need to require the entirety of an object be defined in the request body.\\
When an object is modified, the response body MUST contain a representation of
the object after modification.\\
While RFC 2616 states that servers MAY create objects for the passed representations if they do not already exist, Traffic Ops API endpoint authors
MUST instead use POST handlers for object creation.\\
All endpoints that support the PUT request method MUST also support the
If-Not-Modified-Since HTTP header.

\subsubsection{PATCH}
At the time of this writing, no Traffic Ops API endpoints handle the PATCH request method. PATCH requests that the server's stored data be mutated
in some way using data provided in the request body. Unlike PUT, PATCH is not \emph{idempotent}, which essentially means that it can be used to
change only part of a stored object. When an object is modified, the response body MUST contain a representation of the object after modification,
and that representation SHOULD fully describe the modified object, even the parts that were not modified.\\
Handlers that implement PATCH in the Traffic Ops API MUST use conditional
requests to ensure that race conditions are not a problem, specifically they
MUST support using ETag and If-Match, and SHOULD also support
If-Not-Modified-Since.\\
All Traffic Ops API endpoints that support PUT MUST also support PATCH.
Clients SHOULD use PATCH requests rather than PUT requests for modifying existing resources whenever it is supported.

\subsubsection{DELETE}
DELETE destroys an object stored on the server. Typically the request will contain identifying information for the object(s) to be destroyed either
in the request URI or in the request's body. Traffic Ops API endpoint authors MUST use this request method whenever an object identified by the
request URI is being destroyed. When such deletion successfully occurs, the response body MUST contain a representation of the destroyed object.

\subsection{HTTP Response Codes}
Proper use of HTTP response codes can significantly improve user interfaces built on top of the API. What follows is a (non-exhaustive) set of
response codes and their appropriate use in the context of Traffic Ops. For more complete information, refer to
\href{https://developer.mozilla.org/en-US/docs/Web/HTTP/Status}{the Mozilla Developer Network's HTTP Response Code list}.

\subsubsection{200 OK}
This indicates the request succeeded, with no additional semantics. This MUST be the exact response status code of successful GET requests. This is
also the default "success" response code for any other request, and is technically permissible even when more specific options exist.

\subsubsection{201 Created}
This indicates that a resource was successfully created on the server. This SHOULD be the response status code of POST requests that create a new
object or objects on the server, and in that case the response should also include a Location header that provides a URI where a representation of
the newly created object may be requested.

\subsubsection{202 Accepted}
\code{202 Accepted} SHOULD be used when the server is performing some task asynchronously (e.g. refreshing DNSSEC keys) but the status of that task
cannot be ascertained at the current time. Ideally in this case, when the task completes - either successfully or by failing - the Traffic Ops changelog
should be updated to indicate that status, along with information to uniquely identify the task (e.g. username and date/time when the task started).

\subsubsection{400 Bad Request\label{sec:400}}
In general this is used when there's something syntactically wrong with the client's request. For example, Traffic Ops MUST respond with this code
when the request body was improperly encoded. In most cases, this is also the proper response code when the client submits data that is not semantically
correct. For example, dates/times represented as timestamp strings in an unsupported format should trigger this response code.\\
This is also the default "client failure" response code for any other request, and is technically permissible even when more specific options exist.\\
The response body MUST include an entry in the \code{alerts} array that describes to the client what was wrong with the request.

\subsubsection{401 Unauthorized\label{sec:401}}
This MUST be the response code when a client without valid authorization information in the HTTP headers (i.e. a "mojolicious" cookie) requests a resource
which cannot be accessed without first authorizing. Which should be everything except \code{/ping} and endpoints that provide authorization.

\subsubsection{403 Forbidden\label{sec:403}}
This SHOULD be used whenever the client is logged-in, but still does not have access to the resource they are requesting. It MAY also be used when they
have \emph{some} access to the resource, but not with the specific request method they used (the other acceptable option is \code{405 Method Not Allowed},
in which case the Accept HTTP header should also be sent enumerating the allowed methods).\\
This can pertain to restricted access on the basis of Role, User Capability/Permissions, as well as Tenancy.\\
The response body SHOULD contain an Alert describing on what basis access was denied.

\subsubsection{404 Not Found\label{sec:404}}
This MUST be the returned status code when the client requests a path that does not exist on the server. This also MAY be returned when the user provides
identifying information for a requested object that does not actually identify an object. Finally, endpoints MAY choose to use this instead of
\code{403 Forbidden} when a specifically requested object is forbidden to the client based on their Role, User Capabilities/Permissions and/or Tenancy, in
order to hide the existence of such objects from the client. However, note that this is not entirely effective and can be detrimental to the user experience,
and so authors SHOULD use \code{403 Forbidden} when possible in that case.

\subsubsection{409 Conflict}
This SHOULD be used when the request cannot be completed because the current state of the server is fundamentally incompatible with the request. For example,
creating a new user with an email that is already in use should result in this response.\\
Additionally, this MAY be used instead of \code{404 Not Found} when the client is requesting a link between an object identified by the request URI and some
other object (e.g. when assigning a server to a Delivery Service) when the other object does not exist. If the request URI identifies an object that does not
exist, the response MUST use \code{404 Not Found} instead.\\
This is also the proper response status code when the conditions of a request cannot be met, e.g. when a client submits a PATCH request for a resource with an
If-Match header that does not match the stored object's ETag.\\
The request body MUST indicate what the conflict is that prevented the request from being fulfilled.

\subsubsection{500 Internal Server Error}
When the Traffic Ops server encounters some error - through no fault of the client or their request - that renders it incapable of servicing the client's
request, it MUST return this status code if no other code is more appropriate. The response body in this case SHOULD indicate that an error occurred, but
MUST NOT divulge details about what data was being processed, what (if any) other components are not functioning properly, or what process failed. Generally
it is advisable that the resultant \code{alerts} array entry just say "Internal Server Error" and nothing else.

\subsubsection{501 Not Implemented}
This is the response code used when the client requests an API version not implemented by the server. It SHOULD NOT be used in any other case.

\subsubsection{502 Bad Gateway}
This code indicates that some other service on which the endpoint's processes depend has given back improper data or an error response. It MAY be used
(with caution) by plugin developers, but SHOULD NOT be used by authors of proper API endpoints, as that divulges information about failing connected systems
and potentially gives an attacker information about Traffic Control's weak points. API endpoint authors should instead use \code{500 Internal Server Error}.

\subsubsection{504 Gateway Timeout}
This code indicates that a connection timeout occurred when attempting to contact some other service on which the endpoint's processes depend. It MAY be used
(with caution) by plugin developers, but SHOULD NOT be used by authors of proper API endpoints, as that divulges information about failing connected systems
and potentially gives an attacker information about Traffic Control's weak points. API endpoint authors should instead use \code{500 Internal Server Error}.

\subsection{Content Negotiation}
The default Content-Type for all API responses should be \code{application/json} whenever content-negotiation does not occur. It's not
necessary to specify the character encoding because all valid JSON is UTF-8-encoded. If the client requests data from Traffic Ops in a
format unsatisfiable by \code{application/json}, the response MAY address the issue in a Warning-level Alert, but that is not
necessary. In this case, Traffic Ops MUST NOT respond with \code{406 Not Acceptable} or another error code based solely on content-negotiation.
Instead, an \code{application/json} response should be returned to the client.\\
Similarly, if the request does not define a Content-Type for its request body, then Traffic Ops API endpoints MUST treat the body as though it
were encoded as \code{application/json}, and attempt to parse it as such.

\subsection{Documentation}
All endpoints must be properly documented. For guidelines for writing API documentation, refer to
\href{https://traffic-control-cdn.readthedocs.io/en/latest/development/documentation_guidelines.html#documenting-api-routes}{that section of the official ATC documentation}.

\subsection{Passing Request Data}
Request data may be passed in the request body or as a\\ \code{application/x-www-form-urlencoded}-encoded query string in the request URI, or as a part of the
request path. Request data MUST NOT be passed through a portion of the request path unless it uniquely identifies a resource with which the client may
interact. For example, \code{/foos/\{\{ID\}\}} is an acceptable path for dealing with the particular "Foo" object that has some identifier \code{ID} (subject
to the restrictions in Section \ref{sec:duplicate-endpoints}), but \code{logs/\{\{Number of Days\}\}/days} is unacceptable because reasonable default behavior can
be provided if no number of days is given in the query string parameters, and that doesn't help uniquely identify a resource. Request path parameters should
use double "curly-braces" (\code{\{} and \code{\}}) to call out variable components of the request path in documentation and references. Request path parameters
MUST NOT be used for data that is optional to the request (somewhat obviously). Note that all endpoints dealing with the manipulation of Data Model objects as
described in Section \ref{sec:data-model} MUST support the GET HTTP request method.\\
The decision to pass data in the request body or query string is mainly up to the author, but some helpful tips:

\begin{itemize}
	\item GET and DELETE requests do not typically provide request bodies.
	\item Query parameters should nearly always be optional. If data is required by an endpoint, consider requiring it in the request body. If the data identifies a resource, it ought to be a path parameter.
	\item Request body data often represents objects that are being created or updated. If an object is being created or updated, it ought to be defined in the
	request body, and if any additional data is (possibly optionally) required then it ought to be passed in the query string to separate it from the object definition.
	\item The following query parameters are reserved for special use by Traffic Ops endpoint handlers, and may not be used for any purpose other
	than their prescribed functions.
	\begin{itemize}
		\item \code{limit}
		\item \code{newer\_than}
		\item \code{offset}
		\item \code{older\_than}
		\item \code{page\_size}
		\item \code{page}
		\item \code{reverse}
		\item \code{sort\_by}
	\end{itemize}
\end{itemize}

\subsection{Identifiers}
All objects manipulated by the Traffic Ops API MUST have a property that can, by itself, uniquely identify objects of that type. The best case
scenario is that the identifying property is a human-friendly name of some kind, e.g. username for users. It may be possible to identify an
object uniquely by another property, or by some combination of other properties, but it is necessary to choose exactly one property to expressly
serve the purpose of unique identification (ideally in addition to providing useful data).\\
Authors are encouraged to avoid numeric identifiers that serve no purpose beyond identification, whenever possible. In some cases, though, this
is not possible.

\subsection{Duplicate Endpoints\label{sec:duplicate-endpoints}}
No two endpoints should serve the same purpose. While it's fine to overlap a bit, an endpoint like \code{/foo\_bars} should not exist solely to
edit the Bars property of Foo objects (which can ostensibly be edited just fine on the object itself), for example. Ideally, there should be
exactly one way to accomplish something through the API.\\
A caveat, though, is object relationships. For example, a Delivery Service has zero or more Cache Servers assigned to it, and in turn Cache
Servers may be assigned to zero or more Delivery Services\footnote{Rails developers may be familiar with this as the concept of a
"has-and-belongs-to-many" relationship.}. Thus it is permissible to be able to edit the Delivery Services property of a Cache Server using the
\code{/cache\_servers} API endpoint as well as to be able to edit the Cache Servers property of a Delivery Service using the
\code{/delivery\_services} API endpoint - though they arguably provide equivalent functionality in that way.

\subsection{Date/Time Format\label{sec:datetime}}
Dates MUST be represented in either \href{https://tools.ietf.org/html/rfc3339}{RFC3339} format (with or without nanosecond precision) or as
integers indicating the number of nanoseconds past the Unix epoch at which the date/time occurs. In either case, Dates included in responses
from Traffic Ops MUST be in UTC. Wherever date/times are accepted as input, Traffic Ops API endpoints MUST accept either format.\\
Traffic Ops endpoints MUST return dates and times in \href{https://tools.ietf.org/html/rfc3339}{RFC3339} format with nanosecond precision.
Endpoint MAY provide ways for the client to specify alternate representations, but these SHOULD be restricted to only Unix epoch timestamps
in nanoseconds.

\subsection{Pagination and Sorting\label{sec:pagination}}
All endpoints that involve retrieving representations of objects (i.e. GET requests for collection objects e.g. Cache Servers) MUST implement
sorting and pagination controls. Sorting controls SHOULD be capable of sorting results in either ascending or descending order by any property
of the objects the endpoint returns.\\
Pagination and sorting controls MUST be made available by the query parameters in Table \ref{tbl:pag-and-sort-qparams}, and SHOULD NOT be made
available by other means.

\begin{table}[h]
%\centering
\caption{Sorting and Pagination Query Parameters\label{tbl:pag-and-sort-qparams}}
\begin{tabular}{|l|l|}
\hline
\textbf{Parameter} & \textbf{Meaning}\\
\hline
\code{page\_size} & An unsigned integer that defines the number of objects that\\
                  & will appear in a single "page"\\
\hline
\code{page}       & An unsigned integer that defines the page that will be returned.\\
                  & The minimum value is "1"\\
\hline
\code{offset}     & An unsigned integer that defines an offset from the first result\\
                  & at which the returned results will start. The minimum value is\\
                  & "0".\\
\hline
\code{limit}      & An unsigned integer that sets an upper limit on the number of\\
                  & results returned. The minimum value is "0", which means\\
                  & "no limit"\\
\hline
\code{reverse}    & A boolean value which MUST be recognized as "truthy" by\\
                  & having the value \code{true}, but endpoints MAY also consider \code{1},\\
                  & \code{yes}, and case-insensitive variants of those values as "truthy"\\
\hline
\code{sort\_by}   & An arbitrary string which MUST be the name of a property of\\
                  & the returned objects by which results will be sorted\\
\hline
\end{tabular}
\end{table}

Endpoints MAY return errors when a client request gives these parameters improper or invalid values, but MUST at least provide a warning.
When ambiguity or errors in pagination controls render pagination, sorting, offsetting and/or limiting impossible, the handler MUST NOT
perform \emph{\textbf{any}} of those operations.\\
When limiting, pagination, or offsetting is performed, the response MUST include
the \code{count} property of the response's \code{summary} property, indicating
the total number of available objects - satisfying all object-property filtering
query parameters given - that could be returned.\\
Endpoints MUST accept any top-level property that appears on the serialized objects it returns as a valid value for \code{sort\_by},
\emph{except} for properties that are themselves collections (array, set, object etc.). Such values MUST be acceptable in the same "case"
as they appear in the serialization, and endpoints SHOULD NOT accept case transformation or aliases for properties.\\

\subsubsection{Multiple Sorting}
Traffic Ops API endpoints MUST accept single values for \code{sort\_by}, but MAY also accept a priority list of properties by which to sort.
If implemented, this list MUST be recognized as comma-separated values of the \code{sort\_by} parameter. Multiple of the same value appearing
in the definition of \code{sort\_by} constitutes an ambiguity and must be handled accordingly.\\
For example, suppose the fictional \code{/foos} endpoints returns Foo objects. Given the client requests this endpoint with
\code{GET /api/3.0/foos HTTP/1.1}, suppose the response resembles Listing \ref{code:unsorted-foos}.\\
To sort this response first by the \code{fizz} property and then by the \code{test} property, the client might make the request\\
\code{GET /api/3.0/foos?sort\_by=fizz,test HTTP/1.1}. The order in which the properties appear defines their respective
priorities. The response to this request, then, will be as shown in Listing \ref{code:sorted-foos}.

\begin{codelisting}
\captionof{listing}{Unsorted Foos Response}
\label{code:unsorted-foos}
\begin{minted}[tabsize=2]{http}
HTTP/1.1 200 OK
Content-Type: application/json
Transfer-Encoding: chunked

{
	"response": [
		{
			"fizz": 1,
			"test": 1
		},
		{
			"fizz": 0,
			"test": 1
		},
		{
			"fizz": 0,
			"test": 0
		}
	]
}
\end{minted}
\end{codelisting}

\newpage{}

\begin{codelisting}
\captionof{listing}{Sorted Foos Response}
\label{code:sorted-foos}
\begin{minted}[tabsize=2]{http}
HTTP/1.1 200 OK
Content-Type: application/json
Transfer-Encoding: chunked

{
	"response": [
		{
			"fizz": 0,
			"test": 0,
		},
		{
			"fizz": 0,
			"test": 1
		},
		{
			"fizz": 1,
			"test": 1
		}
	]
}
\end{minted}
\end{codelisting}

\subsection{Age Filtering}
Whenever object age is a property of that object (which is quite often in the form of \code{lastModified}), Traffic Ops endpoint
handlers that respond to requests for object representations (i.e. GET requests) SHOULD support filtering by age. If age filtering
is implemented, it MUST be made available using the query parameters in Table \ref{tbl:age-filtering-qparams}.

\begin{table}[h]
\centering
\caption{Age Filtering Query Parameters}
\label{tbl:age-filtering-qparams}
\begin{tabular}{|l|l|}
\hline
\textbf{Parameter} & \textbf{Meaning}\\
\hline
\code{newer\_than} & A timestamp to be used as the lower limit on an object's age.\\
                   & Objects older than this MUST NOT appear in the response\\
                   & body.\\
\hline
\code{older\_than} & A timestamp to be used as the upper limit an object's age.\\
                   & Objects newer than this MUST NOT appear in the response\\
                   & body.\\
\hline
\end{tabular}
\end{table}

The format of these timestamps - in accordance with Section \ref{sec:datetime} of this document - MUST be accepted as Unix epoch timestamps
in nanoseconds, AND in the form of RFC3339 date/time strings.\\
Endpoints MAY return errors when a client request gives these parameters improper or invalid values, but MUST at least provide a warning.
When ambiguity or errors in age filtering controls render age filtering impossible, the handler MUST NOT perform age filtering.

\subsection{Tenancy\label{sec:tenancy}}
When a client requests access to a set of stored objects that are "tenantable" inevitably some of them will be inaccessible to the user
on the basis of their Tenant. Traffic Ops endpoint handlers that respond to requests for such object representations (i.e. GET requests)
SHOULD filter their results implicitly according to the requesting Tenant's access. Any request that would modify, create, or destroy an
object to which the requesting Tenant has access MUST NOT be fulfilled by the server (obviously) and in that case the response status code
SHOULD be \code{403 Forbidden} as described in Section \ref{sec:403}, but MAY also be \code{404 Not Found} as described in Section \ref{sec:404}
or \code{400 Bad Request} as described in \ref{sec:400} depending on the circumstances.

\subsection{Naming Conventions}
The names of properties of objects as they appear in said objects' serializations ought to conform to "camelCase" naming. Initialisms,
abbreviations, and acronyms that appear in property names should be capitalized unless they are at the very beginning of the name. For
example, \code{myIPAddress} and \code{someProperty} are both well-formed property names, while \code{IPAddress}, \code{someproperty} and
\code{SomeProperty} are not.\\
Query parameters that follow "camelCase" naming are assumed to be filters that act on the properties of objects, so query parameters with
any other function shouldn't use "camelCase". Instead, query parameters with special functions should use "snake\_case" to distinguish them.\\
API endpoints themselves should have a name that conveys their purpose. For example, \code{/cdns} is an endpoint that deals with manipulating,
creating, destroying, or retrieving representations of CDNs. Request paths should use "snake\_case" to separate words whenever necessary, and
should never include the action being performed by the handler; instead that is decided by the request \emph{method}. For example,
\code{/myObject/delete} is a poor request path name for both of those reasons. Furthermore, when an endpoint deals with an object type of
which there are typically multiple, the request path should be plural, e.g. \code{/cdns} is better than \code{/cdn}.\\
API endpoints MAY support trailing slashes (\code{/}) in the request path, but MUST NOT include suffixes that indicate a particular endcoding
("file extensions"); that's what the Content-Type header is for. For example, in API version 1.x, \code{/foos} and \code{/foos.json} are both
equally valid ways to access the \code{/foos} endpoint handlers - this is no longer allowed!

\subsection{Relationships as Objects}
Relationships SHOULD NOT be represented through the API as objects in their own right. For example, instead of an endpoint like
\code{/delivery\_service\_servers} used to manipulate assignments of Cache Servers to Delivery Services, a Delivery Service itself
has Servers as a property. Thus assignments are manipulated by manipulating that property. So the only endpoints necessary for fully
defining and dealing with such relationships are \code{/delivery\_services} and \code{/servers}.
