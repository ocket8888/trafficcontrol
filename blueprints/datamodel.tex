% Licensed to the Apache Software Foundation (ASF) under one
% or more contributor license agreements.  See the NOTICE file
% distributed with this work for additional information
% regarding copyright ownership.  The ASF licenses this file
% to you under the Apache License, Version 2.0 (the
% "License"); you may not use this file except in compliance
% with the License.  You may obtain a copy of the License at
%
%   http://www.apache.org/licenses/LICENSE-2.0
%
% Unless required by applicable law or agreed to in writing,
% software distributed under the License is distributed on an
% "AS IS" BASIS, WITHOUT WARRANTIES OR CONDITIONS OF ANY
% KIND, either express or implied.  See the License for the
% specific language governing permissions and limitations
% under the License.

\section{Data Model\label{sec:data-model}}
This section defines the various objects that may be interacted with through the
Traffic Ops API and their respective properties.\\
An object's properties may principally be of several different types:

\begin{itemize}
	\item Integers. Integers may be signed or unsigned, to signify whether or not
	they are capable of being less than zero. Representation formats which are
	incapable of distinguishing integral and floating-point numbers (e.g. JSON)
	or of separately describing signed and unsigned types (e.g. JSON again) MUST
	represent these as numbers. In particular, they MUST NOT be represented as
	string types. Furthermore, they MUST NOT be represented with a non-zero
	decimal component and generally SHOULD NOT be represented with a decimal part
	at all, if possible. Unsigned quantities MUST NOT include a sign (even
	\code{+}, were it supported by the format).

	\item Floating-point numbers. Floating point numbers are guaranteed to have a
	precision of at least 64-bits, but clients SHOULD NOT treat representations
	with a greater apparent precision as if they were more precise than 64 bits
	can offer. Endpoints MUST NOT implement Not-a-Number nor either positive or
	negative Infinity values as floating-point numbers, even if the
	representation format supports them.

	\item Strings. Strings SHOULD always be encoded (along with the rest of the
	representation, if possible) in UTF-8.

	\item Dates/times. As discussed in section \ref{sec:datetime}, these may be
	expressed as either strings in RFC3339 format or as a signed integer number
	of nanoseconds since the Unix epoch if the representation format does not
	support date/time types natively (e.g. JSON).

	\item Booleans. Boolean concepts MUST be represented as boolean data types in
	representations that do support them. For example, in JSON,
	\code{\{"foo":true\}} is acceptable but \code{\{"foo":"true"\}},
	\code{\{"foo":1\}} and especially \code{\{"foo":"1"\}} are all unacceptable.
	In representation formats that do not support boolean types, they may be
	represented as the number zero for "false" and the number one for "true" as
	unsigned integers.

	\item Collections. There are several collection types used to model ATC data. Commonly, they all may contain arbitrary data types, but SHOULD NOT contain data of more than one type in any representation format.

	\begin{itemize}
		\item Sets. A set is an unordered collection of unique items. The
		contents of a set are not guaranteed to be in any particular order, and
		clients SHOULD NOT expect anything else\footnote{Endpoint authors are,
		however, encouraged to order these consistently for ease of use, testing,
		and debugging. This order need not - and indeed SHOULD NOT be documented.}
		(unless the endpoint provides sorting functionality expressly), and in
		particular SHOULD NOT conclude that a difference in ordering indicates
		that data has changed. If a representation format is not capable of
		expressing set types (e.g. JSON), then sets may be expressed as a list
		array or other more generic collection, but MUST NEVER contain duplicates.

		\item Lists/Arrays. A list/array is an \emph{ordered} collection of data,
		and as such the ordering MUST be consistent in representations, as well as
		meaningful.

		\item Maps. A map may be used to associate any non-collection type "key"
		with a "value" of arbitrary type. Maps MUST NOT contain duplicate keys in
		any representation format. If a representation format is not capable of
		representing maps (e.g. JSON), then they may be represented as objects
		where the keys are coerced to property names and their values the
		respective values of those properties.
	\end{itemize}
\end{itemize}

With the "fallback" behavior described for each type above, note that valid
representation formats MUST be capable of representing at least floating-point
values, string values, array/lists of such primitive values, and either nested
object structures or maps.

\subsection{CDNs}
A CDN is a full collection of caching and infrastructural servers and ATC
components required to service distributed requests for content, as well as the
configuration options for the methods of content delivery and the content
delivered therein.

\subsubsection{Cache Servers}
A CDN's collection of Cache Servers is represented as set of Cache Server IDs.

\subsubsection{Delivery Services}
A CDN's collection of Delivery Services is represented as a set of Delivery
Service Names.

\subsubsection{DNSSEC Enabled}
A CDN that has "DNSSEC Enabled" is capable of supporting DNSSEC configurations on
its constituent Delivery Services. This does \emph{not} mean that all DNS
requests through ATC for this CDN's Domain will be secured with DNSSEC.\\
This is a boolean concept and as such is represented by boolean values.

\subsubsection{DNSSEC Keys}
% TODO: is this necessary? or is this more of a Delivery Service thing?
An array/list of the Zone-Signing and Key-Signing Keys of this CDN. The
Zone-Signing Key will appear first, followed immediately by the Key-Signing Key

\subsubsection{Domain}
The Top-Level Domain used by the CDN, represented as a string. This MUST be a
uniquely valued property across all CDNs, and MUST not be an invalid DNS label as
defined by
\href{https://tools.ietf.org/html/rfc1035#section-2.3.1}{Section 2.3.1 of RFC1035}.

\subsubsection{Infrastructure Servers}
A CDN's collection of Infrastructure Servers is represented as a set of
Infrastructure Server IDs.

\subsubsection{Name}
A CDN's "Name" is a string that uniquely identifies it among all CDNs. It MUST
NOT be allowed to contain "special characters" - meaning anything other than
alphanumeric characters, spaces, hyphens, colons and underscores. It further MUST
NOT be allowed to begin with a non-alphanumeric character nor end with a space.

\subsubsection{Origins}
A CDN's collection of Origins is represented by a set of Origin IDs.

\subsubsection{Traffic Monitors}
A CDN's collection of Traffic Monitors is represented by a set of Traffic Monitor
IDs. Note that while a functioning CDN must contain at least one Traffic Monitor,
no such restriction is placed on CDN objects in general.

\subsubsection{Traffic Portals}
A CDN's collection of Traffic Portal is represented by a set of Traffic Portal
ID's.

\subsubsection{Traffic Routers}
A CDN's collection of Traffic Routers is represented by a set of Traffic Router
IDs. Note that while a functioning CDN must contain at least one Traffic Router,
no such restriction is placed on CDN objects in general.

\subsubsection{Traffic Stats Servers}
A CDN's collection of Traffic Stats Servers is represented by a set of Traffic
Stats Server IDs.

\subsubsection{Traffic Vaults}
A CDN's collection of Traffic Vaults is represented by a set of Traffic Vault IDs.



\subsection{Cache Groups}



\subsection{Cache Servers}

\subsubsection{Cache Group}

\subsubsection{Capabilities}

\subsubsection{CDN}

\subsubsection{Delivery Services}

\subsubsection{Domain}

\subsubsection{Host Name}

\subsubsection{HTTP Port}

\subsubsection{HTTPS Port}

\subsubsection{ID}

\subsubsection{IPv4 Address}

\subsubsection{IPv6 Address}

\subsubsection{Notes}

\subsubsection{Physical Location}

\subsubsection{Profile}

\subsubsection{Revalidation Pending}

\subsubsection{Status}

\subsubsection{Tags}

\subsubsection{Type}

\subsubsection{Updates Pending}



\subsection{Delivery Services}
Delivery Services are, at their most basic, an association between a source of
content and a set of Cache Servers and configuration options used to distribute
that content.

\subsubsection{Active}
A Delivery Service that is "active" is one that is functionally in service, and
capable of delivering content. This is a boolean concept and is represented by
boolean values. A Delivery Service cannot be Active if it is not Primed.

\subsubsection{Primed}
A Delivery Service that is "primed" has had its configuration distributed to the
various servers required to serve its content. However, the content itself is
still inaccessible\footnote{The content is not available through normal routing.
This does not, though, guarantee that Cache Servers do not already have the
content stored and/or are incapable of serving it if routing is bypassed.}. A
Delivery Service that is Active is necessarily Primed.



\subsection{Infrastructure Servers}



\subsection{Origins}



\subsection{Profiles and Parameters}



\subsection{Physical Locations}



\subsection{Roles and Permissions\label{sec:roles-and-perms}}



\subsection{Tags}



\subsection{Tenants}



\subsection{Traffic Monitors}



\subsection{Traffic Portals}



\subsection{Traffic Routers}



\subsection{Traffic Stats Servers}



\subsection{Traffic Vaults}



\subsection{Users\label{sec:users}}
