% Licensed to the Apache Software Foundation (ASF) under one
% or more contributor license agreements.  See the NOTICE file
% distributed with this work for additional information
% regarding copyright ownership.  The ASF licenses this file
% to you under the Apache License, Version 2.0 (the
% "License"); you may not use this file except in compliance
% with the License.  You may obtain a copy of the License at
%
%   http://www.apache.org/licenses/LICENSE-2.0
%
% Unless required by applicable law or agreed to in writing,
% software distributed under the License is distributed on an
% "AS IS" BASIS, WITHOUT WARRANTIES OR CONDITIONS OF ANY
% KIND, either express or implied.  See the License for the
% specific language governing permissions and limitations
% under the License.

\section{Data Model\label{sec:data-model}}
This section defines the various objects that may be interacted with through the
Traffic Ops API and their respective properties.\\
An object's properties may principally be of several different types:

\begin{itemize}
	\item Integers. Integers may be signed or unsigned, to signify whether or
	not they are capable of being less than zero. Representation formats which
	are incapable of distinguishing integral and floating-point numbers (e.g.
	JSON) or of separately describing signed and unsigned types (e.g. JSON
	again) MUST represent these as numbers. In particular, they MUST NOT be
	represented as string types. Furthermore, they MUST NOT be represented with
	a non-zero decimal component and generally SHOULD NOT be represented with a
	decimal part at all, if possible. Unsigned quantities MUST NOT include a
	sign (even \code{+}, were it supported by the format).

	\item Floating-point numbers. Floating point numbers are guaranteed to have
	a precision of at least 64-bits, but clients SHOULD NOT treat
	representations with a greater apparent precision as if they were more
	precise than 64 bits can offer. Endpoints MUST NOT implement Not-a-Number
	nor either positive or negative Infinity values as floating-point numbers,
	even if the representation format supports them.

	\item Strings. Strings SHOULD always be encoded (along with the rest of the
	representation, if possible) in UTF-8.

	\item Dates/times. As discussed in section \ref{sec:datetime}, these may be
	expressed as either strings in RFC3339 format or as a signed integer number
	of nanoseconds since the Unix epoch if the representation format does not
	support date/time types natively (e.g. JSON).

	\item Booleans. Boolean concepts MUST be represented as boolean data types
	in representations that do support them. For example, in JSON,
	\code{\{"foo":true\}} is acceptable but \code{\{"foo":"true"\}},
	\code{\{"foo":1\}} and especially \code{\{"foo":"1"\}} are all unacceptable.
	In representation formats that do not support boolean types, they may be
	represented as the number zero for "false" and the number one for "true" as
	unsigned integers.

	\item Collections. There are several collection types used to model ATC
	data. Commonly, they all may contain arbitrary data types, but SHOULD NOT
	contain data of more than one type in any representation format.

	\begin{itemize}
		\item Sets. A set is an unordered collection of unique items. The
		contents of a set are not guaranteed to be in any particular order, and
		clients SHOULD NOT expect anything else\footnote{Endpoint authors are,
		however, encouraged to order these consistently for ease of use,
		testing, and debugging. This order need not - and indeed SHOULD NOT be
		documented.} (unless the endpoint provides sorting functionality
		expressly), and in particular SHOULD NOT conclude that a difference in
		ordering indicates that data has changed. If a representation format is
		not capable of expressing set types (e.g. JSON), then sets may be
		expressed as a list array or other more generic collection, but MUST
		NEVER contain duplicates.

		\item Lists/Arrays. A list/array is an \emph{ordered} collection of
		data, and as such the ordering MUST be consistent in representations, as
		well as meaningful.

		\item Maps. A map may be used to associate any non-collection type "key"
		with a "value" of arbitrary type. Maps MUST NOT contain duplicate keys
		in any representation format. If a representation format is not capable
		of representing maps (e.g. JSON), then they may be represented as
		objects where the keys are coerced to property names and their values
		the respective values of those properties.
	\end{itemize}
\end{itemize}

With the "fallback" behavior described for each type above, note that valid
representation formats MUST be capable of representing at least floating-point
values, string values, array/lists of such primitive values, and either nested
object structures or maps.

% Licensed to the Apache Software Foundation (ASF) under one
% or more contributor license agreements.  See the NOTICE file
% distributed with this work for additional information
% regarding copyright ownership.  The ASF licenses this file
% to you under the Apache License, Version 2.0 (the
% "License"); you may not use this file except in compliance
% with the License.  You may obtain a copy of the License at
%
%   http://www.apache.org/licenses/LICENSE-2.0
%
% Unless required by applicable law or agreed to in writing,
% software distributed under the License is distributed on an
% "AS IS" BASIS, WITHOUT WARRANTIES OR CONDITIONS OF ANY
% KIND, either express or implied.  See the License for the
% specific language governing permissions and limitations
% under the License.

\subsection{\code{/cdns}}
This endpoint deals with manipulations and representations of the group of CDN
objects configured in Traffic Ops.

\subsubsection{GET}
Retrieves CDN representations.
\begin{description}
	\item[Required Permissions] \code{cdns-read}
	\item[Response Type] Array
\end{description}

\paragraph{Request Structure}
This method of this endpoint implements the Age Filtering and Sorting and
Pagination query parameters as outlined in Sections \ref{sec:age-filtering} and
\ref{sec:pagination}, respectively. It further provides the query parameters in
Table \ref{tbl:cdns:get:qparams}

\begin{table}[h]
	\centering
	\caption{GET \code{/cdns} Query Parameters\label{tbl:cdns:get:qparams}}
	\begin{tabularx}{\linewidth}{|l|X|}
		\hline
		\textbf{Parameter} & \textbf{Description}\\
		\hline
		dnssecEnabled & Filters results to only contain CDNs with this DNSSEC
			Enabled setting - must represent a boolean (\code{true} or
			\code{false}). If this cannot parse to a boolean, it will cause the
			endpoint to ignore the filter and emit a warning about the bad
			value for this parameter.\\
		\hline
		domain & Filters results to only contain CDNs that have this Domain\\
		\hline
		name & Filters results to only contain CDNs that have this Name\\
		\hline
		omit & This can be either a single property name (e.g.
			\code{omit=cacheServers}) or a comma-delimited array of property
			names (e.g. \code{omit=trafficMonitors,trafficRouters}). The named
			propert(y/ies) will be stripped from the output objects; that is,
			the returned objects will be missing these properties from their
			representations. The properties that can be omitted in this way
			are:
			\begin{itemize}
				\item Cache Servers
				\item Delivery Services
				\item Infrastructure Servers
				\item Origins
				\item Traffic Monitors
				\item Traffic Routers
				\item Traffic Stats Servers
			\end{itemize}
			If this parameter is provided and cannot be parsed as either a
			single property name or a comma-delimited list thereof, or if an
			attempt is made to omit a property other than those allowable, a
			400 Bad Request error response will be returned.\\
		\hline
	\end{tabularx}
\end{table}

\begin{codelisting}
\captionof{listing}{Request Example}
\begin{minted}{http}
GET /api/4.0/cdns?name=CDN-in-a-Box HTTP/1.1
Host: trafficops.infra.ciab.test
Accept: application/json, */*;q=0.9
Cookie: mojolicious=...

\end{minted}
\end{codelisting}

\paragraph{Response Structure}
The response is an array of representations of CDN objects, each representation
extended with the \code{lastUpdated} property containing the Date/Time at which
the CDN object was last modified.\\
This method of this endpoint also implements the \code{count} property of the
top-level \code{summary} object as described in Section
\ref{sec:summary-object}.

\begin{codelisting}
\captionof{listing}{Response Example}
\begin{minted}[tabsize=2]{http}
HTTP/1.1 200 OK
Content-Type: application/json
Server: Traffic Ops/5.0
Date: Wed, 14 Nov 2018 20:46:57 GMT
Content-Length: 237

{ "response": [
	{
		"cacheServers": [ 1, 2 ],
		"deliveryServices": [ "Demo1" ],
		"dnssecEnabled": false,
		"domain": "mycdn.ciab.test",
		"infrastructureServers": [ 1, 2, 3 ],
		"lastUpdated": "2018-11-14T18:21:14Z",
		"name": "CDN-in-a-Box",
		"origins": [ 1 ],
		"trafficMonitors": [ 1 ],
		"trafficRouters": [ 1 ],
		"trafficStatsServers": [ 1 ]
	}
],
"summary": {
	"count": 2
}}
\end{minted}
\end{codelisting}

\subsubsection{POST}
Creates a new CDN.
\begin{description}
	\item[Required Permissions] \code{cdns-create}
	\item[Response Type] Array
\end{description}

\subsection{\code{/cdns/\{\{CDN Name\}\}}}
This endpoint deals with manipulation and representation of a single CDN object
identified by its Name.\\
For all methods of this endpoint, if the part of the path composed of
\code{CDN Name} is not the name of a CDN that exists in Traffic Ops, a 404 Not
Found error response will be returned in accordance with Section \ref{sec:404}.

\subsection{\code{/cdns/\{\{CDN Name\}\}/snapshot}}

% Licensed to the Apache Software Foundation (ASF) under one
% or more contributor license agreements.  See the NOTICE file
% distributed with this work for additional information
% regarding copyright ownership.  The ASF licenses this file
% to you under the Apache License, Version 2.0 (the
% "License"); you may not use this file except in compliance
% with the License.  You may obtain a copy of the License at
%
%   http://www.apache.org/licenses/LICENSE-2.0
%
% Unless required by applicable law or agreed to in writing,
% software distributed under the License is distributed on an
% "AS IS" BASIS, WITHOUT WARRANTIES OR CONDITIONS OF ANY
% KIND, either express or implied.  See the License for the
% specific language governing permissions and limitations
% under the License.

\subsection{Cache Groups}
A Cache Group is exactly what it sounds like it is: a group of Cache Servers.
Typically a Cache Group is representative of the available Cache Servers within a
specific geographical location. Despite that Cache Servers have their own
Physical Locations, when Cache Servers are chosen to serve content to a client
based on geographic location the geographic location actually used for
comparisons is that for the Cache Group that contains it, not the geographic
location of the Cache Server itself.

\begin{codelisting}
\captionof{listing}{Cache Group Object as a TypeScript Interface}
\begin{minted}[tabsize=2]{typescript}
interface CacheGroup {
	cacheServers: Set<bigint>;
	latitude: number;
	longitude: number;
	name: string;
	tags: Set<string>;
	type: 'EDGE' | 'MID';
}
\end{minted}
\end{codelisting}

\subsubsection{Cache Servers}
All of the Cache Servers within a Cache Group are represented by a set of their
numeric IDs.

\subsubsection{Latitude}
A Cache Group's Latitude is a floating-point number that simply represents a
Cache Group's geographic latitude on the range [-90, 90] - with the positive
range representing North and the negative range representing South.

\subsubsection{Longitude}
A Cache Group's Longitude is a floating-point number that simply represents a
Cache Group's geographic longitude on the range [-180, 180] - with the positive
range representing East and the negative range representing West.

\subsubsection{Name}
A Cache Group is uniquely identified by its Name, which is a string that MUST
NOT be empty.

\subsubsection{Tags}
The Tags associated with a Cache Group is represented by a set of strings that
are Tag Names.

\subsubsection{Type}
Cache Groups can only contain one Type of Cache Server, which is herein
reflected by the string constant value of said Type.

\begin{itemize}
	\item \code{EDGE} - This Cache Group may only contain Cache Servers that
		are of the \code{EDGE} Type.
	\item \code{MID} - This Cache Group may only contain Cache Servers that are
		of the \code{MID} Type.
\end{itemize}

% Licensed to the Apache Software Foundation (ASF) under one
% or more contributor license agreements.  See the NOTICE file
% distributed with this work for additional information
% regarding copyright ownership.  The ASF licenses this file
% to you under the Apache License, Version 2.0 (the
% "License"); you may not use this file except in compliance
% with the License.  You may obtain a copy of the License at
%
%   http://www.apache.org/licenses/LICENSE-2.0
%
% Unless required by applicable law or agreed to in writing,
% software distributed under the License is distributed on an
% "AS IS" BASIS, WITHOUT WARRANTIES OR CONDITIONS OF ANY
% KIND, either express or implied.  See the License for the
% specific language governing permissions and limitations
% under the License.

\subsection{Cache Servers}
Cache servers are the Mid-and-Edge-tier HTTP caching proxies ultimately
responsible for caching and serving content for a Delivery Service.

\begin{codelisting}
\captionof{listing}{Cache Server Object as a TypeScript Interface}
\begin{minted}[tabsize=2]{typescript}
interface IPAddress {
	address: string;
	gateway: string | null;
	serviceAddress: boolean;
}

interface Interface {
	ipAddresses: Array<IPAddress> & {0: IPAddress};
	maxBandwidth: bigint;
	monitor: boolean;
	mtu: bigint | null;
	name: string;
}

interface CacheServer {
	cacheGroup: string;
	capabilities: Set<string>;
	cdn: string;
	domain: string;
	hostName: string;
	httpPort: bigint;
	httpsPort: bigint;
	id: bigint;
	interfaces: Array<Interface> & {0: Interface};
	notes: string;
	physicalLocation: string;
	profile: string;
	revalidationPending: boolean;
	status: "ADMIN_DOWN" | "OFFLINE" | "ONLINE" | "REPORTED";
	tags: Set<string>;
	type: 'EDGE' | 'MID';
	updatePending: boolean;
}
\end{minted}
\end{codelisting}

\subsubsection{Cache Group}
The Cache Group to which the Cache Server belongs is represented as a string
that uniquely names it.

\subsubsection{Capabilities}
A Cache Server's "Capabilities" express a Cache Server's ability to serve
certain kinds of traffic. They are represented on Cache Server objects as a set
of Capability Names.

\subsubsection{CDN}
The CDN to which a Cache Server belongs is represented by its Name, which
uniquely identifies it.

\subsubsection{Domain\label{sec:server:domain}}
This is the "domain" part of the Cache Server's Fully Qualified Domain Name
(FQDN) as a string. For example, a Cache Server with an FQDN of
\code{github.com} has a Domain of \code{com} and a Cache Server with an FQDN of
\code{trafficcontrol.apache.org} has a Domain of \code{apache.org}.\\
In other words, this is every part of the Cache Server's Fully Qualified Domain
Name that isn't part of its Host Name as defined in Section
\ref{sec:server:hostname}.

\subsubsection{Host Name\label{sec:server:hostname}}
This is the "host" part of the Cache Server's Fully Qualified Domain Name (FQDN)
as a string. For example, a Cache Server with an FQDN of \code{github.com} has a
Host Name of \code{github}, and a Cache Server with an FQDN of
\code{trafficcontrol.apache.org} has a Host Name of \code{trafficcontrol}.\\
In other words, a Cache Server's Host Name is the only part of its FQDN that
is not part of its Domain as defined in Section \ref{sec:server:domain}.\\
This field does NOT need to be unique, though operators are encouraged to make
this - or at least the concatenation \code{\emph{Host Name}.\emph{Domain}}
- unique for ease of operation.

\subsubsection{HTTP Port}
An unsigned number that designates the port on which this Cache Server listens
for incoming HTTP requests. Note that this may be zero, which will cause the
Cache Server to be incapable of serving HTTP traffic.

\subsubsection{HTTPS Port}
An unsigned number that designates the port on which this Cache Server listens
for incoming HTTPS requests. Note that this may be zero, which will cause the
Cache Server to be incapable of serving HTTPS traffic.

\subsubsection{ID}
A Cache Server's "ID" is an unsigned integer that uniquely identifies it among
all Cache Servers. It serves no purpose beyond unique identification of the
Cache Server.

\subsubsection{Interfaces}
Interfaces is a set of objects that represent the network interfaces used by a
Cache Server. This set MUST NOT ever be empty. An Interface's properties are
herein listed.

\paragraph{IP Addresses}
A network interface's IP Addresses are represented as a set of objects containing
the various properties of an IP Address. A network interface MUST always have at
least one IP Address object in this set. Their properties are herein listed.\\
Note that there is no structural distinction made between different IP
versions.

\indent{}\subparagraph{Address}
Address is the actual IP Address being described. This may be represented
internally - by either clients or the server - as an actual network address
type, but MUST be represented in API payloads as a string type - even if the
representation encoding supports native network address types.

\indent{}\subparagraph{Gateway}
Gateway is the IP Address of the network gateway used to access this IP
Address. This may be represented internally - by either clients or the server -
as an actual network address type, but MUST be represented in API payloads as a
string type - even if the representation encoding supports native network
address types.\\
This MUST NOT ever be allowed to be represented as an empty string. However, if
it is a "null"-type, then the implication is that no network gateway is used to
access the IP Address.

\indent{}\subparagraph{Service Address}
Service Address is a boolean that describes whether or not an IP Address is one
on which a Cache Server provides its service. Only one such IP Address is
allowed to exist for a given Cache Server per IP version. That is, a Cache
Server may have one IPv4 Service Address and one IPv6 Service Address, but is
not permitted to have two or more IPv4 Service Addresses.

\paragraph{Max Bandwidth}
This unsigned integer describes the maximum bandwidth - in kb/s - that is
allowed for this interface to be considered "healthy" by Traffic Monitors.\\
This has no effect if this Interface's Monitor value is not True.\\
The value \code{0} has the meaning "no limit".

\paragraph{Monitor}
A boolean that describes whether or not this interface should be monitored by
Traffic Monitor. Note that multiple Interfaces that have True Monitor values
is incompatible with a Health Polling Format value of "astats" or
"astats-dsnames".

\paragraph{MTU}
MTU is an unsigned integer that gives the Interface's Maximum Transmission
Unit. It is recommended that UIs built on the API provide the values 9000
and 1500 for selection only, as other values are very rarely correct. However,
the API itself places no such restrictions on the value, which can be any
unsigned integer or a "null"-type.\\
If the value is a "null"-type, then it is assumed that the interface's MTU is
not known or is not relevant for the Cache Server's operation (common on
non-Service-Address-containing Interfaces).

\paragraph{Name}
The string containing a network interface's Name MUST NOT be allowed to be
empty. It should name the actual network interface device on the Cache Server.
For example, two common Names are "eth0" and "bond0".

\subsubsection{Notes}
This section is an arbitrary string containing miscellaneous, human-friendly
information about the Cache Server. Other ATC components SHOULD NOT parse this
for specific information fields, or expect it to be in a particular format.

\subsubsection{Physical Location}
The Physical Location at which a Cache Server resides is represented by a string
containing its Name.

\subsubsection{Profile}
The Profile used by a Cache Server is represented by a string containing its
Name.

\subsubsection{Revalidation Pending}
This boolean represents whether or not the Cache Server has content revalidation
requests yet to satisfy. When a new Content Revalidation Request is created on
one or more Delivery Services to which this Cache Server is assigned, this will
be set to "true" and updated to "false" when the operation has been completed.
Being "false" does not mean that the content invalidation request(s) performed
by the Cache Server have expired and are no longer in effect.

\subsubsection{Status}
The Cache Server's "Status" is a string constant, which MUST always be one of

\begin{itemize}
	\item \code{ADMIN\_DOWN} - The Cache Server is considered unhealthy and its
	thresholds and connectivity state are not monitored. Its existence is not
	disclosed to Traffic Router(s).
	\item \code{OFFLINE} - The Cache Server is considered unhealthy regardless
	of any thresholds or connectivity state.
	\item \code{ONLINE} - The Cache Server will always be considered healthy
	regardless of any thresholds or connectivity state.
	\item \code{REPORTED} - The Cache Server's health is presented to the
	Traffic Router(s) as it is reported by its various thresholds, as determined
	by the Traffic Monitor(s).
\end{itemize}

\subsubsection{Tags}
A Cache Server's "Tags" are represented as a set of Tag Names.

\subsubsection{Type}
A Cache Server's "Type" is a string constant, which MUST always be one of

\begin{itemize}
	\item \code{EDGE} - This is an Edge-tier Cache Server which acts as a
	reverse proxy.
	\item \code{MID} - This is a Mid-tier Cache Server which acts as a forward
	proxy.
\end{itemize}

\subsubsection{Updates Pending}
This boolean represents whether or not the Cache Server has configuration
updates pending. When such updates are applied, this will be set to "false" by
the Cache Server's ORT/ATSTCCFG instance.

% Licensed to the Apache Software Foundation (ASF) under one
% or more contributor license agreements.  See the NOTICE file
% distributed with this work for additional information
% regarding copyright ownership.  The ASF licenses this file
% to you under the Apache License, Version 2.0 (the
% "License"); you may not use this file except in compliance
% with the License.  You may obtain a copy of the License at
%
%   http://www.apache.org/licenses/LICENSE-2.0
%
% Unless required by applicable law or agreed to in writing,
% software distributed under the License is distributed on an
% "AS IS" BASIS, WITHOUT WARRANTIES OR CONDITIONS OF ANY
% KIND, either express or implied.  See the License for the
% specific language governing permissions and limitations
% under the License.

\subsection{Capabilities}
A Capability expresses the capacity of a Cache Server to serve a specific kind
of traffic. They are similar to Tags in that they are applied to group objects
into logical groupings and no special meaning may be inferred from their Name
or from patterns of their Names\footnote{Including by Traffic Ops and the
Traffic Ops API.}. However, a Capability differs from a Tag in that the
presence or absence of a Capability on an object \emph{does} have semantic
meaning.\\
In particular, a Delivery Service has Required Capabilities that define the
kinds of traffic a Cache Server must be able to handle in order to serve
content for that Delivery Service. This requirement is enforced by ensuring
that only Cache Servers that have \emph{all} of a Delivery Service's Required
Capabilities within their own Capabilities set are selected via parentage at
the Cache Server configuration generation step, as well as by Traffic Router
when selecting Edge-Tier Cache Servers to which to direct clients requesting
the Delivery Service's content.\\
Note that while Capabilities are enforced by routing and Cache Server parentage
it is not possible to enforce that a Cache Server actually \emph{has} the
Capabilities that are within its Capabilities set, and in fact a Capability's
Name has no semantic meaning that could be checked in any case. In this way it
is possible to use them to express arbitrary concepts, but operators must be
careful to only assign Capabilities to Cache Servers when it is known that they
actually posses those Capabilities.

\begin{codelisting}
\captionof{listing}{Capability Object as a Typescript Interface}
\begin{minted}[tabsize=2]{typescript}
interface Capability {
	Description: string;
	Name: string;
}
\end{minted}
\end{codelisting}

\subsubsection{Description}
This string contains arbitrary text which ideally describes the capacity
encapsulated and expressed by the Capability, as well as the reason for its
existence.

\subsubsection{Name}
A Capability's Name is a string that uniquely identifies it, and ideally
conveys the capacity it expresses, e.g. a Capability Named \code{400GB\_RAM} may
express that a Cache Server has/must have a minimum of 400GB of RAM device
block storage to be used for caching.\\
A Capability Name MUST consist only of alphanumeric characters, hyphens, and
underscores.



% Licensed to the Apache Software Foundation (ASF) under one
% or more contributor license agreements.  See the NOTICE file
% distributed with this work for additional information
% regarding copyright ownership.  The ASF licenses this file
% to you under the Apache License, Version 2.0 (the
% "License"); you may not use this file except in compliance
% with the License.  You may obtain a copy of the License at
%
%   http://www.apache.org/licenses/LICENSE-2.0
%
% Unless required by applicable law or agreed to in writing,
% software distributed under the License is distributed on an
% "AS IS" BASIS, WITHOUT WARRANTIES OR CONDITIONS OF ANY
% KIND, either express or implied.  See the License for the
% specific language governing permissions and limitations
% under the License.

\subsection{Delivery Services}
Delivery Services are, at their most basic, an association between a source of
content and a set of Cache Servers and configuration options used to distribute
that content.

\begin{codelisting}
\captionof{listing}{Delivery Service as a Typescript Type}
\begin{minted}[tabsize=2]{typescript}
type DeliveryService = DNSDeliveryService |
	DNSMSODeliveryService |
	HTTPDeliveryService |
	HTTPMSODeliveryService |
	SteeringDeliveryService |
	SteeringMSODeliveryService |
	StaticDeliveryService;
\end{minted}
\end{codelisting}

\subsubsection{Common Properties}
Herein described are the properties common to all Delivery Service objects. The
Routing Type of a Delivery Service encapsulates the methods by which clients
may request content routing, and depending on its value the Delivery Service
takes on a set of additional properties. Put simply, these are all different
types of objects that are closely related.\\
This section details all of the properties that are common to \emph{all} types
of Delivery Services.

\begin{codelisting}
\captionof{listing}{Delivery Service Object as a Typescript Interface}
\begin{minted}[tabsize=2]{typescript}
interface BaseDeliveryService {
	anonymousBlocking: boolean;
	bypassDestination: string | null;
	cdn: string;
	deniedAccessRedirect: string | null;
	dnsTTL: bigint | null;
	edns0ClientSubnetEnabled: boolean;
	geographicLimiting: Set<string> | null;
	maximumRecords: bigint | null;
	missLocation: {
		latitude: number;
		longitude: number;
	} | null;
	name: string;
	notes: string;
	routingName: string;
	routingType: 'HTTP' | 'DNS' | 'STEERING' | 'STATIC';
	status: 'ACTIVE' | 'PRIMED' | 'INACTIVE';
	tags: Set<string>;
	tenant: string;
	vanityHostnames: string;
}
\end{minted}
\end{codelisting}

\paragraph{Anonymous Blocking}
A Delivery Service that has "Anonymous Blocking" tells Traffic Router to block
requests from anonymized IP addresses. Whether or not and how well that can
actually be done is dependent on the configuration of each Traffic Router
itself, and if this Delivery Service is DNS-routed the only IP address Traffic
Routers will be capable of checking for anonymization (e.g. known proxy/VPN/TOR
exit node) will be the downstream router requesting the name resolution and thus
is likely much less effective.

\paragraph{Bypass Destination}
This is a string that describes the network location to which clients will be
directed if the traffic served by this Delivery Service exceeds its allowed
maximums. This MUST always be represented as a string - even if the
representation format supports IP Addresses as a native type - and its
interpretation is dependent on the Delivery Service's Routing Type.\\
If the Delivery Service's Routing Type is \code{HTTP}, then this is interpreted
- and validated by the API - as a Fully Qualified Domain Name (FQDN) optionally
followed by a colon and port number that defines an HTTP server to which client
requests will be directed.\\
If the Delivery Service's Routing Type is \code{DNS} or \code{STATIC}, then if
this is a valid IPv4 address it is assumed to be one and will be presented as an
AA record. If not, and it's a valid IPv6 address, then it is assumed to one and
will be presented as an AAAA record. Finally, it may be an FQDN in which case it
will be presented as a CNAME record. If none of these formats can be validated,
then it MUST be rejected by the API.\\
If the Delivery Service's Routing Type is \code{STEERING}, then this MUST be the
Name of an existing Delivery Service - though it need not name one of this
Delivery Service's Targets.

\paragraph{CDN}
The CDN to which a Delivery Service belongs is expressed as a string that is the
Name used to uniquely identify it.

%TODO: use-case for check path?

\paragraph{Deep Caching}
A boolean value that describes whether or not "Deep Caching" may be used for
this Delivery Service.

\paragraph{Denied Access Redirect}
This is a string that describes the network location to which clients will be
directed if they are denied access on the basis of Anonymous Blocking and/or
Geographic Limiting settings. This MUST always be represented as a string - even
if the representation format supports IP Addresses as a native type - and its
interpretation is dependent on the Delivery Service's Routing Type.

\paragraph{DNS TTL}
An unsigned integer that defines the Time-To-Live (TTL) of DNS responses from
the Traffic Router for this Delivery Service's routing, in seconds.
If this is null-typed, the Traffic Router serving this Delivery Service's
traffic will use its pre-configured default TTL for whatever type of record was
requested.

\paragraph{EDNS0 Client Subnet Enabled}
A boolean that describes whether or not the EDNS0 DNS extension mechanism
described in \href{RFC2671}{https://tools.ietf.org/html/rfc2671} should be made
available to clients.\\
Note that the ability of a Traffic Router to actually implement this setting
depends on its own EDNS0 Client Subnet enabled value.

%TODO: Necessary?
%\paragraph{ECS}

\paragraph{Geographic Limiting}
This property describes limitations to the availability of this Delivery
Service's content on the basis of the requesting client's geographic location.
It is a set of strings, each of which is an
\href{https://www.iso.org/obp/ui/#search/code/}{ISO 3166-1} alpha-2 country
code, optionally with ISO 3166-2 subdivisional alphabetic code. This is a "white
list" of countries/subdivisions wherein content is to be made
available\footnote{This property is meant to inform Traffic Router; Cache
Servers cannot be relied upon to approve or deny access on a geographic basis.
Thus, if routing is bypassed, restricted content is totally accessible to
requesting clients.}.\\
Content is \emph{always} available to clients whose IP addresses are found
within the Traffic Routers' Coverage Zone File(s). With that in mind, when this
property is an empty set it means that no geographic regions are "whitelisted"
and thus \emph{only} clients whose IP addresses are found within a Coverage Zone
File will be granted access to content. When this property has a "Null" type,
there is no geographic restriction placed on the Delivery Service's content
access.

%TODO: use-case for header-rewrite text?

\paragraph{Maximum Records}
This unsigned integer sets the maximum number of records returned in DNS
responses. It MUST be greater than zero, or null-typed, which indicates that
there should be no limit placed on the maximum returned records.

\paragraph{Miss Location}
This is an object with two properties, latitude and longitude, which are
floating-point numbers that define a pair of geographic coordinates to be used as
a fallback in the event that attempts to find a geographic location for a client
based on Coverage Zone Files and IP address look-up have failed. If this is
null-typed, then there is no configured miss location and Traffic Router will
use its pre-configured default fallback location. \footnote{The Miss Location of
a Delivery Service is uneditable by users that do not have the
\code{delivery-service-miss-location} Permission (as well as any other
Permission(s) required by the endpoint used for editing).}

\paragraph{Name}
A Delivery Service's "Name" is a string that uniquely identifies it among all
Delivery Services. It MUST only contain alphanumerics, hyphens, underscores and
spaces, and MUST NOT begin with a non-alphanumeric character nor end with a
non-alphanumeric character. This is used to generate part of the default request
hostnames by replacing all non-alphanumeric characters with a hyphen.

\paragraph{Notes}
This section is an arbitrary string containing miscellaneous, human-friendly
information about the Delivery Service. Other ATC components SHOULD NOT parse
this for specific information fields, or expect it to be in a particular format.

%TODO Raw-Remap use-case?
%TODO Regex-Remap use-case?

\paragraph{Routing Name}
The lowest-level DNS label (e.g. in \code{traffic-control-cdn.readthedocs.io}
the lowest-level DNS label is \code{traffic-control-cdn}) used to route to a
Delivery Service is set by this string. UIs currently default this value to
"cdn", but it may be changed for vanity purposes. It MAY NOT be an empty string
and MUST be a valid DNS label.

%TODO: ANY_MAP use-case?
\paragraph{Routing Type}
After creation, Routing Type is a read-only property. It's a string that
describes the type of routing used to serve content for the Delivery Service.
Each of these routing types has its own section below.
\begin{itemize}
	\item \code{HTTP}
	\item \code{DNS}
	\item \code{STEERING}
	\item \code{STATIC}
\end{itemize}

\paragraph{Status}
The "Status" of a Delivery Service is a string constant that expresses its
ability to serve content at the present moment in time. It may have one of three
values:

\begin{itemize}
	\item \code{ACTIVE} A Delivery Service that is "active" is one that is
	functionally in service, and fully capable of delivering content. This means
	that its configuration is deployed to Cache Servers and it is available for
	routing traffic.
	\item \code{PRIMED} A Delivery Service that is "primed" has had its
	configuration distributed to the various servers required to serve its
	content. However, the content itself is still inaccessible\footnote{The
	content is not available through normal routing. This does not, though,
	guarantee that Cache Servers do not already have the content stored and/or
	are incapable of serving it if routing is bypassed.}.
	\item \code{INACTIVE} A Delivery Service that is "inactive" is not available
	for routing and has not had its configuration distributed to its assigned
	Cache Servers.
\end{itemize}

\paragraph{Supported Protocols}
This is a set of strings that name protocols served by the Delivery Service.
Note that this is the method used to retrieve content from the caching system,
not the method used for routing. The only protocols officially supported by ATC
are "HTTP" and "HTTPS".\\
This set is case-insensitive, such that if a Delivery Service is created with a
Supported Protocols set containing "HTTP" the resulting set is equivalent to
what would result from creating it with a Supported Protocols set containing
"http". Representations produced by the Traffic Ops API MUST always use
only uppercase characters.

\paragraph{Tags}
The Tags associated with a Delivery Service are represented by a set of strings
that are Tag Names.

\paragraph{Tenant}
The Tenant to which a Delivery Service belongs is represented by a string that
is that Tenant's unique Name.

\paragraph{Vanity Hostnames}
"Vanity Hostnames" is a set of strings that are Fully Qualified Domain Names
(FQDN) which may be used as alternates to the standard
\code{\emph{Routing Name}.\emph{Name}.\emph{CDN Domain}} FQDN when requesting
content from the Delivery Service.\\
No two Delivery Services may be allowed to share any single Vanity Hostname.\\
Note that Traffic Ops - and in fact Traffic Control in general - cannot and
does not guarantee that these vanity names will work, only that Traffic Router
will respond to them as equivalents to a normal Delivery Service FQDN when
content is requested through it. In general, because Vanity Hostnames are
typically outside of the CDN's Domain (which is the only domain for which
Traffic Router must be authoritative), this requires the DNS servers that are
authoritative for each Vanity Hostname's Domain to contain records that will
point to Traffic Router for resolution of these names.

\subsubsection{DNS-Routed Properties}
A DNS-Routed Delivery Service is defined primarily by having a Routing Type of
"DNS". These Delivery Services direct clients to content by responding to DNS
queries for Delivery Service hostnames with records indicating the network
addresses of cache servers.

\begin{codelisting}
\captionof{listing}{DNS-Routed Delivery Service as a Typescript Interface}
\begin{minted}[tabsize=2]{typescript}
interface DNSDeliveryService extends BaseDeliveryService {
	bypassTTL: bigint | null;
	caching: 'CACHE' | 'RAM_ONLY' | 'NO_CACHE';
	dscp: bigint;
	maximumOriginConnections: bigint | null;
	origin: bigint | Set<bigint>;
	queryStringHandling: 'DROP' | 'IGNORE' | 'USE';
	rangeRequestHandling: 'NO_CACHE' | 'WHOLE_OBJECT' | 'CACHE';
	requiredCapabilities: Set<string>;
	routingType: 'DNS';
	topology: string;
}
\end{minted}
\end{codelisting}

\paragraph{Bypass TTL}
An unsigned integer that defines the Time-To-Live (TTL) of DNS responses from
the Traffic Router for this Delivery Service's Bypass Destination, in seconds.
If this is null-typed, the Traffic Router serving this Delivery Service's
traffic will use its pre-configured default TTL for whatever type of record was
requested.

\paragraph{Caching}
Caching describes how Delivery Service content is cached - if at all. It is a
string content restricted to one of the values:

\begin{itemize}
	\item \code{CACHE} The Delivery Service's content will be cached normally.
	\item \code{RAM\_ONLY} The Delivery Service's content will only be cached
	in RAM block devices.
	\item \code{NO\_CACHE} The Delivery Service's content is proxied through
	Cache Servers without ever being actually cached.
\end{itemize}

\paragraph{DSCP}
Sets the
\href{https://tools.ietf.org/html/rfc2474}{Differentiated Services Code Point}
which will be marked on the Delivery Service's traffic. This is an unsigned
integer with a maximum value of 64.

\paragraph{Maximum Origin Connections}
This is an unsigned integer which determines the maximum number of connections
that any \emph{one} Cache Server may open to the Delivery Service's Origin.\\
If this is null-typed, it has the special meaning "no limit".

\paragraph{Origin}
In normal cases, the Origin whose content is served by this Delivery Service is
identified by an unsigned integer that is its ID. However, in the event that
the Delivery Service is a Multi-Site Origin Delivery Service, it will have a
set of such identifiers indicating all Origins used by the Delivery Service.
More detail about Multi-Site Origin Delivery Services can be found in
\ref{sec:mso-props}.

\paragraph{Query String Handling}
Query String Handling is a string with one of three possible values that
describe how the cache layers should handle query strings in requested URLs.

\begin{description}
	\item[\code{DROP}] The first cache layer encountered (EDGE-tier Cache
	Servers) will totally ignore all query strings, they will not be used in the
	cache key, and will not be passed upstream in requests.
	\item[\code{IGNORE}] The first cache layer encountered (EDGE-tier Cache
	Servers) will not use query strings as part of the cache key, but will pass
	them upstream in requests.
	\item[\code{USE}] The first cache layer encountered (EDGE-tier Cache
	Servers) will use query strings in the cache key and will pass them upstream
	in requests.
\end{description}

\paragraph{Range Request Handling}
Range Request Handling is a string with one of three possible values that
describe how cache servers should handle HTTP Range requests.

\begin{description}
	\item[\code{NO\_CACHE}] Range requests will not be cached, and requests for
	ranges will always be proxied upstream all the way to the origin.
	\item[\code{WHOLE\_OBJECT}] When a request for a range of an object is
	received, the entire object is fetched from upstream and cached, and
	subsequent range requests will be served by taking ranges of the cached
	object.
	\item[\code{CACHE}] Range requests will be cached as normal requests, each
	unique range representing a unique, cached object - this includes ranges
	that may overlap.
\end{description}

\paragraph{Required Capabilities}
The Capabilities required by a Delivery Service for cache servers to serve its
content are given by a set of strings that are the names of those Capabilities.

\paragraph{Topology}
The Topology used by a Delivery Service is represented by a string that is the
Topology's unique Name.

\subsubsection{HTTP-Routed Properties}
An HTTP-Routed Delivery Service is defined primarily by having a Routing Type of
"HTTP". These Delivery Services direct clients to content by responding to DNS
queries for Delivery Service hostnames with records indicating its own network
address, then replying to subsequent HTTP requests for Delivery Service content
with HTTP 302 Found responses directing clients to cache servers.

\begin{codelisting}
\captionof{listing}{HTTP-Routed Delivery Service as a Typescript Interface}
\begin{minted}[tabsize=2]{typescript}
interface HTTPDeliveryService extends BaseDeliveryService {
	additionalResponseHeaders: Map<string, string>;
	caching: 'CACHE' | 'RAM_ONLY' | 'NO_CACHE';
	consistentHashingRegularExpression: RegExp | null;
	dscp: bigint;
	loggedRequestHeaders: Set<string>;
	maximumOriginConnections: bigint | null;
	origin: bigint | Set<bigint>;
	queryStringHandling: 'DROP' | 'IGNORE' | 'USE';
	rangeRequestHandling: 'NO_CACHE' | 'WHOLE_OBJECT' | 'CACHE';
	requiredCapabilities: Set<string>;
	routingType: 'HTTP';
	significantQueryParameters: Set<bigint> | null;
	topology: string;
}
\end{minted}
\end{codelisting}

\paragraph{Additional Response Headers}
Any additional HTTP headers desired to appear in HTTP responses from Traffic
Router are indicated here as header names mapped to their desired values.

\paragraph{Caching}
Caching describes how Delivery Service content is cached - if at all. It is a
string content restricted to one of the values:

\begin{itemize}
	\item \code{CACHE} The Delivery Service's content will be cached normally.
	\item \code{RAM\_ONLY} The Delivery Service's content will only be cached
	in RAM block devices.
	\item \code{NO\_CACHE} The Delivery Service's content is proxied through
	Cache Servers without ever being actually cached.
\end{itemize}

\paragraph{Consistent Hashing Regular Expression}
When Traffic Router performs consistent hashing on client HTTP requests to find
a cache server to which to redirect, it will use this regular expression - if it
is not null-typed - to extract the parts of the request path to use as the
hashing key.

\paragraph{DSCP}
Sets the
\href{https://tools.ietf.org/html/rfc2474}{Differentiated Services Code Point}
which will be marked on the Delivery Service's traffic. This is an unsigned
integer with a maximum value of 64.

\paragraph{Logged Request Headers}
Any HTTP headers in client requests that should appear in Traffic Router logs
are identified in this set of string by header name.

\paragraph{Maximum Origin Connections}
This is an unsigned integer which determines the maximum number of connections
that any \emph{one} Cache Server may open to the Delivery Service's Origin.\\
If this is null-typed, it has the special meaning "no limit".

\paragraph{Origin}
In normal cases, the Origin whose content is served by this Delivery Service is
identified by an unsigned integer that is its ID. However, in the event that
the Delivery Service is a Multi-Site Origin Delivery Service, it will have a
set of such identifiers indicating all Origins used by the Delivery Service.
More detail about Multi-Site Origin Delivery Services can be found in
\ref{sec:mso-props}.

\paragraph{Query String Handling}
Query String Handling is a string with one of three possible values that
describe how the cache layers should handle query strings in requested URLs.

\begin{description}
	\item[\code{DROP}] The first cache layer encountered (EDGE-tier Cache
	Servers) will totally ignore all query strings, they will not be used in the
	cache key, and will not be passed upstream in requests.
	\item[\code{IGNORE}] The first cache layer encountered (EDGE-tier Cache
	Servers) will not use query strings as part of the cache key, but will pass
	them upstream in requests.
	\item[\code{USE}] The first cache layer encountered (EDGE-tier Cache
	Servers) will use query strings in the cache key and will pass them upstream
	in requests.
\end{description}

\paragraph{Range Request Handling}
Range Request Handling is a string with one of three possible values that
describe how cache servers should handle HTTP Range requests.

\begin{description}
	\item[\code{NO\_CACHE}] Range requests will not be cached, and requests for
	ranges will always be proxied upstream all the way to the origin.
	\item[\code{WHOLE\_OBJECT}] When a request for a range of an object is
	received, the entire object is fetched from upstream and cached, and
	subsequent range requests will be served by taking ranges of the cached
	object.
	\item[\code{CACHE}] Range requests will be cached as normal requests, each
	unique range representing a unique, cached object - this includes ranges
	that may overlap.
\end{description}

\paragraph{Required Capabilities}
The Capabilities required by a Delivery Service for cache servers to serve its
content are given by a set of strings that are the names of those Capabilities.

\paragraph{Significant Query Parameters}
If this is not null-typed, then client HTTP request paths are modified prior to
consistent hashing by parsing any query string that's present in the request
path as an application/x-www-form-urlencoded set of parameter-value pairs, and
stripping all pairs that don't have keys present in this set of strings.

\paragraph{Topology}
The Topology used by a Delivery Service is represented by a string that is the
Topology's unique Name.

\subsubsection{Static-Routed Properties}
\begin{codelisting}
\captionof{listing}{Static-Routed Delivery Service as a Typescript Interface}
\begin{minted}[tabsize=2]{typescript}
interface StaticDeliveryService extends BaseDeliveryService {
	origin: string;
	routingType: 'STATIC';
}
\end{minted}
\end{codelisting}

\paragraph{Origin}

\subsubsection{Steering-Routed Properties}
\begin{codelisting}
\captionof{listing}{Steering-Routing Delivery Service as a Typescript Interface}
\begin{minted}[tabsize=2]{typescript}
interface SteeringTarget {
	target: string;
	type: 'STEERING_WEIGHT' |
		'STEERING_ORDER' |
		'STEERING_GEO_WEIGHT' |
		'STEERING_GEO_ORDER';
	value: bigint;
}

interface SteeringDeliveryService extends BaseDeliveryService {
	routingType: 'STEERING';
	targets: Set<SteeringTarget>;
}
\end{minted}
\end{codelisting}

\paragraph{Targets}

\subsubsection{Multi-Site Origin Properties\label{sec:mso-props}}

% Licensed to the Apache Software Foundation (ASF) under one
% or more contributor license agreements.  See the NOTICE file
% distributed with this work for additional information
% regarding copyright ownership.  The ASF licenses this file
% to you under the Apache License, Version 2.0 (the
% "License"); you may not use this file except in compliance
% with the License.  You may obtain a copy of the License at
%
%   http://www.apache.org/licenses/LICENSE-2.0
%
% Unless required by applicable law or agreed to in writing,
% software distributed under the License is distributed on an
% "AS IS" BASIS, WITHOUT WARRANTIES OR CONDITIONS OF ANY
% KIND, either express or implied.  See the License for the
% specific language governing permissions and limitations
% under the License.

\subsection{Global Configuration}
The Global Configuration singleton object contains information pertaining to the
configuration of the Traffic Control system as a whole - most notably it stores
Traffic Ops API access information and provides information for geographically
locating clients of Apache Traffic Control CDNs.

\begin{codelisting}
\captionof{listing}{Global Configuration Object as a Typescript Interface}
\label{code:datamodel:global-configuration}
\begin{minted}[tabsize=2]{typescript}
/**
 * GlobalConfiguration is the singleton object that stores global configuration
 * for a Traffic Control system.
 */
interface GlobalConfiguration {
	/** The default latitude to use for CDN clients when geolocation fails */
	defaultGeoMissLatitude: number;
	/** The default longitude to use for CDN clients when geolocation fails */
	defaultGeoMissLongitude: number;
	/** The location of an IPv4 geolocation database */
	geolocationIPv4PollingURL: URL;
	/** The location of an IPv6 geolocation database */
	geolocationIPv6PollingURL: URL;
	/** A URL at which information about the ATC system may be found */
	informationURL: URL | null;
	/** The name of the ATC instance */
	instanceName: string;
	/** The maximum duration - in days - of content invalidation jobs */
	maxRevalidationDays: bigint | null; // >= 0
	/** A URL for a reverse proxy to the Traffic Ops instance(s) */
	reverseProxyURL: URL & {search: "", pathname: "/"} | null;
	/** The name of the tool that serves the Traffic Ops API */
	toolName: string;
	/** The canonical URL used to make requests to the Traffic Ops API */
	trafficOpsURL: URL & {search: "", pathname: "/"};
}
\end{minted}
\end{codelisting}

\subsubsection{Default Geo Miss Latitude}
This floating point number is the default latitude to use when choosing a
geographically appropriate location for a client and detecting the client's
location (either via a Coverage Zone or database lookup) has failed. It MUST be
on the interval [-90, 90] - with the positive range representing North and the
negative range representing South.

\subsubsection{Default Geo Miss Longitude}
This floating point number is the default longitude to use when choosing a
geographically appropriate location for a client and detecting the client's
location (either via a Coverage Zone or database lookup) has failed. It MUST be
on the interval [-180, 180] - with the positive range representing East and the
negative range representing West.

\subsubsection{Geolocation IPv4 Polling URL}
This string is a URL at which a database (Neustar or MaxMind) for converting
IPv4 addresses to geographic locations may be requested. This is primarily used
by Traffic Router.

\subsubsection{Geolocation IPv6 Polling URL}
This string is a URL at which a database (Neustar or MaxMind) for converting
IPv6 addresses to geographic locations may be requested. This is primarily used
by Traffic Router.

\subsubsection{Information URL}
This string is a URL at which information about the Apache Traffic Control
system may be found for new users and/or clients. It has no specific semantics,
and may be "null"-typed to indicate that no such URL is available.

\subsubsection{Instance Name}
This string names the Apache Traffic Control instance, e.g. "my CDN for image
macros". It is used in certain user interface displays and email communications.

\subsubsection{Max Revalidation Days}
This unsigned integer indicates the maximum time for which an Invalidation Job
may use as a duration, in days. If this is "null"-typed, then there is no
upper bound placed on the duration of Invalidation Jobs.\footnote{Note that a
value of zero ($0$) will cause all Invalidation Jobs to be rejected on creation,
as there will be no valid duration.}

\subsubsection{Reverse Proxy URL}
If Traffic Ops uses a reverse proxy to help with caching/load balancing, then a
URL at which it may be requested will be found in this string. It SHOULD be an
HTTPS URL, and MUST NOT include any path besides \code{/}, nor may it include
any query string or document fragment parts.\\
If this is "null"-typed, then no such proxy exists.

\subsubsection{Tool Name}
This string is the name of the tool that serves the Traffic Ops API (and so is
usually "Traffic Ops") as it appears in certain user interface displays and
email communications.

\subsubsection{Traffic Ops URL}
This string is the canonical URL used to access Traffic Ops, including any port.
It SHOULD be an HTTPS URL, and MUST NOT include any path besides \code{/}, nor
may it include any query string or document fragment parts. This MUST NOT be a
URL of a reverse proxy for Traffic Ops, as that should appear as the Reverse
Proxy URL.

% Licensed to the Apache Software Foundation (ASF) under one
% or more contributor license agreements.  See the NOTICE file
% distributed with this work for additional information
% regarding copyright ownership.  The ASF licenses this file
% to you under the Apache License, Version 2.0 (the
% "License"); you may not use this file except in compliance
% with the License.  You may obtain a copy of the License at
%
%   http://www.apache.org/licenses/LICENSE-2.0
%
% Unless required by applicable law or agreed to in writing,
% software distributed under the License is distributed on an
% "AS IS" BASIS, WITHOUT WARRANTIES OR CONDITIONS OF ANY
% KIND, either express or implied.  See the License for the
% specific language governing permissions and limitations
% under the License.
\subsection{Infrastructure Servers}
Infrastructure servers are arbitrary servers that aren't ATC components. This
is intended primarily for record-keeping purposes for organizations with large
support infrastructure for their CDNs, as no out-of-the-box functionality is
provided by ATC for these servers.

\subsubsection{CDN}
This is a string that is the Name of the CDN to which the Infrastructure Server
belongs, if any. If this is "null"-typed, then it is understood that the server
does not operate in a way that is limited in scope to a single CDN.

\subsubsection{Domain}
The "domain" part of the Infrastructure Server's Fully Qualified Domain Name
(FQDN) as a string. For example, an Infrastructure Server with an FQDN of
\code{github.com} has a Domain of \code{com} and an Infrastructure Server with
an FQDN of \code{trafficcontrol.apache.org} has a Domain of
\code{apache.org}.\\
In other words, this is every part of the Infrastructure Server's FQDN that
isn't part of its Host Name.

\subsubsection{Host Name}
This is the "host" part of the Infrastructure Server's Fully Qualified Domain
Name (FQDN) as a string. For example, an Infrastructure Server with an FQDN of
\code{github.com} has a Host Name of \code{github}, and an Infrastructure
Server with an FQDN of \code{trafficcontrol.apache.org} has a Host Name of
\code{trafficcontrol}.\\
In other words, an Infrastructure Server's Host Name is the only part of its
FQDN that is not part of its Domain.\\
This field does NOT need to be unique, though operators are encouraged to make
this - or at least the concatenation \code{\emph{Host Name}.\emph{Domain}} -
unique for ease of operation.

\subsubsection{ID}
An integral, unique identifier for the Infrastructure Server. It carries no
meaning or significance beyond being a unique identifier.

\subsubsection{Notes}
This is a string of arbitrary text for miscellaneous purposes.

\subsubsection{Online}
This boolean value indicates whether or not the Infrastructure Server is online
and/or currently serving its main service.

\subsubsection{Physical Location}
This string gives the Name of a Physical Location within which this
Infrastructure Server resides. If this is "null"-typed value, then it is
understood that the Infrastructure Server has no well-defined and/or
meaningful Physical Location.

\subsubsection{Tags}
The Tags on an Infrastructure Server are represented by a set of Tag Names as
strings.

\subsubsection{Service Port}
This unsigned integer is the network port on which the server's primary service
listens for connections, e.g. \code{80} for HTTP servers.\\
This value may not exceed 65535. If this value is "null"-typed, then it is
understood that the service provided by this server may not be reached on any
particular network port (reliably or at all).

\subsubsection{Service Protocol}
This string of alphanumeric characters defines the protocol used by the
Infrastructure Server's primary service, e.g. \code{"HTTP"} for HTTP servers.\\
If this is "null"-typed, then it is understood that the server's primary service
is not offered using any particular protocol (reliably or at all).

% Licensed to the Apache Software Foundation (ASF) under one
% or more contributor license agreements.  See the NOTICE file
% distributed with this work for additional information
% regarding copyright ownership.  The ASF licenses this file
% to you under the Apache License, Version 2.0 (the
% "License"); you may not use this file except in compliance
% with the License.  You may obtain a copy of the License at
%
%   http://www.apache.org/licenses/LICENSE-2.0
%
% Unless required by applicable law or agreed to in writing,
% software distributed under the License is distributed on an
% "AS IS" BASIS, WITHOUT WARRANTIES OR CONDITIONS OF ANY
% KIND, either express or implied.  See the License for the
% specific language governing permissions and limitations
% under the License.

\subsection{Invalidation Jobs}
Invalidation Jobs - also called "content invalidation jobs", "revalidations",
"content revalidation jobs", "revalidation jobs", or even just "jobs" - are, at
their most basic, periods of time during which specific content should not be
served from cache but instead considered cache-invalid and fetched from
upstream. These objects are limited in scope to within a particular Delivery
Service, but are not strictly "Tenantable", as such.

\begin{codelisting}
\captionof{listing}{Invalidation Job Object as a Typescript Interface}
\label{code:datamodel:invalidation-job}
\begin{minted}[tabsize=2]{typescript}
interface InvalidationJob {
	assetPattern: RegExp;
	createdBy: string;
	deliveryService: string;
	id: bigint;
	startTime: Date;
	tenant: string;
	timeToLive: bigint;
}
\end{minted}
\end{codelisting}

\subsubsection{Asset Pattern}
This is a regular expression that defines the content paths which will be
"invalidated". It MUST begin with \code{/} since all valid request paths begin
with \code{/}. Example: \code{/.*\textbackslash{}.jpg}.

\subsubsection{Created By}
The User that created the Invalidation Job may be referenced by this string
which is their Username.

\subsubsection{Delivery Service}
The Delivery Service within which the Invalidation Job operates is given by this
string which is its Name.

\subsubsection{ID}
ID is an unsigned integer that uniquely identifies an Invalidation Job. There is
no semantic meaning to this property other than unique identification.

\subsubsection{Start Time}
This is the date and time at which the Invalidation Job is set to begin. Upon
creation, this MUST be at most one hour before the current time\footnote{This is
done to account for potential differences between what time the server thinks it
is and what time the client thinks it is - it is neither the intention nor,
indeed, possible that a content invalidation job begin before it is submitted.}.

\subsubsection{Tenant}
The Tenant of an Invalidation Job is a string which is the Name of the Tenant to which it belongs. The Delivery Service to which the Invalidation Job applies
MUST fall within the Tenancy of this Tenant\footnote{This is enforced upon
creation by the API.}.

\subsubsection{Time to Live}
This unsigned integer number of hours for which the Invalidation Job should
remain "active" - that is, this defines the window within which the specified
content will not be served from cache.\\
This may not be zero ($0$). This has an upper bound at creation time defined by
the Max Revalidation Days property of the Global Configuration object.

% Licensed to the Apache Software Foundation (ASF) under one
% or more contributor license agreements.  See the NOTICE file
% distributed with this work for additional information
% regarding copyright ownership.  The ASF licenses this file
% to you under the Apache License, Version 2.0 (the
% "License"); you may not use this file except in compliance
% with the License.  You may obtain a copy of the License at
%
%   http://www.apache.org/licenses/LICENSE-2.0
%
% Unless required by applicable law or agreed to in writing,
% software distributed under the License is distributed on an
% "AS IS" BASIS, WITHOUT WARRANTIES OR CONDITIONS OF ANY
% KIND, either express or implied.  See the License for the
% specific language governing permissions and limitations
% under the License.

\subsection{Origins}
Origins are sources of content for the CDN to serve. They are typically external
to the CDN's infrastructure and in many cases may in fact not be administered by
the same organization as the CDN.

%TODO: necessary?
% \subsubsection{ID}
% An Origin's ID is an unsigned integer that identifies it uniquely. It serves no
% other purpose than to uniquely identify an Origin.

\subsubsection{IPv4 Address}
This string is a valid IP (v4) address at which the Origin may be found, or
"null"-typed to indicate there is no static IPv4 Address where the Origin may be
relied on to be.

\subsubsection{IPv6 Address}
This string is a valid IP (v6) address at which the Origin may be found, or
"null"-typed to indicate there is no static IPv6 Address where the Origin may be
relied on to be.

\subsubsection{Notes}
This is a free text field used to record miscellaneous notes about the Origin.

\subsubsection{Physical Location}
The Physical Location at which an Origin resides is represented by a string that
is its Name.

\subsubsection{Tags}
The Tags associated with an Origin - or, with which it is "tagged" - are a set
of strings that are Tag Names.

\subsubsection{Tenant}
The Tenant to which an Origin belongs is represented by a string which is its
Name.

\subsubsection{URL}
An Origin's URL is a string that is a valid URL which is used by the CDN (and
ostensibly any client) to request content. As such, the only supported protocol
schemes are those for HTTP and HTTPS - though this is not enforced as a
restriction on the values of the URL.\\
The port part of the network location within the URL may be included, but will
be inferred from the scheme if it is not, e.g. 80 for HTTP protocol schemes.\\
An Origin's URL is also used to uniquely identify it. URLs are considered unique
if they are unique \emph{in totality} - but \emph{not} including inferred
quantities. That is, \code{https://github.com} is uniquely distinguishable from
\code{http://github.com}, but \emph{not} from \code{https://github.com:443}.\\
URLs MUST NOT have paths - other than optionally specifying the root path
(\code{/}) - as content is always assumed to be retrieved relative to the root
of the Origin service.

% Licensed to the Apache Software Foundation (ASF) under one
% or more contributor license agreements.  See the NOTICE file
% distributed with this work for additional information
% regarding copyright ownership.  The ASF licenses this file
% to you under the Apache License, Version 2.0 (the
% "License"); you may not use this file except in compliance
% with the License.  You may obtain a copy of the License at
%
%   http://www.apache.org/licenses/LICENSE-2.0
%
% Unless required by applicable law or agreed to in writing,
% software distributed under the License is distributed on an
% "AS IS" BASIS, WITHOUT WARRANTIES OR CONDITIONS OF ANY
% KIND, either express or implied.  See the License for the
% specific language governing permissions and limitations
% under the License.

\subsection{Profiles}
A Profile is a set of configuration options which may be associated with one or
more Cache Servers. It contains configuration options that are typically common
to a great number of Cache Servers within a CDN, and can therefore be changed
on an arbitrary number of Cache Servers simultaneously.

\subsubsection{Bandwidth Threshold}
The "Bandwidth Threshold" of a Cache Server is the maximum used bandwidth which
can be considered "healthy" for the Cache Server. It is suggested that clients
presenting controls to create/register a new Cache Server present the most
common value within the server's Cache Group as the default. Note that the value
may be zero, in which case the Cache Server will always be considered unhealthy
for the reason of exceeding this threshold.\\
If this value is null-typed, it has the meaning "no limit".

\subsubsection{CDN}
Profiles are scoped to a CDN, because a CDN is defined as all of the
configuration and infrastructure necessary to serve content for its constituent
Delivery Services. So being able to share that configuration across CDNs breaks
this encapsulation.\\
Therefore a Profile belongs to a CDN which is identified by this string which
is the CDN's Name.

\subsubsection{Description}
Description is a human-readable string that ideally describes what the Profile
is for and to what servers it ought to be limited in its assignment.\\
This may be an empty string, but clients of the API are encouraged to promote
its sensible use as much as possible.

\subsubsection{Hard Disks}
A collection of Hard Disk Drives (HDDs)\footnote{This designation is meant to
contrast with "RAM Drives" where a block device maps to a section of main
memory. In practice, Hard Disks may be any block device, include Solid-State
Drives rather than actually HDDs.} available to the Cache Server for caching
content, represented as a set of file paths as strings to said devices. These
devices MUST be accessible for full read and write operations to the user as
whom the caching proxy software runs on the Cache Server.

\subsubsection{History Count}
This unsigned integer determines the amount of health and statistics polls
Traffic Monitor shall maintain for each Cache Server it monitors that uses this
Profile. A "null"-typed value indicates that Traffic Monitor will use its
default - which is not defined by the Traffic Ops API.

\subsubsection{Health Polling Path}
This is a string that names a special request path recognized by the Cache
Server's caching proxy software which will result in the retrieval of statistics
rather than content. This MUST NOT be allowed to be an empty string. It MAY
contain a query string or even document fragment.\\
The only health-and-statistics-gathering implementation truly supported by ATC
is the special "astats" Traffic Server plugin.

\subsubsection{Health Polling Format}
This string names the format in which statistics are retrieved from the Cache
Servers using this Profile. The allowed values are defined by Traffic Monitor,
and MUST NOT be checked by the Traffic Ops API as a part of validation. The
string may not be empty, but if it is "null"-valued, Traffic Monitor MUST treat
it as the default format - which is defined to be "stats\_over\_http".

\subsubsection{Loadavg Threshold}
A Cache Server's "Loadavg Threshold" is a floating-point number that represents
the one-minute "loadavg" above which the Cache Server will be considered
unhealthy. Note that it may be zero, in which case the Cache Server will always
be considered unhealthy for the reason of exceeding this threshold.\\
If this is null-typed, it has the special meaning "no limit".\\
For more information consult the
\href{https://linux.die.net/man/3/getloadavg}{\code{getloadavg(3)} manual page},
and/or the
\href{https://linux.die.net/man/5/proc}{\code{/proc(5)} manual page}.

\subsubsection{Name}
A Profile's Name is a string that uniquely identifies it among all Profiles.\\
It MUST NOT be empty and MUST only contain alphanumeric characters,
underscores, and hyphens.

\subsubsection{Parameters}
A Profile's principle job is to contain Parameters. Each Parameter represents
some arbitrary piece of configuration related to a Cache Server's operation.
These are mainly used by the configuration file generation process to help
create caching-proxy-server-implementation-specific configuration files, but in
practice may represent other miscellaneous configuration options.\\
Each Parameter is an object, the properties of which are herein described.

\paragraph{Configuration File}
A Parameter's Configuration File is a string that typically names the
configuration file to which configuration is added by this Parameter. This
is merely the \emph{name} of the file, and not its full \emph{file path}, for
example a configuration file for Apache Traffic Server that typically is placed
at \code{/etc/trafficserver/records.config} would be named in a Configuration
File property of a Parameter as simply \code{records.config}. The location of
the file on the Cache Server is implementation-specific, and the configuration
file generation system is responsible for knowing where that is.\\
This string MUST NOT be empty - even if it configures something that is not
contained within an actual configuration file it ought to have a Configuration
File value that describes what it configures.\\
The combination of a Parameter's Configuration File and its Name must be unique
among a Profile's Parameters.

\paragraph{Name}
The Name of a Parameter is a string describes the specific configuration option
that the Parameter configures. Typically this is the name of the configuration
field within a configuration file generated for the Cache Server(s) using this
Profile.\\
This string MUST NOT be empty.\\
The combination of a Parameter's Configuration File and its Name must be unique
among a Profile's Parameters.

\paragraph{Secure\label{sec:profile:param:secure}}
Secure is a boolean value which - if \code{true} - will cause its Value to be
obscured for users without the \code{secure-parameters} Permission, and such
users will be unable to edit the Parameter in any way - or even remove it from
the Profile (even if they have all other required Permissions). However, a
Profile can be replaced on a Cache Server even if it has Secure Parameters.

\paragraph{Value}
A Parameter's Value is a string, but it may be semantically interpreted by
configuration management systems in arbitrary ways. For this reason it is very
important that clients of the API ensure that users are empowered to enter
semantically correct values for Parameters based on whatever configuration
management system is in use, to the highest degree possible - because the API
can make no such guarantees and performs no such checks.\\
This string MUST NOT be empty.

\subsubsection{Query Time Threshold}
A Cache Server's "Query Time Threshold" is an unsigned integer number of
milliseconds after which a Traffic Monitor polling it for health will consider
it unhealthy, even if it successfully returns healthy metrics. Note that it may
be zero, in which case the Cache Server(s) using this Profile will always be
considered unhealthy for the reason of exceeding this threshold.\\
If this value is null-typed, it has the special meaning "no threshold"
(however, Traffic Monitor polling timeouts still apply).

\subsubsection{RAM Disks}
A collection of RAM Disk Drives available to the Cache Server for caching
content, represented as a set of file paths as strings to said devices. These
devices MUST be accessible for full read and write operations to the user as
whom the caching proxy software runs on the Cache Server.

\subsubsection{Tags}
The Tags associated with a Profile are represented by a set of strings that are
Tag Names.

\subsubsection{Type\label{sec:profile:type}}
The Type of a Profile determines which kind of Cache Server may use it. Only
Cache Servers with the same Type as a Profile may use that Profile. This helps
prevent accidental misconfiguration in a sea of Parameters.\\
A Profile's Type is a string restricted to one of the following values:

\begin{itemize}
	\item \code{EDGE} - This Profile may only be used by Edge-tier Cache
		Servers which act as reverse proxies.
	\item \code{MID} - This Profile may only be used by Mid-tier Cache Servers
		which acts as forward proxies.
\end{itemize}

% Licensed to the Apache Software Foundation (ASF) under one
% or more contributor license agreements.  See the NOTICE file
% distributed with this work for additional information
% regarding copyright ownership.  The ASF licenses this file
% to you under the Apache License, Version 2.0 (the
% "License"); you may not use this file except in compliance
% with the License.  You may obtain a copy of the License at
%
%   http://www.apache.org/licenses/LICENSE-2.0
%
% Unless required by applicable law or agreed to in writing,
% software distributed under the License is distributed on an
% "AS IS" BASIS, WITHOUT WARRANTIES OR CONDITIONS OF ANY
% KIND, either express or implied.  See the License for the
% specific language governing permissions and limitations
% under the License.

\subsection{Physical Location}
Physical Locations are exactly what they sound like, a real, physical place
where component hardware is stored.\\
Mainly this contains metadata for support purposes, but it can also be used
to provide some routing fallback behavior.

\subsubsection{Address}
This is a text field with no defined structure, but semantically it represents
a Physical Location's real-world, physical address. If it is not an empty
value, it is assumed to contain enough information to send a letter through
normal postage to the Physical Location (though the site may not actually be
capable of receiving mail).\\
Addresses may consist of alphanumeric characters, hyphens, periods, spaces, and
newlines, but may neither begin nor end with a space or newline.

\subsubsection{Email}
A string that contains an email address (unless it is an empty string) which is
used for communicating with the Physical Location's "Point of Contact".

\subsubsection{Name}
A string that uniquely identifies the Physical Location. It MUST only contain
alphanumeric characters and spaces, but may neither begin nor end with a space.

\subsubsection{Notes}
This string has no defined structure or semantic meaning and is used to contain
miscellaneous information.

\subsubsection{Phone Number}
A string which, if not empty, is presumed to be a telephone number at which the
Physical Location's "Point of Contact" may be contacted.\\
It may only contain numerics and hyphens, and may neither start nor end with a
hyphen.

\subsubsection{Point of Contact Name}
A string which, if not empty, is presumed to be the proper name of the
designated "Point of Contact" for the Physical Location.\\
This may contain alphabetic character, periods, hyphens, spaces, commas, and
apostrophes, but may neither begin nor end with a hyphen or a space.

% Licensed to the Apache Software Foundation (ASF) under one
% or more contributor license agreements.  See the NOTICE file
% distributed with this work for additional information
% regarding copyright ownership.  The ASF licenses this file
% to you under the Apache License, Version 2.0 (the
% "License"); you may not use this file except in compliance
% with the License.  You may obtain a copy of the License at
%
%   http://www.apache.org/licenses/LICENSE-2.0
%
% Unless required by applicable law or agreed to in writing,
% software distributed under the License is distributed on an
% "AS IS" BASIS, WITHOUT WARRANTIES OR CONDITIONS OF ANY
% KIND, either express or implied.  See the License for the
% specific language governing permissions and limitations
% under the License.

\subsection{Roles\label{sec:roles-and-perms}}
A Role controls what a User can and can't do through the API by declaring the
User's Permissions. The Permissions required by the different API endpoints,
request methods of endpoints, and particular modes of operations of the methods
of those endpoints are declared by the endpoints themselves.

\begin{codelisting}
\captionof{listing}{Role Object as a TypeScript Interface}
\begin{minted}[tabsize=2]{typescript}
interface Role {
	description: string;
	name: string;
	permissions: Set<string>;
}
\end{minted}
\end{codelisting}

\subsubsection{Description}
A Role's Description is a string containing a human-readable description of
what the Role is for and what, in general terms, it is capable of accomplishing
through the API.\\
This may be an empty string, but clients of the API are encouraged to attempt
to promote the idea of useful descriptions as much as possible, to avoid
creating a clutter of Roles that no one understands and/or a swathe of Roles
which no one can determine are safe to delete.

\subsubsection{Name}
A Role's Name is a string that uniquely identifies it - and as such MUST NOT be
empty - and may only contain alphanumeric characters, spaces, and periods.

\subsubsection{Permissions}
The Permissions allowed to a Role are represented as a set of strings which are
the Permission's names.\\
The Permissions that exist - and therefore may be on a Role - are a finite set
and if it is attempted to add a Permission which does not exist to a Role,
the attempt will result in an error from Traffic Ops, and will not be
successful. This set of available Permissions is defined by the Permissions
each individual endpoint declares are required for part or all of their
operation, and therefore cannot (and should not) be enumerated
here\footnote{As a special case, section\ref{sec:profile:param:secure} of the
Data Model describes a Permission that will always exist for as long as that
property of a Profile's Parameter(s) exists.}.

% Licensed to the Apache Software Foundation (ASF) under one
% or more contributor license agreements.  See the NOTICE file
% distributed with this work for additional information
% regarding copyright ownership.  The ASF licenses this file
% to you under the Apache License, Version 2.0 (the
% "License"); you may not use this file except in compliance
% with the License.  You may obtain a copy of the License at
%
%   http://www.apache.org/licenses/LICENSE-2.0
%
% Unless required by applicable law or agreed to in writing,
% software distributed under the License is distributed on an
% "AS IS" BASIS, WITHOUT WARRANTIES OR CONDITIONS OF ANY
% KIND, either express or implied.  See the License for the
% specific language governing permissions and limitations
% under the License.

\subsection{Tags}
Tags may be used to group "taggable" objects for logical or administrative
purposes.\\
While plug-ins and other third-party code such as custom configuration
management systems may glean some semantic meaning from a Tag's Name or
presence on an object, Traffic Control components - \emph{including Traffic Ops
and its API} - MUST NOT infer any special meaning from any Tag's Name or
presence on an object, nor attach any semantic value to a Tag's presence on an
object, a Tag's Name, or any pattern of Tag Name.\\
What is known as a "taggable" object is simply any object with a "Tags"
property that contains a set of Tag Names which are the tags associated with
said object. Thus, any object types added to the API at any point in the future
may be "taggable", but for the data model herein outlined the following objects
those that are "taggable":

\begin{itemize}
	\item Cache Groups
	\item Cache Servers
	\item Delivery Services
	\item Infrastructure Servers
	\item Origins
	\item Physical Locations
	\item Profiles
	\item Tenants
	\item Topologies
	\item Users
\end{itemize}

\subsubsection{Name}
Name is the only property of a Tag, and is therefore a unique identifier for
a Tag. It is a string that ought to give some indication of what the Tag is for
but ultimately any meaning for the Tag is left to system administrators to
define and maintain.\\
This string MUST NOT be empty and MUST only contain alphanumeric characters,
"=", and "_".

% Licensed to the Apache Software Foundation (ASF) under one
% or more contributor license agreements.  See the NOTICE file
% distributed with this work for additional information
% regarding copyright ownership.  The ASF licenses this file
% to you under the Apache License, Version 2.0 (the
% "License"); you may not use this file except in compliance
% with the License.  You may obtain a copy of the License at
%
%   http://www.apache.org/licenses/LICENSE-2.0
%
% Unless required by applicable law or agreed to in writing,
% software distributed under the License is distributed on an
% "AS IS" BASIS, WITHOUT WARRANTIES OR CONDITIONS OF ANY
% KIND, either express or implied.  See the License for the
% specific language governing permissions and limitations
% under the License.

\subsection{Tenants}
A Tenant is ultimately a set of users with permissions to access a set of
resources. This concept differs from a Role or Permission in that it is not
a specific action being controlled, but rather any set of actions performed on
specific objects - \emph{not} specific types of objects.\\
Object types that are scoped to Tenants are called "tenantable" and the types
of objects that are "tenantable" are cataloged in Section
\ref{sec:tenantable}. Note that a Tenant falls within its own Tenancy, as well
as the Tenancy of all of its ancestors.\\
Traffic Ops comes with one Tenant that is defined to be the "highest-level" or,
more commonly, the "root" Tenant. It is the special Tenant which is the only
one allowed to not have a Parent, and all tenantable objects fall within its
Tenancy. This Tenant is always Active, and has the name \code{root}.

\subsubsection{Active}
Active is a boolean concept that describes whether or not a Tenant's group of
users are allowed to actively manipulate resources. Resources within an
inactive Tenant continue to function in the ways in which they are configured,
but members of the inactive Tenant are incapable of manipulating that
configuration until they are once again "active".\\
When a Tenant is inactive, \emph{all} users within that Tenancy are inactive.
For example, if a Tenant A exists and user 1 is within that Tenant then if
Tenant A becomes inactive user 1 may no longer manipulate the objects that
fall within the A Tenancy (including A itself). If Tenant A has a child Tenant
B and some user 2 is within that Tenancy, then when A becomes inactive all of
the users in B are also affected in that they cannot manipulate any resources
that fall within the A Tenancy (including all resources that fall within the
B Tenancy). Thus the Active property of a Tenant is only considered when
determining if a Tenant is active or inactive if all of that Tenant's
ancestor Tenants are themselves active; otherwise the inactive state of an
ancestor overrides the active/inactive state of the descendant.\\
The "root" Tenant is never allowed to become inactive.

\subsubsection{Name}
The Name of a Tenant is a string that uniquely identifies it. A Tenant Name
MUST contain only alphanumeric characters, underscores, and hyphens.

\subsubsection{Parent}
A Tenant's Parent is a string which is the Name of the Tenant from which this
Tenant directly descends.\\
The Parent of a Tenant is not allowed to be either the Tenant itself, or any
descendant of that Tenant - a "descendant" being any Tenant which either has
this Tenant as its direct Parent, or has as its direct Parent a descendant of
the Tenant.\\
A tenantable object is said to fall within the Tenancy of a Tenant if it is
either assigned to the Tenant or some descendant thereof. For example, if a
Delivery Service X is within the B Tenant, and the B Tenant's Parent is the A
Tenant, and the A Tenant's Parent is the "root" Tenant, then X falls within the
Tenancy of "root", A, and B. However, if the B Tenancy is the Parent of the C
Tenant, then X is \emph{not} within the C Tenancy.\\
Only the "root" Tenant is permitted to have a "null"-typed Parent. In
particular, the "root" Tenant is defined to be the only Tenant which has a
"null"-typed Parent.

% Licensed to the Apache Software Foundation (ASF) under one
% or more contributor license agreements.  See the NOTICE file
% distributed with this work for additional information
% regarding copyright ownership.  The ASF licenses this file
% to you under the Apache License, Version 2.0 (the
% "License"); you may not use this file except in compliance
% with the License.  You may obtain a copy of the License at
%
%   http://www.apache.org/licenses/LICENSE-2.0
%
% Unless required by applicable law or agreed to in writing,
% software distributed under the License is distributed on an
% "AS IS" BASIS, WITHOUT WARRANTIES OR CONDITIONS OF ANY
% KIND, either express or implied.  See the License for the
% specific language governing permissions and limitations
% under the License.

\subsection{Topologies}
A Topology is a interconnected tree of Cache Groups that defines the flow of
content for a Delivery Service.

\subsubsection{Description}
A Topology's Description is a short string (no actual length limit) that
describes the Topology.

\subsubsection{Name}
This string uniquely identifies a Topology. It MUST only contain alphanumeric
characters and hyphens/dashes (-).

\subsubsection{Nodes}
A Topology's Nodes are an array of the Cache Groups and their relationships.
Each Node is itself an object.

\paragraph{Cache Group}
The Cache Group to which a Node refers is identified by this string, which is
its Name.

\paragraph{Parents}
The Parents of this Node are identified as an array of unsigned integers which
are the indexes in the Topology's Nodes array at which they occur. The maximum
number of parents is two, where the first element in the Parents array is
commonly referred to as the "primary parent" and the second one is referred to
as the "secondary parent".

% Licensed to the Apache Software Foundation (ASF) under one
% or more contributor license agreements.  See the NOTICE file
% distributed with this work for additional information
% regarding copyright ownership.  The ASF licenses this file
% to you under the Apache License, Version 2.0 (the
% "License"); you may not use this file except in compliance
% with the License.  You may obtain a copy of the License at
%
%   http://www.apache.org/licenses/LICENSE-2.0
%
% Unless required by applicable law or agreed to in writing,
% software distributed under the License is distributed on an
% "AS IS" BASIS, WITHOUT WARRANTIES OR CONDITIONS OF ANY
% KIND, either express or implied.  See the License for the
% specific language governing permissions and limitations
% under the License.

\subsection{Traffic Monitors}
A Traffic Monitor is a service that communicates with Traffic Ops over HTTP and
is responsible for collecting health and statistics information from Cache
Servers to determine if they are fit to service content.

\begin{codelisting}
\captionof{listing}{Traffic Monitor Object as a TypeScript Interface}
\begin{minted}[tabsize=2]{typescript}
interface TrafficMonitor {
	cdn: string;
	configPollingInterval: bigint;
	domain: string;
	eventCount: int;
	healthPollingInterval: bigint;
	heartbeatPollingInterval: bigint;
	hostName: string;
	id: bigint;
	notes: string;
	online: boolean;
	physicalLocation: string;
	tags: Set<string>;
	threadCount: bigint;
	timePad: bigint;
}
\end{minted}
\end{codelisting}

\subsubsection{CDN}
The CDN to which a Traffic Monitor belongs is represented by its Name, which
uniquely identifies it. The Monitor will monitor all Cache Servers that are in
its CDN.

\subsubsection{Config Polling Interval}
Config Polling Interval is an unsigned integer that defines an interval - in
milliseconds - on which the Monitor should poll Traffic Ops for its
configuration changes. This MUST NOT be \code{0}.

\subsubsection{Domain}
This is the "domain" part of the Traffic Monitor's Fully Qualified Domain Name
(FQDN) as a string. For example, a Traffic Monitor with an FQDN of
\code{github.com} has a Domain of \code{com} and a Traffic Monitor with an FQDN
of \code{trafficcontrol.apache.org} has a Domain of \code{apache.org}.

\subsubsection{Event Count}
This unsigned integer defines the number of "events" of which the Monitor will
keep track internally.\\
The value \code{0} means "no limit".

\subsubsection{Health Polling Interval}
This unsigned integer defines an interval - in milliseconds - on which to poll
Cache Server health. This MUST NOT be \code{0}.

\subsubsection{Heartbeat Polling Interval}
This unsigned integer defines an interval - in milliseconds - on which to poll
Cache Server "heartbeats". This poll is not meant to determine health, but
rather whether or not the Cache Server(s) may be reached over the network at
all. This MUST NOT be \code{0}.

\subsubsection{Host Name}
This is the "host" part of the Traffic Monitor's Fully Qualified Domain Name
(FQDN) as a string. For example, a Traffic Monitor with an FQDN of
\code{github.com} has a Host Name of \code{github}, and a Traffic Monitor with an
FQDN of \code{trafficcontrol.apache.org} has a Host Name of
\code{trafficcontrol}.\\
This field does NOT need to be unique, though operators are encouraged to make
this - or at least the concatenation \code{\emph{Host Name}.{Domain}} - unique
for ease of operation.

\subsubsection{ID}
A Traffic Monitor's "ID" is an unsigned integer that uniquely identifies it
among all Traffic Monitors. It serves no purpose beyond unique identification of
the Traffic Monitor.

\subsubsection{Notes}
This section is an arbitrary string containing miscellaneous, human-friendly
information about the Traffic Monitor. Other ATC components SHOULD NOT parse
this for specific information fields, or expect it to be in a particular
format.

\subsubsection{Online}
This boolean value indicates whether or not the Traffic Monitor is online and
polling Cache Servers. An "offline" (Online is false) Traffic Monitor is not
considered in peer-polling and quorums, is not queried for available Cache
Servers by Traffic Routers, and Traffic Stats Servers do not collect statistics
from offline Traffic Monitors.

\subsubsection{Physical Location}
The Physical Location at which a Traffic Monitor resides is represented by a
string containing its Name.

\subsubsection{Tags}
A Traffic Monitor's "Tags" are represented as a set of Tag Names.

\subsubsection{Thread Count}
Thread Count is an unsigned integer that determines the number of threads used
to concurrently poll Cache Servers. It is recommended that UIs built on the
Traffic Ops API set this to \code{0} by default and caution users against
changing it. The value \code{0} will cause Traffic Monitor to choose a value
for itself\footnote{A recommendation for Traffic Monitor engineers: a
reasonable default value would be the number of available CPU cores.}.

\subsubsection{Time Pad}
This is an unsigned integer that defines a set number of milliseconds to add
to request timers to help spread requests out for Traffic Control systems that
use a large number of Traffic Monitors.

% Licensed to the Apache Software Foundation (ASF) under one
% or more contributor license agreements.  See the NOTICE file
% distributed with this work for additional information
% regarding copyright ownership.  The ASF licenses this file
% to you under the Apache License, Version 2.0 (the
% "License"); you may not use this file except in compliance
% with the License.  You may obtain a copy of the License at
%
%   http://www.apache.org/licenses/LICENSE-2.0
%
% Unless required by applicable law or agreed to in writing,
% software distributed under the License is distributed on an
% "AS IS" BASIS, WITHOUT WARRANTIES OR CONDITIONS OF ANY
% KIND, either express or implied.  See the License for the
% specific language governing permissions and limitations
% under the License.

\subsection{\code{/traffic\_portals}}
This endpoint deals with manipulations and representations of Traffic Portal
objects configured in Traffic Ops.

\subsubsection{GET}
Retrieves Traffic Portal representations.
\begin{description}
	\item[Required Permissions] \code{traffic-portals-read}
	\item[Response Type] Array
\end{description}

\paragraph{Request Structure}
This method of this endpoint implements the Age Filtering and Sorting and
Pagination query parameters as outlined inSections \ref{sec:age-filtering} and
\ref{sec:pagination}, respectively. It further provides the query parameters in
Table \ref{tbl:trafficportals:get:qparams}.

\begin{table}[h]
\centering
\caption{GET \code{/traffic\_portals} Query Parameters\label{tbl:trafficportals:get:qparams}}
\begin{tabularx}{\linewidth}{|l|X|}
	\hline
	\textbf{Parameter} & \textbf{Description}\\
	\hline
	ipv4Address & Filters results to contain only the Traffic Portals with the
	              given IPv4 Address.\\
	ipv6Address & Filters results to contain only the Traffic Portals with the
	              given IPv6 Address.\\
	online      & Filters results to contain only the Traffic Portals that have
	              the provided "Online" value - valid values are "true" and
	              "false".\\
	tag         & Filters results to contain only the Traffic Portals with the
	              given Tag. Multiple tags may be given, either in regular,
	              \code{application/x-www-form-urlencoded} format like e.g.
	              \code{tag=Foo\&tag=Bar} \emph{or} as a comma-separated list,
	              e.g. \code{tag=Foo,Bar}. In either case - or even in the case
	              of mixed conventions - the filtered results will contain only
	              those Traffic Portals that have \emph{all} specified Tags.\\
	url         & Filters results to contain only the Traffic Portals with the
	              given URL. This URL must be valid, and may not contain a
	              non-root path - e.g. \code{https://example.com:443/} is
	              acceptable, whereas \code{https://example.com:443/foo} is
	              not.\\
	\hline
\end{tabularx}
\end{table}


\begin{codelisting}
\captionof{listing}{Request Example}
\begin{minted}{http}
GET /api/4.0/traffic_portals?online=true HTTP/1.1
Host: trafficops.infra.ciab.test
Accept: application/json, */*;q=0.9
Cookie: mojolicious=...

\end{minted}
\end{codelisting}

\paragraph{Response Structure}
The response is a set of representations of Traffic Portal objects, each
representation extended with the \code{lastUpdated} property containing the
Date/Time at which the Traffic Portal object was last modified.\\
This method of this endpoint also implements the \code{count} property of the
top-level \code{summary} object as described in Section
\ref{sec:summary-object}.

\begin{codelisting}
\captionof{listing}{Response Example}
\begin{minted}[tabsize=2]{http}
HTTP/1.1 200 OK
Content-Type: application/json
Server: Traffic Ops/5.0
Date: Wed, 18 Nov 2020 20:46:57 GMT
Content-Length: 237
ETag: Sample Text

{ "response": [
	{
		"ipv4Address": "192.168.240.12",
		"ipv6Address": null,
		"lastUpdated": "2009-11-10T23:00:00Z",
		"notes": "The default Traffic Portal instance for CDN-in-a-Box.",
		"online": true,
		"tags": [],
		"url": "https://trafficportal.infra.ciab.test:443/"
	}
],
"summary": {
	"count": 1
}}
\end{minted}
\end{codelisting}

\subsubsection{POST}
Creates new Traffic Portal objects.
\begin{description}
	\item[Required Permissions] \code{traffic-portals-write}
	\item[Response Type] Object
\end{description}

\paragraph{Request Structure}
The body of a POST request to \code{/traffic\_portals} is a representation of a
Traffic Portal object to be created. Only one query parameter is supported, as
shown in Table \ref{tbl:trafficportals:post:qparams}.

\begin{table}[h]
\centering
\caption{POST \code{/traffic\_portals} Query Parameters\label{tbl:trafficportals:post:qparams}}
\begin{tabularx}{\linewidth}{|l|X|}
	\hline
	\textbf{Parameter} & \textbf{Description}\\
	\hline
	lookupAddress & When this query string parameter is given, and is "true",
	                then when the request body contains a \code{null}
	                \code{ipv4Address} or \code{ipv6Address}, the value will be
	                "filled" in by looking up the host name portion of the
	                request body's \code{url} property. This may also be one of
	                the strings "ipv4Address" and "ipv6Address", to limit the
	                "filling in" behavior to the named IP address property.\\
	\hline
\end{tabularx}
\end{table}

\begin{codelisting}
\captionof{listing}{Request Example}
\begin{minted}[tabsize=2]{http}
POST /api/4.0/traffic_portals?lookupAddress=ipv4Address HTTP/1.1
Host: trafficops.infra.ciab.test
Accept: application/json, */*;q=0.9
Cookie: mojolicious=...
Content-Type: application/json
Content-Length: 197

{
	"ipv4Address": null,
	"ipv6Address": null,
	"notes": "The default Traffic Portal instance for CDN-in-a-Box.",
	"online": true,
	"tags": [],
	"url": "https://trafficportal.infra.ciab.test:443/"
}
\end{minted}
\end{codelisting}

\paragraph{Response Structure}
The response is a representation of the created Traffic Portal object, augmented
with the \code{lastUpdated} property containing the Date/Time at which the
Traffic Portal object was last modified (which should be approximately equal to
the current time).

\begin{codelisting}
\captionof{listing}{Response Example}
\begin{minted}[tabsize=2]{http}
HTTP/1.1 201 Created
Content-Type: application/json
Server: Traffic Ops/5.0
Date: Wed, 18 Nov 2020 20:46:57 GMT
Content-Length: 237
ETag: Sample Text
Location: /traffic_portals/https%3A%2F%2Ftrafficportal.infra.ciab.test%3A443

{ "response": {
	"ipv4Address": "192.168.240.12",
	"ipv6Address": null,
	"lastUpdated": "2009-11-10T23:00:00Z",
	"notes": "The default Traffic Portal instance for CDN-in-a-Box.",
	"online": true,
	"tags": [],
	"url": "https://trafficportal.infra.ciab.test:443/"
},
"alerts": [
	{
		"level": "success",
		"text": "Traffic Portal 'https://trafficportal.infra.ciab.test:443' was created."
	}
]}
\end{minted}
\end{codelisting}

\subsection{\code{/traffic\_portals/\{\{URL\}\}}}
This endpoint deals with manipulations and representations of a single Traffic
Portal object identified by the \code{URL} in the request path.

\subsubsection{GET}
Retrieves a Traffic Portal representation.
\begin{description}
	\item[Required Permissions] \code{traffic-portals-read}
	\item[Response Type] Object
\end{description}

\paragraph{Request Structure}
This method of this endpoint provides no query parameters.

\begin{codelisting}
\captionof{listing}{Request Example}
\begin{minted}{http}
GET /api/4.0/traffic_portals/https%3A%2F%2Ftrafficportal.infra.ciab.test%3A443 HTTP/1.1
Host: trafficops.infra.ciab.test
Accept: application/json, */*;q=0.9
Cookie: mojolicious=...

\end{minted}
\end{codelisting}

\paragraph{Response Structure}
The response is a representation of the requested Traffic Portal object,
extended with the \code{lastUpdated} property containing the Date/Time at which
the Traffic Portal object was last modified.

\begin{codelisting}
\captionof{listing}{Response Example}
\begin{minted}[tabsize=2]{http}
HTTP/1.1 200 OK
Content-Type: application/json
Server: Traffic Ops/5.0
Date: Wed, 18 Nov 2020 20:46:57 GMT
Content-Length: 237
ETag: Sample Text

{ "response": {
	"ipv4Address": "192.168.240.12",
	"ipv6Address": null,
	"lastUpdated": "2009-11-10T23:00:00Z",
	"notes": "The default Traffic Portal instance for CDN-in-a-Box.",
	"online": true,
	"tags": [],
	"url": "https://trafficportal.infra.ciab.test:443/"
}}
\end{minted}
\end{codelisting}

\subsubsection{PUT}
Replaces a Traffic Portal object with one provided.

\begin{description}
	\item[Required Permissions] \code{traffic-portals-write}
	\item[Response Type] Object
\end{description}

\paragraph{Request Structure}
The body of a PUT request to \code{/traffic\_portals/\{\{URL\}\}} is a
representation of a Traffic Portal object to replace the one identified in the
request path. Only one query parameter is supported, as shown in Table
\ref{tbl:trafficportals:put:qparams}.

\begin{table}[h]
\centering
\caption{POST \code{/traffic\_portals} Query Parameters\label{tbl:trafficportals:post:qparams}}
\begin{tabularx}{\linewidth}{|l|X|}
	\hline
	\textbf{Parameter} & \textbf{Description}\\
	\hline
	lookupAddress & When this query string parameter is given, and is "true",
	                then when the request body contains a \code{null}
	                \code{ipv4Address} or \code{ipv6Address}, the value will be
	                "filled" in by looking up the host name portion of the
	                request body's \code{url} property. This may also be one of
	                the strings "ipv4Address" and "ipv6Address", to limit the
	                "filling in" behavior to the named IP address property.\\
	\hline
\end{tabularx}
\end{table}

\begin{codelisting}
\captionof{listing}{Request Example}
\begin{minted}[tabsize=2]{http}
PUT /api/4.0/traffic_portals/https%3A%2F%2Ftrafficportal.infra.ciab.test%3A443?lookupAddress=ipv4Address HTTP/1.1
Host: trafficops.infra.ciab.test
Accept: application/json, */*;q=0.9
Cookie: mojolicious=...
Content-Type: application/json
Content-Length: 197
If-Unmodified-Since: Wed, 18 Nov 2020 20:46:57 GMT

{
	"ipv4Address": null,
	"ipv6Address": "::1",
	"notes": "The default Traffic Portal instance for CDN-in-a-Box.",
	"online": true,
	"tags": [],
	"url": "https://trafficportal.infra.ciab.test:443/"
}
\end{minted}
\end{codelisting}

\paragraph{Response Example}
The response is a representation of the requested Traffic Portal object, after
modifications and augmented with the \code{lastUpdated} property containing the
Date/Time at which the Traffic Portal object was last modified (which should be
approximately equal to the current time).

\begin{codelisting}
\captionof{listing}{Response Example}
\begin{minted}[tabsize=2]{http}
HTTP/1.1 200 OK
Content-Type: application/json
Server: Traffic Ops/5.0
Date: Wed, 18 Nov 2020 20:46:57 GMT
Content-Length: 237
ETag: Sample Text

{ "response": {
	"ipv4Address": "192.168.240.12",
	"ipv6Address": "::1",
	"lastUpdated": "2020-18-11T20:46:57.1Z",
	"notes": "The default Traffic Portal instance for CDN-in-a-Box.",
	"online": true,
	"tags": [],
	"url": "https://trafficportal.infra.ciab.test:443/"
},
"alerts": [
	{
		"level": "success",
		"text": "Traffic Portal https://trafficportal.infra.ciab.test:443 was updated."
	}
]}
\end{minted}
\end{codelisting}

\subsubsection{PATCH}
Modifies the identified Traffic Portal with the partial representation provided.

\begin{description}
	\item[Required Permissions] \code{traffic-portals-write}
	\item[Response Type] Object
\end{description}

\paragraph{Request Structure}
The body of a PATCH request to \code{/traffic\_portals/\{\{URL\}\}} is a partial
representation of a Traffic Portal object to overwrite the properties of the one
identified in the request path with those provided in the body.

\begin{codelisting}
\captionof{listing}{Request Example}
\begin{minted}[tabsize=2]{http}
PATCH /api/4.0/traffic_portals/https%3A%2F%2Ftrafficportal.infra.ciab.test%3A443 HTTP/1.1
Host: trafficops.infra.ciab.test
Accept: application/json, */*;q=0.9
Cookie: mojolicious=...
Content-Type: application/json
Content-Length: 24
If-None-Match: Different Sample Text

{
	"ipv6Address": null
}
\end{minted}
\end{codelisting}

\paragraph{Response Example}
The response is a representation of the requested Traffic Portal object, after
modifications and augmented with the \code{lastUpdated} property containing the
Date/Time at which the Traffic Portal object was last modified (which should be
approximately equal to the current time).

\begin{codelisting}
\captionof{listing}{Response Example}
\begin{minted}[tabsize=2]{http}
HTTP/1.1 200 OK
Content-Type: application/json
Server: Traffic Ops/5.0
Date: Wed, 18 Nov 2020 20:46:57 GMT
Content-Length: 237
ETag: Sample Text

{ "response": {
	"ipv4Address": "192.168.240.12",
	"ipv6Address": null,
	"lastUpdated": "2020-18-11T20:46:57.2Z",
	"notes": "The default Traffic Portal instance for CDN-in-a-Box.",
	"online": true,
	"tags": [],
	"url": "https://trafficportal.infra.ciab.test:443/"
},
"alerts": [
	{
		"level": "success",
		"text": "Traffic Portal https://trafficportal.infra.ciab.test:443 was updated."
	}
]}
\end{minted}
\end{codelisting}

\subsubsection{DELETE}
Deletes the identified Traffic Portal.

\begin{description}
	\item[Required Permissions] \code{traffic-portals-write}
	\item[Response Type] Object
\end{description}

\paragraph{Request Structure}
DELETE requests to \code{/traffic\_portals/\{\{URL\}\}} may have no body, nor
does this method of this endpoint provide any query parameters.

\begin{codelisting}
\captionof{listing}{Request Example}
\begin{minted}[tabsize=2]{http}
DELETE /api/4.0/traffic_portals/https%3A%2F%2Ftrafficportal.infra.ciab.test%3A443 HTTP/1.1
Host: trafficops.infra.ciab.test
Accept: application/json, */*;q=0.9
Cookie: mojolicious=...
If-Unmodified-Since: Wed, 18 Nov 2020 20:46:57 GMT

\end{minted}
\end{codelisting}

\paragraph{Response Example}
The response is a representation of the deleted Traffic Portal object, augmented
with the \code{lastUpdated} property containing the Date/Time at which the
Traffic Portal object was last modified (which should be approximately equal to
the current time).

\begin{codelisting}
\captionof{listing}{Response Example}
\begin{minted}[tabsize=2]{http}
HTTP/1.1 200 OK
Content-Type: application/json
Server: Traffic Ops/5.0
Date: Wed, 18 Nov 2020 20:46:57 GMT
Content-Length: 237
ETag: Sample Text

{ "response": {
	"ipv4Address": "192.168.240.12",
	"ipv6Address": null,
	"lastUpdated": "2020-18-11T20:46:57.3Z",
	"notes": "The default Traffic Portal instance for CDN-in-a-Box.",
	"online": true,
	"tags": [],
	"url": "https://trafficportal.infra.ciab.test:443/"
},
"alerts": [
	{
		"level": "success",
		"text": "Traffic Portal https://trafficportal.infra.ciab.test:443 was deleted."
	}
]}
\end{minted}
\end{codelisting}

% Licensed to the Apache Software Foundation (ASF) under one
% or more contributor license agreements.  See the NOTICE file
% distributed with this work for additional information
% regarding copyright ownership.  The ASF licenses this file
% to you under the Apache License, Version 2.0 (the
% "License"); you may not use this file except in compliance
% with the License.  You may obtain a copy of the License at
%
%   http://www.apache.org/licenses/LICENSE-2.0
%
% Unless required by applicable law or agreed to in writing,
% software distributed under the License is distributed on an
% "AS IS" BASIS, WITHOUT WARRANTIES OR CONDITIONS OF ANY
% KIND, either express or implied.  See the License for the
% specific language governing permissions and limitations
% under the License.

\subsection{Traffic Routers}
Traffic Routers are Traffic Router instances. They contain all of the server
information as well as configuration for the Traffic Router service.

\subsubsection{API Port}
This unsigned integer defines the port on which Traffic Router serves its HTTP
API. This MUST be less than $2^{16}$ ($65,536$).

\subsubsection{Cache Control Max Age}
This unsigned integer is the value - in seconds - used for the "max-age"
parameter of the Cache-Control HTTP header in HTTP responses to the client. If
it is "null"-typed then Traffic Router will use its default setting - which is
defined by Traffic Router and not by the Traffic Ops API.

\subsubsection{CDN}
The CDN to which a Traffic Router belongs is represented by its Name, which
uniquely identifies it. The Traffic Router MUST be an/the authoritative DNS
server for the Domain of the CDN to which it belongs.

\subsubsection{Coverage Zone Polling Interval}
This unsigned integer is the interval - in milliseconds - on which Traffic
Router will poll for an updated Coverage Zone File. A "null"-typed value causes
Traffic Router to use its default interval duration, which is not defined by the
Traffic Ops API.

\subsubsection{Deep Coverage Zone Polling Interval}
This unsigned integer is the interval - in milliseconds - on which Traffic
Router will poll for an updated Deep Coverage Zone File. A "null"-typed value
causes Traffic Router to use its default interval duration, which is not defined
by the Traffic Ops API.

\subsubsection{DNSSEC}
This property is an object, the properties of which are configuration options
for DNSSEC-related behavior.

\paragraph{Allow Expired Keys}
A boolean which, when true, will allow Traffic Router to use expired DNSSEC keys
to sign zones.

\paragraph{Effective Multiplier}
An integer that is used when creating an effective date for a new DNSSEC key
set. New keys are generated with an effective date that is the effective
multiplier multiplied by the Time to Live less than the old key's expiration
date. A "null"-typed value instructs Traffic Router to use its pre-configured
default value, which is not defined by the Traffic Ops API\footnote{At the time
of this writing, the default value is 2.}.

\paragraph{Fetch Interval}
An unsigned integer that defines the interval - in seconds - on which Traffic
Router will check the Traffic Ops API for new DNSSEC keys. A "null"-typed value
instructs Traffic Router to use its pre-configured default value - which is not
defined by the Traffic Ops API.

\paragraph{Fetch Retries}
An unsigned integer that defines the number of times Traffic Router will attempt
to load DNSSEC keys before giving up. A "null"-typed value instructs Traffic
Router to use its pre-configured default number of retries - which is not
defined by the Traffic Ops API\footnote{The default value at the time of this
writing is 5.}.

\paragraph{Fetch Timeout}
An unsigned integer that defines the timeout - in milliseconds - for requests to
the DNSSEC key management endpoint(s) of the Traffic Ops API. A "null"-typed
value indicates that Traffic Router should not time-out such requests.

\paragraph{Fetch Wait}
An unsigned integer that defines the number of milliseconds Traffic Router will
wait between attempts to load DNSSEC keys ("retries"). A "null"-typed value
instructs Traffic Router to use its pre-configured default value - which is not
defined by the Traffic Ops API.

\paragraph{Generation Multiplier}
An unsigned integer used to determine when new DNSSEC keys need to be generated.
Keys are re-generated if expiration is less than the Generation Multiplier
multiplied by the Time to Live. A "null"-typed value instructs Traffic Router to
use its pre-configured default multiplier - which is not defined by the Traffic
Ops API\footnote{The default value at the time of this writing is 10.}.

\paragraph{Zone Comparisons}
A boolean which, when true - and DNSSEC is enabled on the CDN to which this
Traffic Router belongs - allows Traffic Router to compare existing zones with
newly generated zones. If the newly generated zone is the same as the existing
zone, Traffic Router will simply re-use the existing signed zone instead of
signing the same, new zone.

\subsubsection{Domain}
This is the "domain" part of the Traffic Router's Fully Qualified Domain Name
(FQDN) as a string. For example, a Traffic Router with an FQDN of
\code{github.com} has a Domain of \code{com} and a Traffic Router with an FQDN
of \code{trafficcontrol.apache.org} has a Domain of \code{apache.org}.

\subsubsection{Dynamic Zone Cache Priming}
This property is an object whose properties are configuration options for
"Dynamic Zone Cache Priming".

\paragraph{Prime}
A boolean which, if true, will allow Traffic Router to prime the dynamic zone
cache.

\paragraph{Priming Limit}
An unsigned integer used to limit the number of permutations to prime when
"Dynamic Zone Cache Priming". A "null"-typed value instructs Traffic Router to
use its pre-configured default, which is not defined by the Traffic Ops
API\footnote{The default value is 500 at the time of this writing.}.

\subsubsection{EDNS0 Client Subnet Enabled}
A boolean that sets whether or not the EDNS0 DNS extension mechanism described
in \href{https://tools.ietf.org/html/rfc2671}{RFC2671} should be made available
to clients.

\subsubsection{Forced Diversity}
A boolean which, when true, causes Traffic Router to diversify the list of Cache
Servers returned in responses to requests for STEERING-type Delivery Services'
content by including more unique Edge-Tier Cache Servers in the response to the
client's request.

\subsubsection{Geolocation Polling Interval}
An unsigned integer number of milliseconds that Traffic Router will use as an
interval on which to poll for an geographic IP mapping database. A "null"-typed
value will instruct Traffic Router to use its pre-configured default value -
which is not defined by the Traffic Ops API.

\subsubsection{Host Name}
This is the "host" part of the Traffic Router's Fully Qualified Domain Name
(FQDN) as a string. For example, a Traffic Router with an FQDN of
\code{github.com} has a Host Name of \code{github}, and a Traffic Router with an
FQDN of \code{trafficcontrol.apache.org} has a Host Name of
\code{trafficcontrol}.\\
This field does NOT need to be unique, though operators are encouraged to make
this - or at least the concatenation \code{\emph{Host Name}.{Domain}} - unique
for ease of operation.

\subsubsection{ID}
A Traffic Router's "ID" is an unsigned integer that uniquely identifies it among
all Traffic Routers. It serves no purpose beyond unique identification of the
Traffic Router.

\subsubsection{Notes}
This section is an arbitrary string containing miscellaneous, human-friendly
information about the Traffic Router. Other ATC components SHOULD NOT parse this
for specific information fields, or expect it to be in a particular format.

\subsubsection{Online}
This boolean value indicates if this Traffic Router is online and capable of
routing clients to Cache Servers etc. An "offline" (Online is false) Traffic
Router is excluded from responses to Traffic Ops API endpoints that serve as
proxies for the Traffic Router API in some capacity.

\subsubsection{Physical Location}
The Physical Location at which a Traffic Router resides is represented by a
string containing its Name.

\subsubsection{SOA}
This property is an object that contains configuration information for the
"Start of Authority" (SOA) records served by the Traffic Router.

\paragraph{Admin}
An email address for the administrator of the DNS zones for which Traffic Router
is authoritative. This string may be "null"-typed to indicate such an address is
unavailable.

\paragraph{Expire}
An unsigned integer number of seconds which will be used in the "expire" field
the Traffic Router DNS Server will respond with on SOA records.

\paragraph{Minimum}
An unsigned integer number of seconds which will be used as the value for the
"minimum" field the Traffic Router DNS Server will respond with on SOA records.

\paragraph{Refresh}
An unsigned integer number of seconds which will be used as the value for the
"refresh" field the Traffic Router DNS Server will respond with on SOA records.

\paragraph{Retry}
An unsigned integer number of seconds which will be used as the value for the
"retry" field the Traffic Router DNS Server will respond with on SOA records.

\subsubsection{Secure API Port}
This unsigned integer defines the port on which Traffic Router serves its HTTPS
API. This MUST be less than $2^{16}$ ($65,536$).

\subsubsection{Steering Polling Interval}
An unsigned integer number of seconds on which Traffic Router will check for new
steering mappings. A "null"-typed value instructs Traffic Router to use its
pre-configured default interval - which is not defined by the Traffic Ops API.

\subsubsection{TTLs}
This property is a map of record types (in lower-case) to unsigned integers that
are the Time-to-Live values - in seconds - that Traffic Router will use for the named
DNS record type.\\
The meaningful keys in this map are (but are not limited to for the forseeable
future):

\begin{itemize}
	\item a
	\item aaaa
	\item dnskey
	\item ds
	\item ns
	\item soa
\end{itemize}

\subsubsection{Zones}
The Zones of a Traffic Router is a property that is itself an object having
properties which express configuration options for DNS zone management.

\paragraph{Cache Maintenance Interval}
This unsigned integer defines an interval - in seconds - on which Traffic Router
will check for zones that need to be re-signed or if dynamic zones need to be
expired from its cache. If it "null"-typed, Traffic Router is instructed to use
its pre-configured default interval, which is not defined by the Traffic Ops
API.

\paragraph{Dynamic Initial Capacity}
This unsigned integer defines an initial size to be used by the Traffic Router's
zone manager's cache. If this is "null"-typed, Traffic Router is instructed to
use its pre-configured default value, which is not defined by the Traffic Ops
API\footnote{At the time of this writing the default size is $10,000$.}.

\paragraph{Dynamic Response Expiration}
This unsigned integer defines how long - in seconds - Traffic Router will allow
a dynamic zone to remain in cache after its last access. If it is "null"-typed,
Traffic Router is instructed to use its pre-configured default duration, which
is not defined by the Traffic Ops API\footnote{At the time of this writing the
default duration is $3,600$.}.

\paragraph{Initialization Timeout}
This unsigned integer sets a cap - in minutes - on the amount of time Traffic
Router will spend initializing its dynamic zone manager. Larger numbers impact
startup time, but allow for more zones to be pre-computed. If it is "null"-typed
Traffic Router is instructed to use its pre-configured default timeout, which is
not defined by the Traffic Ops API\footnote{At the time of this writing the
default timeout is $10$.}.

\paragraph{Thread Count}
This unsigned integer sets a number of concurrent threads for the Traffic
Router's zone manager to operate on. The value 0 is not allowed. If this is
"null"-typed, the number of threads is calculated based on the number of
allowed/available CPU cores as defined by Thread Multiplier.

\paragraph{Thread Multiplier}
This floating point number is multiplied by the number of available CPU cores on
the Traffic Router to calculate the number of threads the zone manager will use,
rounded down to the nearest integer. For example, a value of $1.0$ will instruct
Traffic Router to use a thread for every available CPU core.\\
This has no effect if Thread Count is not "null"-typed. The value MUST be
greater than zero. If this is "null"-typed (and Thread Count is not), Traffic
Router will use its pre-configured default multiplier, which is not defined by
the Traffic Ops API\footnote{At the time of this writing, the default multiplier
is $0.75$.}.

% Licensed to the Apache Software Foundation (ASF) under one
% or more contributor license agreements.  See the NOTICE file
% distributed with this work for additional information
% regarding copyright ownership.  The ASF licenses this file
% to you under the Apache License, Version 2.0 (the
% "License"); you may not use this file except in compliance
% with the License.  You may obtain a copy of the License at
%
%   http://www.apache.org/licenses/LICENSE-2.0
%
% Unless required by applicable law or agreed to in writing,
% software distributed under the License is distributed on an
% "AS IS" BASIS, WITHOUT WARRANTIES OR CONDITIONS OF ANY
% KIND, either express or implied.  See the License for the
% specific language governing permissions and limitations
% under the License.

\subsection{Traffic Stats Servers}
Traffic Stats Servers are instances of Traffic Stats and contain information
relating to their network connection information as well as configuration.

\subsubsection{Database Servers}
The databases used by Traffic Stats to store statistics are given as a set of
unsigned integers that are Infrastructure Server IDs.

\subsubsection{Domain}
The "domain" part of the Traffic Stats Server's Fully Qualified Domain Name
(FQDN) as a string. For example, a Traffic Stats Server with an FQDN of
\code{github.com} has a Domain of \code{com} and a Traffic Stats Server with an
FQDN of \code{trafficcontrol.apache.org} has a Domain of \code{apache.org}.\\
In other words, this is every part of the Traffic Stats Server's FQDN that
isn't part of its Host Name.

\subsubsection{Host Name}
This is the "host" part of the Traffic Stats Server's Fully Qualified Domain
Name (FQDN) as a string. For example, a Traffic Stats Server with an FQDN of
\code{github.com} has a Host Name of \code{github}, and a Traffic Stats Server
with an FQDN of \code{trafficcontrol.apache.org} has a Host Name of
\code{trafficcontrol}.\\
In other words, a Traffic Stats Server's Host Name is the only part of its FQDN
that is not part of its Domain.\\
This field does NOT need to be unique, though operators are encouraged to make
this - or at least the concatenation \code{\emph{Host Name}.\emph{Domain}} -
unique for ease of operation.

\subsubsection{ID}
An integral, unique identifier for the Traffic Stats Server. It carries no
meaning or significance beyond being a unique identifier.

\subsubsection{Notes}
This is a string of arbitrary text for miscellaneous purposes.

\subsubsection{Physical Location}
This string gives the Name of a Physical Location within which this Traffic
Stats Server resides. If this is "null"-typed value, then it is understood that
the Traffic Stats Server has no well-defined and/or meaningful Physical
Location.

% Licensed to the Apache Software Foundation (ASF) under one
% or more contributor license agreements.  See the NOTICE file
% distributed with this work for additional information
% regarding copyright ownership.  The ASF licenses this file
% to you under the Apache License, Version 2.0 (the
% "License"); you may not use this file except in compliance
% with the License.  You may obtain a copy of the License at
%
%   http://www.apache.org/licenses/LICENSE-2.0
%
% Unless required by applicable law or agreed to in writing,
% software distributed under the License is distributed on an
% "AS IS" BASIS, WITHOUT WARRANTIES OR CONDITIONS OF ANY
% KIND, either express or implied.  See the License for the
% specific language governing permissions and limitations
% under the License.

\subsection{Traffic Vaults}
A Traffic Vault is a service for storing "secrets" - which is a general term
mainly used to encompass encryption private keys. URL signing keys, URI
signature keys, SSL keys, and DNSSEC keys are all stored within Traffic Vaults.

% \subsubsection{Domain\label{sec:server:domain}}
% This is the "domain" part of the Traffic Vault's Fully Qualified Domain Name
% (FQDN) as a string. For example, a Traffic Vault with an FQDN of
% \code{github.com} has a Domain of \code{com} and a Traffic Vault with an FQDN of
% \code{trafficcontrol.apache.org} has a Domain of \code{apache.org}.\\
% In other words, this is every part of the Traffic Vault's Fully Qualified Domain
% Name that isn't part of its Host Name as defined in Section
% \ref{sec:server:hostname}.

% \subsubsection{Host Name\label{sec:server:hostname}}
% This is the "host" part of the Traffic Vault's Fully Qualified Domain Name
% (FQDN) as a string. For example, a Traffic Vault with an FQDN of
% \code{github.com} has a Host Name of \code{github}, and a Traffic Vault with an
% FQDN of \code{trafficcontrol.apache.org} has a Host Name of
% \code{trafficcontrol}.\\
% In other words, a Traffic Vault's Host Name is the only part of its FQDN that
% is not part of its Domain as defined in Section \ref{sec:server:domain}.\\
% This field does NOT need to be unique, though operators are encouraged to make
% this - or at least the concatenation \code{\emph{Host Name}.\emph{Domain}}
% - unique for ease of operation.

\subsubsection{FQDN}
A Traffic Vault's FQDN is a string that is a valid DNS name that uniquely
identifies it among Traffic Vaults. It is also the DNS name at which its service
may be requested.

\subsubsection{Notes}
A string of arbitrary text for miscellaneous purposes.

\subsubsection{Online}
This boolean value indicates whether or not the Traffic Vault is online and
capable of storing/retrieving secrets. Traffic Ops will not use an "offline"
(Online is false) Traffic Vault to either store or retrieve stored
secrets\footnote{This means that it's possible - depending on implementation -
for an offline Traffic Vault to become out-of-sync with its peers. Operators
should take steps to ensure that Traffic Vaults brought online have up-to-date
secrets to serve to Traffic Ops.}.

% Licensed to the Apache Software Foundation (ASF) under one
% or more contributor license agreements.  See the NOTICE file
% distributed with this work for additional information
% regarding copyright ownership.  The ASF licenses this file
% to you under the Apache License, Version 2.0 (the
% "License"); you may not use this file except in compliance
% with the License.  You may obtain a copy of the License at
%
%   http://www.apache.org/licenses/LICENSE-2.0
%
% Unless required by applicable law or agreed to in writing,
% software distributed under the License is distributed on an
% "AS IS" BASIS, WITHOUT WARRANTIES OR CONDITIONS OF ANY
% KIND, either express or implied.  See the License for the
% specific language governing permissions and limitations
% under the License.

\subsection{Users\label{sec:users}}
A user represents a person who may be authenticated with the Traffic Ops API and
their ability to interact with Traffic Ops Data Model objects. A User object
also contains "meta" information such as their email address.

\begin{codelisting}
\captionof{listing}{User Object as a TypeScript Interface}
\begin{minted}[tabsize=2]{typescript}
interface User {
	address: string;
	email: string;
	fullName: string;
	phoneNumber: string;
	role: string;
	tags: Set<string>;
	tenant: string;
	username: string;
}
\end{minted}
\end{codelisting}

\subsubsection{Address}
This is a text field with no defined structure, but semantically it represents a
user's real-world, physical address. If it is not an empty value, it is assumed
to contain enough information to send a letter through normal postage to the
user.\\
Addresses may consist of alphanumeric characters, hyphens, periods, spaces, and
newlines, but may not end with a space or newline.

\subsubsection{Email}
A user's email address, which is used for initial registration and password
recovery. A User object is not guaranteed to have a non-empty Email because the
initial, default User will not have one. However, as Emails are unique among all
users, all subsequently created Users \emph{should} have non-empty Emails.\\
Representation formats that do not support email addresses natively as a type
MUST represent this a string.

\subsubsection{Full Name}
A user's "full" name, as it would appear on a letter mailed to them, as a
string. This field, if not empty, is presumed to be the user's name as it would
appear on normal postage. This field MUST NOT be allowed to contain
non-alphabetic characters.

\subsubsection{Phone Number}
A string which, if not empty, is presumed to be a telephone number at which the
user may be contacted. It MUST only be allowed to contain numerics and hyphens,
and must neither start nor end with a hyphen.

\subsubsection{Role}
A user's permissions are encapsulated by their Role, which is represented as a
map containing two keys, "name" that maps to the Role's Name as a string and
"permissions" which maps to a set of Permission Names as strings.

\subsubsection{Tags}
The Tags associated with a User are represented by a set of strings that are
Tag Names.\\
Note that Tags \emph{cannot be used to control a User's access to or ability to
interact with \textbf{anything}}. These are controlled entirely by a User's
Tenant and Role.

\subsubsection{Tenant}
The scope of a user's permissions is expressed as a string that is the Name of
the Tenant to which the user belongs.

\subsubsection{Username}
A user's "Username" is a string that is used to uniquely identify them among all
Users. It is only allowed to contain alphanumeric characters.

