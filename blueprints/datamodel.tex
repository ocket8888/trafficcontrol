% Licensed to the Apache Software Foundation (ASF) under one
% or more contributor license agreements.  See the NOTICE file
% distributed with this work for additional information
% regarding copyright ownership.  The ASF licenses this file
% to you under the Apache License, Version 2.0 (the
% "License"); you may not use this file except in compliance
% with the License.  You may obtain a copy of the License at
%
%   http://www.apache.org/licenses/LICENSE-2.0
%
% Unless required by applicable law or agreed to in writing,
% software distributed under the License is distributed on an
% "AS IS" BASIS, WITHOUT WARRANTIES OR CONDITIONS OF ANY
% KIND, either express or implied.  See the License for the
% specific language governing permissions and limitations
% under the License.

\section{Data Model\label{sec:data-model}}
This section defines the various objects that may be interacted with through the
Traffic Ops API and their respective properties.\\
An object's properties may principally be of several different types:

\begin{itemize}
	\item Integers. Integers may be signed or unsigned, to signify whether or
	not they are capable of being less than zero. Representation formats which
	are incapable of distinguishing integral and floating-point numbers (e.g.
	JSON) or of separately describing signed and unsigned types (e.g. JSON
	again) MUST represent these as numbers. In particular, they MUST NOT be
	represented as string types. Furthermore, they MUST NOT be represented with
	a non-zero decimal component and generally SHOULD NOT be represented with a
	decimal part at all, if possible. Unsigned quantities MUST NOT include a
	sign (even \code{+}, were it supported by the format).

	\item Floating-point numbers. Floating point numbers are guaranteed to have
	a precision of at least 64-bits, but clients SHOULD NOT treat
	representations with a greater apparent precision as if they were more
	precise than 64 bits can offer. Endpoints MUST NOT implement Not-a-Number
	nor either positive or negative Infinity values as floating-point numbers,
	even if the representation format supports them.

	\item Strings. Strings SHOULD always be encoded (along with the rest of the
	representation, if possible) in UTF-8.

	\item Dates/times. As discussed in section \ref{sec:datetime}, these may be
	expressed as either strings in RFC3339 format or as a signed integer number
	of nanoseconds since the Unix epoch if the representation format does not
	support date/time types natively (e.g. JSON).

	\item Booleans. Boolean concepts MUST be represented as boolean data types
	in representations that do support them. For example, in JSON,
	\code{\{"foo":true\}} is acceptable but \code{\{"foo":"true"\}},
	\code{\{"foo":1\}} and especially \code{\{"foo":"1"\}} are all unacceptable.
	In representation formats that do not support boolean types, they may be
	represented as the number zero for "false" and the number one for "true" as
	unsigned integers.

	\item Collections. There are several collection types used to model ATC
	data. Commonly, they all may contain arbitrary data types, but SHOULD NOT
	contain data of more than one type in any representation format.

	\begin{itemize}
		\item Sets. A set is an unordered collection of unique items. The
		contents of a set are not guaranteed to be in any particular order, and
		clients SHOULD NOT expect anything else\footnote{Endpoint authors are,
		however, encouraged to order these consistently for ease of use,
		testing, and debugging. This order need not - and indeed SHOULD NOT be
		documented.} (unless the endpoint provides sorting functionality
		expressly), and in particular SHOULD NOT conclude that a difference in
		ordering indicates that data has changed. If a representation format is
		not capable of expressing set types (e.g. JSON), then sets may be
		expressed as a list array or other more generic collection, but MUST
		NEVER contain duplicates.

		\item Lists/Arrays. A list/array is an \emph{ordered} collection of
		data, and as such the ordering MUST be consistent in representations, as
		well as meaningful.

		\item Maps. A map may be used to associate any non-collection type "key"
		with a "value" of arbitrary type. Maps MUST NOT contain duplicate keys
		in any representation format. If a representation format is not capable
		of representing maps (e.g. JSON), then they may be represented as
		objects where the keys are coerced to property names and their values
		the respective values of those properties.
	\end{itemize}
\end{itemize}

With the "fallback" behavior described for each type above, note that valid
representation formats MUST be capable of representing at least floating-point
values, string values, array/lists of such primitive values, and either nested
object structures or maps.

\subsection{CDNs}
A CDN is a full collection of caching and infrastructural servers and ATC
components required to service distributed requests for content, as well as the
configuration options for the methods of content delivery and the content
delivered therein.

\subsubsection{Cache Servers}
A CDN's collection of Cache Servers is represented as set of Cache Server IDs.

\subsubsection{Delivery Services}
A CDN's collection of Delivery Services is represented as a set of Delivery
Service Names.

\subsubsection{DNSSEC Enabled}
A CDN that has "DNSSEC Enabled" is capable of supporting DNSSEC configurations
on its constituent Delivery Services. This does \emph{not} mean that all DNS
requests through ATC for this CDN's Domain will be secured with DNSSEC.\\
This is a boolean concept and as such is represented by boolean values.

\subsubsection{DNSSEC Keys}
% TODO: is this necessary? or is this more of a Delivery Service thing?
An array/list of the Zone-Signing and Key-Signing Keys of this CDN. The
Zone-Signing Key will appear first, followed immediately by the Key-Signing Key

\subsubsection{Domain}
The Top-Level Domain used by the CDN, represented as a string. This MUST be a
uniquely valued property across all CDNs, and MUST not be an invalid DNS label
as defined by
\href{https://tools.ietf.org/html/rfc1035#section-2.3.1}{Section 2.3.1 of RFC1035}.

\subsubsection{Infrastructure Servers}
A CDN's collection of Infrastructure Servers is represented as a set of
Infrastructure Server IDs.

\subsubsection{Name}
A CDN's "Name" is a string that uniquely identifies it among all CDNs. It MUST
NOT be allowed to contain "special characters" - meaning anything other than
alphanumeric characters, spaces, hyphens, colons and underscores. It further
MUST NOT be allowed to begin with a non-alphanumeric character nor end with a
space.

\subsubsection{Origins}
A CDN's collection of Origins is represented by a set of Origin IDs.

\subsubsection{Traffic Monitors}
A CDN's collection of Traffic Monitors is represented by a set of Traffic
Monitor IDs. Note that while a functioning CDN must contain at least one Traffic
Monitor, no such restriction is placed on CDN objects in general.

\subsubsection{Traffic Portals}
A CDN's collection of Traffic Portal is represented by a set of Traffic Portal
ID's.

\subsubsection{Traffic Routers}
A CDN's collection of Traffic Routers is represented by a set of Traffic Router
IDs. Note that while a functioning CDN must contain at least one Traffic Router,
no such restriction is placed on CDN objects in general.

\subsubsection{Traffic Stats Servers}
A CDN's collection of Traffic Stats Servers is represented by a set of Traffic
Stats Server IDs.

\subsubsection{Traffic Vaults}
A CDN's collection of Traffic Vaults is represented by a set of Traffic Vault
IDs.



\subsection{Cache Groups}



\subsection{Cache Servers}
Cache servers are the Mid-and-Edge-tier HTTP caching proxies ultimately
responsible for caching and serving content for a Delivery Service.

\subsubsection{Bandwidth Threshold}
The "Bandwidth Threshold" of a Cache Server is the maximum used bandwidth which
can be considered "healthy" for the Cache Server. It is suggested that clients
presenting controls to create/register a new Cache Server present the most
common value within the server's Cache Group as the default.

\subsubsection{Cache Group}
The Cache Group to which the Cache Server belongs is represented as a string
that uniquely names it.

\subsubsection{Capabilities}
A Cache Server's "Capabilities" express a Cache Server's ability to serve
certain kinds of traffic. They are represented on Cache Server objects as a set
of Capability Names.

\subsubsection{CDN}
The CDN to which a Cache Server belongs is represented by its Name, which
uniquely identifies it.

\subsubsection{Delivery Services}
The Delivery Services for which a Cache Service is tasked with serving content
are represented by a set of Delivery Service names. Note that such assignments
do not mean that a Cache Server is capable of serving such content, which truly
depends on whether or not each Delivery Service is Primed and Active.

\subsubsection{Domain}
This is the "domain" part of the Cache Server's Fully Qualified Domain Name
(FQDN) as a string. For example, a Cache Server with an FQDN of
\code{github.com} has a Domain of \code{com} and a Cache Server with an FQDN of
\code{trafficcontrol.apache.org} has a Domain of \code{apache.org}.

\subsubsection{Hard Disks}
A collection of Hard Disk Drives (HDDs)\footnote{This designation is meant to
contrast with "RAM Drives" where a block device maps to a section of main
memory. In practice, Hard Disks may be any block device, include Solid-State
Drives rather than actually HDDs.} available to the Cache Server for caching
content, represented as a set of file paths as strings to said devices. These
devices MUST be accessible for full read and write operations to the user as
whom the caching proxy software runs on the Cache Server.

\subsubsection{Health Polling Path}
This is a string that names a special request path recognized by the Cache
Server's caching proxy software which will result in the retrieval of statistics
rather than content. This MUST NOT be allowed to be an empty string. It MAY
contain a query string or even document fragment.\\
The only health-and-statistics-gathering implementation truly supported by ATC
is the special "astats" Traffic Server plugin.

\subsubsection{Host Name}
This is the "host" part of the Cache Server's Fully Qualified Domain Name (FQDN)
as a string. For example, a Cache Server with an FQDN of \code{github.com} has a
Host Name of \code{github}, and a Cache Server with an FQDN of
\code{trafficcontrol.apache.org} has a Host Name of \code{trafficcontrol}.\\
This field does NOT need to be unique, though operators are encouraged to make
this - or at least the concatenation \code{\emph{Host Name}.{Domain}} - unique
for ease of operation.

\subsubsection{HTTP Port}
An unsigned number that designates the port on which this Cache Server listens
for incoming HTTP requests. If this is a "null"-type then the Cache Server is
assumed to be incapable of serving unsecured HTTP requests.\\
Note that if both HTTP Port and HTTPS Port are "null"-type, then ATC
infrastructure will assume the Cache Server is incapable of serving traffic, and
treat it accordingly - despite that most clients will attempt to send requests
to the default 80 and 443 ports unless otherwise instructed.

\subsubsection{HTTPS Port}
An unsigned number that designates the port on which this Cache Server listens
for incoming HTTPS requests. If this is a "null"-type then the Cache Server is
assumed to be incapable of serving secured HTTPS requests.\\
Note that if both HTTP Port and HTTPS Port are "null"-type, then ATC
infrastructure will assume the Cache Server is incapable of serving traffic, and
treat it accordingly - despite that most clients will attempt to send requests
to the default 80 and 443 ports unless otherwise instructed.

\subsubsection{ID}
A Cache Server's "ID" is an unsigned integer that uniquely identifies it among
all Cache Servers. It serves no purpose beyond unique identification of the
Cache Server.

\subsubsection{IPv4 Address}
The Cache Server's IPv4 address. If it is a "null"-type then the Cache Server is
assumed to only operate in IPv6.\\
Note that if both a Cache Server's IPv4 Address and IPv6 Address are "null"-type
then clients cannot and will not be routed to them for DNS-routed Delivery
Services.

\subsubsection{IPv6 Address}
The Cache Server's IPv6 address. If it is a "null"-type then the Cache Server is
assumed to only operate in IPv6.\\
Note that if both a Cache Server's IPv4 Address and IPv6 Address are "null"-type
then clients cannot and will not be routed to them for DNS-routed Delivery
Services.

\subsubsection{Loadavg Threshold}
A Cache Server's "Loadavg Threshold" is a floating-point number - which MUST be
greater than 0 - that represents the loadavg above which the Cache Server will
be considered unhealthy. For more information consult the
\href{https://linux.die.net/man/3/getloadavg}{\code{getloadavg} manual page}.

\subsubsection{Notes}
This section is an arbitrary string containing miscellaneous, human-friendly
information about the Cache Server. Other ATC components SHOULD NOT parse this
for specific information fields, or expect it to be in a particular format.

\subsubsection{Physical Location}
The Physical Location at which a Cache Server resides is represented by a string
containing its Name.

\subsubsection{Profile}
The Profile used by a Cache Server is represented by a string containing its
Name.

\subsubsection{Query Time Threshold}
A Cache Server's "Query Time Threshold" is an unsigned integer number of
milliseconds after which a Traffic Monitor polling it for health will consider
it unhealthy, even if it successfully returns healthy metrics. The value "0" has
the meaning "no threshold" (however, Traffic Monitor polling timeouts still
apply).

\subsubsection{RAM Disks}
A collection of RAM Disk Drives available to the Cache Server for caching
content, represented as a set of file paths as strings to said devices. These
devices MUST be accessible for full read and write operations to the user as
whom the caching proxy software runs on the Cache Server.

\subsubsection{Revalidation Pending}
This boolean represents whether or not the Cache Server has content revalidation
requests yet to satisfy. When a new Content Revalidation Request is created on
one or more Delivery Services to which this Cache Server is assigned, this will
be set to "true" and updated to "false" when the operation has been completed.
Being "false" does not mean that the content invalidation request(s) performed
by the Cache Server have expired and are no longer in effect.

\subsubsection{Status}
The Cache Server's "Status" is a string constant, which MUST always be one of

\begin{itemize}
	\item \code{ONLINE} - The Cache Server will always be considered healthy
	regardless of any thresholds or connectivity state.
	\item \code{REPORTED} - The Cache Server's health is presented to the
	Traffic Router(s) as it is reported by its various thresholds, as determined
	by the Traffic Monitor(s).
	\item \code{OFFLINE} - The Cache Server is considered unhealthy regardless
	of any thresholds or connectivity state.
	\item \code{ADMIN\_DOWN} - The Cache Server is considered unhealthy and its
	thresholds and connectivity state are not monitored. Its existence is not
	disclosed to Traffic Router(s).
\end{itemize}

\subsubsection{Tags}
A Cache Server's "Tags" are represented as a set of Tag Names.

\subsubsection{Type}
A Cache Server's "Type" is a string constant, which MUST always be one of

\begin{itemize}
	\item \code{EDGE} - This is an Edge-tier Cache Server which acts as a
	reverse proxy.
	\item \code{MID} - This is an Edge-tier Cache Server which acts as a forward
	proxy.
\end{itemize}

\subsubsection{Updates Pending}
This boolean represents whether or not the Cache Server has configuration
updates pending. When such updates are applied, this will be set to "false" by
the Cache Server's ORT/ATSTCCFG instance.


\subsection{Capabilities}


\subsection{Delivery Services}
Delivery Services are, at their most basic, an association between a source of
content and a set of Cache Servers and configuration options used to distribute
that content.

\subsubsection{Common Properties}
Herein described are the properties common to all Delivery Service objects. The
Routing Type of a Delivery Service encapsulates the methods by which clients
may request content routing, and depending on its value the Delivery Service
takes on a set of additional properties. Put simply, these are all different
types of objects that are closely related.\\
This section details all of the properties that are common to \emph{all} types
of Delivery Services.

\paragraph{Anonymous Blocking}
A Delivery Service that has "Anonymous Blocking" tells Traffic Router to block
requests from anonymized IP addresses. Whether or not and how well that can
actually be done is dependent on the configuration of each Traffic Router
itself, and if this Delivery Service is DNS-routed the only IP address Traffic
Routers will be capable of checking for anonymization (e.g. known proxy/VPN/TOR
exit node) will be the downstream router requesting the name resolution and thus
is likely much less effective.

\paragraph{Bypass Destination}
This is a string that describes the network location to which clients will be
directed if the traffic served by this Delivery Service exceeds its allowed
maximums. This MUST always be represented as a string - even if the
representation format supports IP Addresses as a native type - and its
interpretation is dependent on the Delivery Service's Routing Type.\\
If the Delivery Service's Routing Type is \code{HTTP}, then this is interpreted
- and validated by the API - as a Fully Qualified Domain Name optionally
followed by a colon and port number that defines an HTTP server to which client
requests will be directed.\\
If the Delivery Service's Routing Type is \code{DNS}

\paragraph{Deep Caching}
A boolean value that describes whether or not "Deep Caching" may be used for
this Delivery Service.

\paragraph{Caching}
\begin{itemize}
	\item \code{CACHE}
	\item \code{RAM\_ONLY}
	\item \code{NO\_CACHE}
\end{itemize}

%TODO - STEERING doesn't have this?
\paragraph{Cache Servers}
A Delivery Service's assigned Cache Servers are expressed as a set of Cache
Server IDs.

\paragraph{CDN}
The CDN to which a Delivery Service belongs is expressed as a string that is the
Name used to uniquely identify it.

%TODO: use-case for check path?

\paragraph{Denied Access Redirect}
This is a string that describes the network location to which clients will be
directed if they are denied access on the basis of Anonymous Blocking and/or
Geographic Limiting settings. This MUST always be represented as a string - even
if the representation format supports IP Addresses as a native type - and its
interpretation is dependent on the Delivery Service's Routing Type.

\paragraph{DSCP}

\paragraph{ECS}

\paragraph{Geographic Limiting}
This property describes limitations to the availability of this Delivery
Service's content on the basis of the requesting client's geographic location.
It is a set of strings, each of which is an
\href{https://www.iso.org/obp/ui/#search/code/}{ISO 3166-1} alpha-2 country
code, optionally with ISO 3166-2 subdivisional alphabetic code. This is a "white
list" of countries/subdivisions wherein content is to be made
available\footnote{This property is meant to inform Traffic Router; Cache
Servers cannot be relied upon to approve or deny access on a geographic basis.
Thus, if routing is bypassed, restricted content is totally accessible to
requesting clients.}.\\
Content is \emph{always} available to clients whose IP addresses are found
within the Traffic Routers' Coverage Zone File(s). With that in mind, when this
property is an empty set it means that no geographic regions are "whitelisted"
and thus \emph{only} clients whose IP addresses are found within a Coverage Zone
File will be granted access to content. When this property has a "Null" type,
there is no geographic restriction placed on the Delivery Service's content
access.

\paragraph{Maximum Origin Connections}

%TODO: use-case for header-rewrite text?

\paragraph{Miss Location}

\paragraph{Name}
A Delivery Service's "Name" is a string that uniquely identifies it among all
Delivery Services. It MUST only contain alphanumerics, hyphens, underscores and
spaces, and MUST NOT begin with a non-alphanumeric character nor end with a
non-alphanumeric character. This is used to generate part of the default request
hostnames by replacing all non-alphanumeric characters with a hyphen.

\paragraph{Notes}
This section is an arbitrary string containing miscellaneous, human-friendly
information about the Delivery Service. Other ATC components SHOULD NOT parse
this for specific information fields, or expect it to be in a particular format.

%TODO: MSO use-case?
\paragraph{Origin}

\paragraph{Query String Handling}

\begin{itemize}
	\item \code{DROP}
	\item \code{IGNORE}
	\item \code{USE}
\end{itemize}

\paragraph{Range Request Handling}
\begin{itemize}
	\item \code{NO\_CACHE}
	\item \code{WHOLE\_OBJECT}
	\item \code{CACHE}
\end{itemize}

%TODO Raw-Remap use-case?
%TODO Regex-Remap use-case?

\paragraph{Required Capabilities}

\paragraph{Routing Name}

%TODO: ANY_MAP use-case?
\paragraph{Routing Type}
After creation, Routing Type is a read-only property.
\begin{itemize}
	\item \code{HTTP}
	\item \code{DNS}
	\item \code{STEERING}
	\item \code{STATIC}
\end{itemize}

\paragraph{Skip Mid Tier}

\paragraph{Status}
The "Status" of a Delivery Service is a string constant that expresses its
ability to serve content at the present moment in time. It may have one of three
values:

\begin{itemize}
	\item \code{ACTIVE} A Delivery Service that is "active" is one that is
	functionally in service, and fully capable of delivering content. This means
	that its configuration is deployed to Cache Servers and it is available for
	routing traffic.
	\item \code{PRIMED} A Delivery Service that is "primed" has had its
	configuration distributed to the various servers required to serve its
	content. However, the content itself is still inaccessible\footnote{The
	content is not available through normal routing. This does not, though,
	guarantee that Cache Servers do not already have the content stored and/or
	are incapable of serving it if routing is bypassed.}.
	\item \code{INACTIVE} A Delivery Service that is "inactive" is not available
	for routing and has not had its configuration distributed to its assigned
	Cache Servers.
\end{itemize}

\paragraph{Supported Protocols}
This is a set of strings that name protocols served by the Delivery Service.
Note that this is the method used to retrieve content from the caching system,
not the method used for routing. The only protocols officially supported by ATC
are "HTTP" and "HTTPS".\\
This set is case-insensitive, such that if a Delivery Service is created with a
Supported Protocols set containing "HTTP" the resulting set is equivalent to
what would result from creating it with a Supported Protocols set containing
"http". Representations produced by the Traffic Ops API MUST always use
only uppercase characters.

\paragraph{Tenant}
The Tenant to which a Delivery Service belongs is represented by a string that
is that Tenant's unique Name.

\paragraph{Vanity Hostnames}

\subsubsection{DNS-Routed Properties}

\paragraph{Bypass TTL}
An unsigned integer that defines the Time-To-Live (TTL) of DNS responses from
the Traffic Router for this Delivery Service's Bypass Destination, in seconds.

\paragraph{DNS TTL}
An unsigned integer that defines the Time-To-Live (TTL) of DNS responses from
the Traffic Router for this Delivery Service's routing, in seconds.

\paragraph{Max Records}

\subsubsection{HTTP-Routed Properties}

\paragraph{Additional Response Headers}

\paragraph{Consistent Hashing Regular Expression}

\paragraph{Logged Request Headers}

\paragraph{Significant Query Parameters}

\subsubsection{Steering-Routed Properties}

\paragraph{Targets}



\subsection{Infrastructure Servers}



\subsection{Origins}



\subsection{Profiles and Parameters}



\subsection{Physical Locations}



\subsection{Roles and Permissions\label{sec:roles-and-perms}}



\subsection{Tags}



\subsection{Tenants}



\subsection{Traffic Monitors}



\subsection{Traffic Portals}



\subsection{Traffic Routers}



\subsection{Traffic Stats Servers}



\subsection{Traffic Vaults}



\subsection{Users\label{sec:users}}
A user represents a person who may be authenticated with the Traffic Ops API and
their ability to interact with Traffic Ops Data Model objects. A User object
also contains "meta" information such as their email address.

\subsubsection{Address}
This is a text field with no defined structure, but semantically it represents a
user's real-world, physical address. If it is not an empty value, it is assumed
to contain enough information to send a letter through normal postage to the
user.\\
Addresses may consist of alphanumeric characters, hyphens, periods, spaces, and
newlines, but may not end with a space or newline.

\subsubsection{Email}
A user's email address, which is used for initial registration and password
recovery. A User object is not guaranteed to have an Email because the initial,
default User will not have one. However, UIs based on the Traffic Ops API are
encouraged to require that new users have one, as otherwise they will be unable
to recover their account if their password is forgotten.\\
Emails are unique among all Users, unless they are "Null"-valued.\\
Representation formats that do not support email addresses natively as a type
MUST represent this a string.

\subsubsection{Full Name}
A user's "full" name, as it would appear on a letter mailed to them, as a
string. This field, if not empty, is presumed to be the user's name as it would
appear on normal postage. This field MUST NOT be allowed to contain
non-alphabetic characters.

\subsubsection{Phone Number}
A string which, if not empty, is presumed to be a telephone number at which the
user may be contacted. It MUST only be allowed to contain numerics and hyphens,
and must neither start nor end with a hyphen.

\subsubsection{Role}
A user's permissions are encapsulated by their Role, which is represented as a
map containing two keys, "name" that maps to the Role's Name as a string and
"permissions" which maps to a set of Permission Names as strings.

\subsubsection{Tenant}
The scope of a user's permissions is expressed as a string that is the Name of
the Tenant to which the user belongs.

\subsubsection{Username}
A user's "Username" is a string that is used to uniquely identify them among all
Users. It is only allowed to contain alphanumeric characters.
